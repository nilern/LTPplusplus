\documentclass[12pt,a4paper,finnish]{book}

\usepackage[utf8]{inputenc}               % Tekstiasetuksia, sisältää ääkköset
\usepackage[T1]{fontenc}                    % Tekstiasetuksia, T1-koodatut fontit
\usepackage[finnish]{babel}                 % Suomenkielinen tavutus ja otsikot
\usepackage{amsmath}                        % Lisää kaavavoimaa!

\title{Lämpötekniikan perusteita}
\date{\today}
\author{Pauli Jaakkola}

\begin{document}

\maketitle
% Creative Commons, LaTeX-maininnat tähän
\newpage
\pagenumbering{Roman}
\tableofcontents
\newpage
\pagenumbering{arabic}

\chapter{Johdanto}

\part{Perustietoja} % ___________________________________________________________________

\chapter{Kontrollitilavuus ja systeemi} % -----------------------------------------------
% Avoin, suljettu, adiabaattinen

\chapter{Suureet} %---------------------------------------------------------------------
% Tarvittavat prujuista ja kirjoista

\section{Aika ja avaruus}

\subsection{Aika}

\subsection{Pituus, pinta-ala ja tilavuus}

\section{Aine ja energia}

\subsection{Ainemäärä}

\subsection{Massa}

\subsection{Tiheys}

\subsection{Voima}

\subsection{Energia}

\section{Ekstensiivi- ja intensiivisuureet}

\section{Termodynaamiset suureet}

\subsection{Viskositeetti}

\subsection{Lämmönjohtavuus}

\subsection{Terminen diffusiviteetti}

\section{Tilasuureet l. termodynaamiset potentiaalit} 

\subsection{Lämpötila}

\subsection{Paine}

\subsection{Tiheys ja ominaistilavuus}

\subsection{Sisäenergia}

\subsection{Entalpia}

\subsection{Entropia}

\subsection{Helmhotzin vapaaenergia}

\subsection{Gibbsin vapaaenergia}

\subsection{Vapaat tilasuureet l. vapaat muuttujat l. vapausasteet}

\paragraph{Tilanyhtälö}

\section{Suureluettelo}
% Kunkin suureen nimet, symbolit, yksiköt

\chapter{Säilymislait} % ---------------------------------------------------------------

\section{Taselaskenta}
% reynoldsin siirtoteoreema

\section{Massan säilyminen}

\subsection{Suljettu systeemi}

\subsection{Avoin systeemi}

\section{Liikemäärän säilyminen}

\subsection{Suljettu systeemi}

\subsection{Avoin systeemi}

\section{Kulmaliikemäärän säilyminen}

\subsection{Suljettu systeemi}

\subsection{Avoin systeemi}

\section{Energian säilyminen}

\subsection{Suljettu systeemi}

\subsection{Avoin systeemi}

\chapter{Termodynamiikan pääsäännöt} % -------------------------------------------------

\section{Nollas pääsääntö: lämpötilan määritelmä}

\section{Ensimmäinen pääsääntö: energian säilyminen}

\subsection{Systeemin energia ja siirtyvä energia}

\subsection{Suljettu systeemi}

\subsection{Avoin systeemi}

\section{Toinen pääsääntö: epäjärjestyksen lisääntyminen}

\subsection{Energian laatu}

\subsection{Suljettu systeemi}

\subsection{Avoin systeemi}

\section{Kolmas pääsääntö: absoluuttinen nollapiste}

\subsection{Suljettu systeemi}

\subsection{Avoin systeemi}

\chapter{Muut lait} % ------------------------------------------------------------------
% esim. fourierin "laki"

\part{Termodynamiikkaa}% ________________________________________________________________

\chapter{Termodynaamiset prosessit}
% Huomaa tilasuureiden ja muiden ero (Q ja W eivät potentiaaleja)
% Työn ja lämmöntuonti vakiopaineessa, -tilavuudessa, -lämpötilassa, entropiassa...
% Polytrooppiprosessi, mielivaltaiset prosessit

\chapter{Aineen olomuodot ja faasimuutokset}
% Kiinteä, neste, kaasu, ylikrittinen, plasma(...:P)
% Faasidiagrammit

\part{Virtausoppia} % ___________________________________________________________________

\chapter{Virtausaineen käsite}

\part{Lämmönsiirtoa} % ___________________________________________________________________

\part{Matemaattisia menetelmiä} % ________________________________________________________

\chapter{Differentiaalilaskenta} % -------------------------------------------------------
% Osittais- ja kokonaisdifferentiaalit

\section{Derivointi}
% Osittais- ja kokonaisderivaatat

\section{Integrointi}

\section{Differentiaaliyhtälöt}

\chapter{Vektorianalyysi} % --------------------------------------------------------------

\section{Skalaari- ja vektorikentät}

\section{Aikaderivoidut suureet}
% ``Virrat``, totaalinen derivaatta 

\chapter{Dimensioanalyysi} % -------------------------------------------------------------

\section{Dimensiottomat suureet}

\section{Lämpöteknisiä dimensiottomia suureita}
% Suureet-osioon?

\part{Laskuharjoitusten ratkaisuja} % ____________________________________________________

\end{document}

