% Copyright 2013 Pauli Jaakkola
% 
% This program is free software: you can redistribute it and/or modify
% it under the terms of the GNU General Public License as published by
% the Free Software Foundation, either version 3 of the License, or
% (at your option) any later version.
% 
% This program is distributed in the hope that it will be useful,
% but WITHOUT ANY WARRANTY; without even the implied warranty of
% MERCHANTABILITY or FITNESS FOR A PARTICULAR PURPOSE.  See the
% GNU General Public License for more details.
% 
% You should have received a copy of the GNU General Public License
% along with this program.  If not, see <http://www.gnu.org/licenses/>.

\documentclass[12pt,a4paper,finnish]{book}

\usepackage[utf8]{inputenc}                 % Tekstiasetuksia, sisältää ääkköset
\usepackage[T1]{fontenc}                    % Tekstiasetuksia, T1-koodatut fontit
\usepackage{ae,aecompl}                     % Paremman näköiset fontit
\usepackage[finnish]{babel}                 % Suomenkielinen tavutus ja otsikot
% \usepackage{a4wide}
\usepackage{shorttoc}
\usepackage[intlimits]{amsmath}             % Lisää kaavavoimaa!
\usepackage{fixltx2e}                       % \textsubscript
\usepackage{enumerate}
\usepackage{framed}
\usepackage{hyperref}
\makeatother
\hypersetup{
  colorlinks=true,
  linkcolor=blue,
  urlcolor=blue
}



\title{LTP++\\Virtausopin perusteet}
\date{\today}
\author{Pauli Jaakkola}

\begin{document}

\maketitle
% GPL, LaTeX-maininnat tähän
\newpage
\pagenumbering{Roman}
\shorttoc{Sisältö lyhyesti}{0}
\setcounter{tocdepth}{3}
\tableofcontents
\newpage
\pagenumbering{arabic}
\setcounter{secnumdepth}{3}

\chapter*{Johdanto} %------------------------------------------------------------------------------------------
\addcontentsline{toc}{chapter}{Johdanto}
\renewcommand{\thesection}{\arabic{section}}

\section{Lämpötieteet ja lämpötekniikka} %..............................................

On helppo ajatella suoraviivaisesti, että \textbf{tieteet} lähtevät liikkeelle kiinnostuksesta johonkin 
luonnossa tapahtuvaan \textbf{ilmiöön}. 
Sitten tehdään \textbf{perustutkimusta} -- laaditaan \textbf{teorioita} ja \textbf{testataan niitä kokeellisesti}. 
Lopulta kun teoria on riittävän \textbf{yleispätevä}, joku käyttää sitä ja luovuuttaan teknisen tai muun \textbf{käytännön 
sovelluksen} luomiseen. Ollaan edetty \textbf{soveltavaan tutkimukseen}.

\textbf{Lämpötieteet}, kuten tässä kirjassa käsiteltävät

\begin{itemize}
 \item Termodynamiikka
 \item Virtausoppi
 \item Lämmönsiirto
\end{itemize}

ovat kuitenkin suurelta osin ns. \textbf{teknisiä tieteitä} eli insinöörien työkaluja. Ne ovat 
syntyneet pikemminkin tutkittaessa, miten \textbf{lämpöteknisiä} sovelluksia, kuten 

\begin{itemize}
 \item Lämpövoimakoneita (voimalaitokset)
 \item Lämpöpumppuja (jäähdytys ja lämmitys)
 \item Virtauskoneita (pumppuja, puhaltimia, kompressoreja ja turbiineja)
 \item Lämmönsiirtimiä (monien prosessien osana)
\end{itemize}

voitaisiin parantaa. Vaikka lämpötieteet ovat sittemmin monin osin kehittyneet maailmaa syleilevän yleispäteviksi, 
niiden keskeisin tai ainakin hyvödyllisin sovellusalue on edelleen juuri lämpötekniikka.

\section{Suureet, luonnonlait ja kaavat; ymmärrys ja tulokset} %...............................................

Mielikuvamme luonnontieteistä ja teknisistä tieteistä on usein sellainen, että ne koostuvat pääosin \textbf{kaavoista}. 
Itse asiassa kaavat ovat kuitenkin vain korkealle abstraktiotasolle jalostettuja yhteenvetoja siitä ymmärryksestä, 
joka on saavutettu \textbf{teoreettisten mallien} ja \textbf{kokeellisen tutkimuksen} vuorovaikutuksessa.
Kaavoja suositaan luonnontieteissä myös sen takia, että ne ovat \textbf{kvantitatiivisia} tieteitä, jotka 
pyrkivät tarkkuuteen ja yksiselitteisyyteen eli \textbf{eksaktiuteen}\footnote{Luonnontieteilijät pitävät 
joskus -- tai useinkin -- itseään jotenkin ihmistieteilijöitä parempina tällä perusteella. Tämä näkyy 
teekkarien ja humanistien välisessä vastakkainasettelussa mutta myös siinä, että englannin kielen tiedettä 
tarkoittava sana \textit{science} voi yksinään tarkoittaa nimenomaan luonnontiedettä, jopa erotuksena 
ihmistieteistä. Todellisuudessa luonnontieteiden kvantitatiivisuus ja eksaktius johtuu kuitenkin siitä, että 
tarkasteltavat ilmiöt ovat oikeastaan hyvin yksinkertaisia verrattuna vaikkapa ihmisen käyttäytymiseen.}

Useimmiten kaavat kertovat joidenkin \textbf{suureiden} välisen yhteyden matemaattisessa muodossa. Yhteys sinänsä 
saattaa olla syvällinen ajatus, jopa \textbf{luonnonlaki} -- monet kaavat kuvaavat esimerkiksi energian säilymistä:

\begin{align}
 \frac{\dot{Q} + \dot{W}}{\dot{m}} &= \Delta\left(h + \frac{1}{2}\vec{V}^2 + gz\right)\\
 \rho c_v\frac{dT}{dt} &= k\nabla^2T + \dot{Q}
\end{align}

Kuitenkin ehkä suurempi osa luonnontieteen oivalluksista on käsitteellistetty itse suureisiin. Monet kaavatkin 
ovat itse asiassa vain suureiden määritelmiä:

\begin{align}
 H &= U + pV\\
 G &= H - T\Delta S 
\end{align}

Niinpä jos tämän kirjan ``punainen lanka'' ovat luonnonlait, niin ehkä suureet ovat toinen yhtälailla tärkeä
``vihreä lanka''\footnote{Myös erään puolueen lehti. Tämän alaviitteen tarkoitus on kuitenkin huomauttaa, 
ettei tässä ole kyse tuotesijoittelusta tai poliittisesta propagandasta.}. Kaavat ovat toki tärkeitä, sillä 
niillä saadaan \textbf{tuloksia}, mutta vasta niiden taustalla vaikuttavien suureiden ja luonnonlakien 
ymmärtäminen mahdollistaa \textbf{luovuuden}. Tai edes oikeiden kaavojen käytön oikeassa tilanteessa ja siten 
tulosten \textbf{oikeellisuuden}.

\section{Miksi nämä tieteet?}

Lämpötieteellisten ilmiöiden ja -teknisten laitteiden analyysi on käytännössä useimmiten monitieteellistä. 
Mietitäänpä vaikkapa \textbf{lämmönsiirrintä}, joka siirtää \textbf{lämpötehoa $\dot{Q}$} vesivirtauksesta $a$ 
vesivirtaukseen $b$. Seuraavassa esiintyviä kaavoja ei tietenkään tarvitse tässä vaiheessa vielä ymmärtää.

Ensinnäkin meitä tietenkin kiinnostaa lämmönsiirron suunta. \textbf{Termodynamiikan toisen pääsäännön} mukaan 
lämpö siirtyy spontaanisti\footnote{``itsestään, luonnostaan''} korkeammasta lämpötilasta matalampaan. Tämä on 
kokeellinen havainto, mutta klassinen termodynamiikka selittää sen niin, että \textbf{entropian} täytyy kasvaa. 
Tilastollinen termodynamiikka selittää, miksi näin on. Matemaattisesti:

\begin{align}
 &T_a > T_b\\
 &\Rightarrow \dot{Q}_a < 0\\
 &\Rightarrow \dot{Q}_b > 0
\end{align}

Kun lämmönsiirron suunta on nyt selvillä, meitä tietenkin kiinnostaa kummankin vesivirran lämpötilan muutos. 
\textbf{Termodynamiikan ensimmäisen pääsäännön} mukaan energia säilyy eli virtausten entalpiat muuttuvat 
lämmön verran. Oletetaan että kaikki lämpö siirtyy $a$:sta $b$:hen (eikä esim. lämmönsiirtimen rakenteisiin):

\begin{align}
 &\dot{Q}_b = -\dot{Q}_a\\
 &\Delta \dot{H_a} = \dot{Q_a}\\
 &\Delta \dot{H_b} = \dot{Q_b}
\end{align}

Käyttämällä entalpiavirran ja ominaisentalpian ($\dot{H} = \dot{m}h$) sekä ominaisentalpian ja lämpötilan 
($\Delta h = c_p \Delta T$) välisiä yhteyksiä saadaan virtausten lämpötilojen muutoksen lämmönsiirtimessä:

\begin{align}
 &\Delta T_a = \frac{\dot{Q_a}}{c_p\dot{m}_a}\\
 &\Delta T_b = \frac{\dot{Q_b}}{c_p\dot{m}_b}
\end{align}

Tässä vaiheessa ongelmaksi tulee tietenkin sen määrittäminen, miten suuri siirtyvä lämpöteho on. Tähän 
tarvitaan \textbf{lämmönsiirtoa}. Lämpö siirtyy virtauksissa \textbf{konvektiolla} ja \textbf{johtumalla} 
ja putkien läpi johtumalla. Lämmön johtumisen teoria on melko yksinkertainen ja tarkka. Konvektiivisesta 
lämmönsiirrosta saadaan kohtuullinen arvio dimensiottomien lukujen avulla ilmaistuilla kokeellisilla 
korrelaatioilla.

Konvektiivisen lämmönsiirron tarkempi määrittäminen vaatisi \textbf{virtausopin} tuntemusta. Sitä 
tarvitaan myös sen määrittämiseen, miten suuren \textbf{mekaanisen tehon} $\dot{W}$ virtauksen 
pumppaaminen lämmönsiirtimen läpi vaatisi.

Nämä kolme ovat siis keskeisimmät lämpötekniikassa tarvittavat tieteet. Tietenkään poikkitieteellisyys 
ei välttämättä lopu vielä tähän. Esimerkiksi lämmönsiirtimen rakenteiden mitoittamiseen 
lämpötilaeroista johtuvat mekaaniset rasitukset kestäviksi tarvittaisiin lujuuslaskentaa. Usein 
voimalaitoksissa lämpö saadaan joko polttoprosessista tai ydinreaktiosta, joiden analysoimiseen 
tarvitaan fysikaalista kemiaa tai ydinfysiikkaa jne.

\section{Merkinnöistä} %................................................................

Lämpötieteiden kirjallinen perinne on vanha ja julkaisujen määrä valtava. Tämän seikan valossa on täysin 
ymmärrettävää, että \textbf{käytetyt merkinnätkin vaihtelevat melkoisesti}:

\begin{itemize}
 \item Esimerkiksi $q$:lla voidaan merkitä ominaislämpöä ($J/kg$), lämpövirran tiheyttä ($W/m^2$) tai jopa 
  tilavuusvirtaa ($m^3/s$). 
 \item Samaten $u$, $v$ ja $h$ voivat merkitä sisäenergiaa, ominaistilavuutta ja entalpiaa -- tai 
  sitten nopeusvektorin x- ja y-komponentteja sekä lämmönsiirtokerrointa.
\end{itemize}
  
Tämän tilanteen syntyä on edesauttanut myös se, että jo lämpötieteiden sisällä -- saati sitten fysiikassa 
yleensä -- on käytössä niin monta suuretta, että latinalaiset tai kreikkalaisetkaan aakkoset eivät tahdo riittää.

Kun teoria on hyvin hallussa, suureet menevät harvoin sekaisin sekalaisista merkinnöistä huolimatta. 
Laskentatilanteesta, kaavojen muodosta ja yksiköistä näkee, mistä suureista on kyse. Mutta tätä kirjaa 
lukevat ainakin toivottavasti ne, joilla teoria ei ole vielä juuri ollenkaan hallussa.

Niinpä \textbf{olen pyrkinyt yksiselitteiseen merkintätapaan}, jossa eri suureita ei merkitä samalla merkinnällä. 
Mikäli eri kirjaimen käyttäminen olisi täysin yleisen käytännön vastaista käytän vektorimerkkejä 
tai aikaderivaattoja  suureiden erottelemiseksi:
 
\begin{itemize}
 \item ominaistilavuus $v$, vauhti eli nopeusvektorin pituus $|\vec{v}|$
 \item tilavuus $V$, tilavuusvirta $\dot{V}$
\end{itemize}

\renewcommand{\thepart}{\arabic{part}}
\setcounter{part}{-1}
\part{Suureita} % _____________________________________________________________________________________________
\setcounter{chapter}{0}
\renewcommand{\thesection}{\arabic{chapter}.\arabic{section}}

Ennen kuin alamme varsinaisesti käsitellä termodynamiikkaa tai muitakaan lämpötieteitä on syytä palauttaa mieleen 
muutama perussuure yksiköineen ja määritelmineen. Luultavasti suureet ovat ennestään tuttuja etkä halua käyttää 
niihin juurikaan aikaa mutta perusteelliseen ymmärrykseen on hyvä pyrkiä -- pidemmällä tähtäimellä sitä kautta 
pääsee vähemmällä. Ja mistäpä muualta perusteellinen ymmärrys lähtisi kuin perusteista, perusasoista.

% \section{Puhtaista luvuista vektoreihin}
% 
% \begin{itemize}
%  \item \textbf{Puhtailla luvuilla} ei ole yksikköä. Meillä voi olla esimerkiksi 2 pakastinta, 3 turbiinia ja 1 lämmönvaihdin.
%  \item \textbf{Skalaarisuureilla} on yksikkö eli suuruus: pituudella metri, massalla kilogramma jne.
%  \item \textbf{Vektorisuureilla} on suuruus ja suunta, esimerkiksi voiman yksikkö on newton mutta on välttämätöntä 
%  tietää myös mihin suuntaan voima vetää tai työntää.
% \end{itemize}

% kunkin suureen kohdalla, onko se mikä näistä!

\chapter{Perussuureita} %--------------------------------------------------------------------------------------------

\section{Avaruus ja aika} % ................................................................

Lämpötieteissä ulottuvuuksia käsitellään klassisen fysiikan tapaan eli avaruusulottuvuuksia on kolme ja ne ovat 
toisistaan sekä ajan yhdestä ulottuvuudesta erillisiä. Syitä tähän on pohjimmiltaan kaksi:

\begin{enumerate}
 \item Lämpötieteet syntyivät ennen suhteellisuusteoriaa ja muuta modernia fysiikkaa.
 \item Käytännön sovelluksissa on harvinaista joutua käsittelemään tilanteita joissa tarvittaisiin 
 suhteellisuusteorian aika-avaruutta\footnote{Miksi tämä on englanniksi ``spacetime'' ja suomeksi ``aika-avaruus''?}
 tai ylimääräisiä avaruusulottuvuuksia.
\end{enumerate}

\subsection{Avaruus}

\subsubsection{Pituus L}

\textbf{Pituuden} (usein \textbf{L}) yksikkönä käytetään SI-perusyksikkö \textbf{metriä}:

\begin{equation}
 [L] = m
\end{equation}

(``Metri on sellaisen matkan pituus, jonka valo kulkee tyhjiössä aikavälissä 1/299 792 458 sekuntia 
(17. CGPM, 1983).'')

\subsubsection{Pinta-ala A}

\textbf{Pinta-alan} (usein \textbf{A}) yksikkönä käytetään \textbf{neliömetriä}:

\begin{equation}
 [A] = m^2
\end{equation}

(Suorakaiteen pinta-alaa voi kuvata kertomalla sen sivujen pituudet toisillaan. Koska minkä muotoisen tasokuvion 
tahansa voi ajatella muodostuvan esimerkiksi mielivaltaisen pienistä neliöistä, on neliömetri pätevä mittaamaan 
mielivaltaisen muotoisia pinta-aloja).

\subsubsection{Tilavuus V}

\textbf{Tilavuuden} (usein \textbf{V}) yksikkönä käytetään \textbf{kuutiometriä}:

\begin{equation}
 [V] = m^3
\end{equation}

(Suorakulmaisen särmiön tilavuutta voi kuvata kertomalla sen sivujen pituudet toisillaan. Koska minkä muotoisen avaruuskappaleen 
tahansa voi ajatella muodostuvan esimerkiksi mielivaltaisen pienistä kuutioista, on kuutiometri pätevä mittaamaan 
mielivaltaisen muotoisia tilavuuksia).

\subsubsection{Yksiulotteinen sijainti s}

Ensimmäisenä on syytä mainita, että jos tilannetta voidaan kuvata yksiulotteisena\footnote{ajattele vaikkapa 
raiteillaan pysyvää junaa}, käytetään joskus ainoana avaruuskoordinaattina \textbf{sijaintia s}.

\subsubsection{Kolmiulotteinen sijainti \={r}}

Avaruuden ulottuvuuksia on siis kolme ja ne muodostavan kolmiulotteisen avaruuden. 

Mikä tahansa piste tässä avaruudessa voidaan määrittää \textbf{kolmen koordinaatin} avulla. 
Ensin koordinaatit täytyy kalibroida määrittämällä niille nollakohdat sekä yksiköt. ``Kartesiolainen'' eli 
suorakulmainen $(x, y, z)$-koordinaatisto on yleisin, mutta lämpötieteissä eivät ole erityisen harvinaisia tilanteet 
joissa esimerkiksi sylinterikoordinaatisto $(z, r, \theta)$ tai pallokoordinaatisto $(r, \theta, \phi)$ on kätevämpi.

\textbf{Origo} sijaitsee koordinaattien nollakohtien leikkauspisteessä $(0, 0, 0)$. Minkä tahansa pisteen sijainti 
voidaan ilmoittaa \textbf{paikkavektorilla $\vec{r}$} origosta kyseiseen pisteeseen. Kartesiolaisessa koordinaatistossa

\begin{equation}
 \vec{r} = \begin{bmatrix}
            x\\ y\\ z
           \end{bmatrix}
\end{equation}

\subsection{Aika t}

\textbf{Ajan t} yksikkönä käytetään SI-perusyksikkö \textbf{sekuntia}:

\begin{equation}
 [t] = s
\end{equation}

(``Sekunti on 9 192 631 770 kertaa sellaisen säteilyn jaksonaika, joka vastaa cesium 133 -atomin siirtymää 
perustilan ylihienorakenteen kahden energiatason välillä (13. CGPM, 1967).'')

% Kenttäsuureet

\section{Aineen määrä} % ..............................................................

\subsection{Ainemäärä n}

\textbf{Ainemäärä n} kertoo kuinka monta kappaletta jotain hiukkasta (yleensä molekyylia) on. Se on siis itse 
asiassa puhdas luku, mutta koska yleensä käsitellään niin suuria molekyylimääriä, on sille määritetty 
SI-perusyksikkö \textbf{mooli}:

\begin{equation}
 [n] = mol
\end{equation}

``Mooli on sellaisen systeemin ainemäärä, joka sisältää yhtä monta keskenään samanlaista perusosasta kuin 0,012 
kilogrammassa hiili 12:ta on atomeja. Perusosaset voivat olla atomeja, molekyylejä, ioneja, elektroneja,
muita hiukkasia tai sellaisten hiukkasten määriteltyjä ryhmiä. (14. CGPM, 1971)''

0,012 kilogrammassa hiili-12:ta eli yhdessä moolissa olevien hiukkasten lukumäärä on \textbf{Avogadron luku $N_A$}:

\begin{equation}
 [N_A] \approx (6,02214129 \pm 0,00000027)\cdot10^{23}
\end{equation}

Ainemäärä on hyödyllinen yleensä kemiassa (koska reaktioissa väliä on molekyylien määrällä) ja kaasuja käsiteltäessä 
(koska mm. tilavuudet ja paineet riippuvat molekyylien määristä).

\subsection{Massa m} \label{def:massa}

Mekaniikassa meitä kiinnostaa kuitenkin yleensä pikemminkin se, miten ``painava'' tai ``hidas'' käsiteltävä systeemi 
on. Tätä mitataan \textbf{massalla m}. Klassisessa fysiikassa esiintyy itse asiassa kahdenlaista massaa:

\begin{itemize}
 \item \textbf{Hidas massa} on Newtonin II laissa esiintyvä massa. Se mittaa siis sitä miten suuri voima tarvitaan 
 kappaleen kiihdyttämiseen.
 \item \textbf{Painava massa} taas on Newtonin gravitaatiolaissa esiintyvä massa. Se mittaa siis sitä miten suuren 
 voiman gravitaatiokenttä aiheuttaa kappaleeseen\footnote{Vrt. varaus sähkömagneettisissa kentissä.}.
\end{itemize}

Hidas ja painava massa ovat kuitenkin saman suuruiset, mikä ei ollut klassisen fysiikan teorioiden perusteella  
mitenkään itsestään selvää. Kuitenkin jo Galileo Galilei huomasi kokeellisesti, että kaikkien kappaleiden 
kappaleen \textbf{putoamiskiihtyvyys g} on sama. Näin voi olla vain, mikäli hidas ja painava massa ovat yhtä suuret.

% Todistus

Suppea suhteellisuusteoria pätee vain vakionopeudella liikkuville koordinaatistoille (arkisemmin ``tarkkailijoille''). 
Se sai alkunsa sähkömagneettisten aaltojen teoriassa tehdystä havainnosta että valon nopeus tyhjiössä on 
koordinaatiston nopeudesta riippumaton vakio.

Yleinen suhteellisuusteoria pätee myös kiihtyvässä liikkeessä oleville koordinaatistoille. Sen perustava oivallus 
oli nimenomaan se, että hidas ja painava massa tuskin ovat sattumalta täsmälleen yhtä suuret. Putoamiskiihtyvyys 
on kiihtyvyys, joka aiheutuu aika-avaruuden kaareutumisesta massan ympärillä.

Massan SI-perusyksikkö on \textbf{kilogramma kg}:

\begin{equation}
 [m] = kg
\end{equation}

``Kilogramma on yhtä suuri kuin kansainvälisen kilogramman prototyypin massa (1. ja 3. CGPM, 1889 ja 1901).''

Kilogramma on ainoa SI-perusyksikkö, joka vielä perustuu tällaiseen prototyyppiin. Tämä on ongelmallista ensinnäkin 
siksi, että prototyyppi ei ole toistettavissa ja toisekseen siksi että - kauhistus sentään - prototyypin massa 
ei mittausten mukaan ole vakio. Alun perin kilogramma piti määritellä ``1 litra vettä on massaltaan kilogramman 
4 $^{\circ}$C:n lämpötilassa''. Vesipohjaiseen määritelmään siirtymistä on myöhemminkin ehdotettu, joskin niin että 
määritelmä vastaisi nykyistä kilogramman määritelmää paremmalla tarkkuudella.

\subsection{Moolimassa M}

Systeemin massa ja ainemäärä riippuvat toisistaan \textbf{moolimassan M} kautta:

\begin{equation}
 M = \frac{m}{n}
\end{equation}

Moolimassan SI-yksiköksi tulee

\begin{equation}
 [M] = \left[\frac{m}{n}\right] = \frac{[m]}{[n]} = \frac{kg}{mol}
\end{equation}

Tämä on kuitenkin niin suuri yksikkö että helpommin käsiteltäviä lukuja saadaan käyttämällä yksikkönä joko 
g/mol (yleisin) tai kg/kmol.

\chapter{Yksinkertaisia johdannaissuureita} % --------------------------------------------------------------------

\section{Aikaderivaattasuureet} % ................................................................

% Suureen muutos / ajan muutos - muutos tietyssä ajassa - aika mieliv. lyhyeksi - aikaderivointi

% Ajan ja tilan kytkentä - vaikka itsenäisiä, muuttuvat yhtä aikaa

\subsection{Nopeus \={v}}

Yksiulotteisen sijainnin muutoksen suhde ajan muutokseen on \textbf{vauhti $|\vec{v}|$}:

\begin{equation}
 v = \frac{\Delta s}{\Delta t}
\end{equation}

Jos vauhti halutaan hetkellisesti eli mielivaltaisen lyhyenä ajanhetkenä tämä lähestyy aikaderivaattaa:

\begin{equation}
 v = \frac{ds}{dt}
\end{equation}

Kun tämä siirretään kolmiulotteiseen avaruuteen paikkavektorin $\vec{r}$ derivaataksi saadaan \textbf{nopeus $\vec{v}$} 
joka on siis myös vektorisuure:

\begin{equation}
 \vec{v} = \frac{d\vec{r}}{dt}
\end{equation}

Aikaderivaattaa on usein tapana merkitä pisteellä:

\begin{equation}
 \vec{v} = \frac{d\vec{r}}{dt} = \dot{\vec{r}}
\end{equation}

Nopeuden yksiköksi tulee sama kuin vauhdinkin eli

\begin{equation}
 [v] = \left[\frac{ds}{dt}\right] = \left[\frac{s}{t}\right] = \frac{m}{s}
\end{equation}

\subsection{Kiihtyvyys \={a}}

Vauhdin muutos ajan suhteen on \textbf{kiihtyvyys $|\vec{a}|$}:

\begin{equation}
 a = \frac{\Delta v}{\Delta t}
\end{equation}

Jos vauhti halutaan hetkellisesti eli mielivaltaisen lyhyenä ajanhetkenä tämä lähestyy aikaderivaattaa:

\begin{equation}
 a = \frac{dv}{dt}
\end{equation}

Kolmiulotteisessa avaruudessa nopeusvektorin $\vec{v}$ derivaataksi saadaan \textbf{kiihtyvyysvektori $\vec{a}$}:

\begin{equation}
 \vec{a} = \frac{d\vec{v}}{dt} = \dot{\vec{v}} = \ddot{\vec{r}}
\end{equation}

Kiihtyvyyden yksikkö on

\begin{equation}
 [a] = \left[\frac{dv}{dt}\right] = \left[\frac{d^2s}{dt^2}\right] = \left[\frac{s}{t^2}\right] = \frac{m}{s^2}
\end{equation}

\subsection{Tilavuusvirta \.{V}}

Kun halutaan tietää, kuinka suuri tilavuus kulkee jonkin pinnan läpi aikayksikössä voidaan 
se määrittää vastaavalla menettelyllä kuin nopeus. Keskimääräinen \textbf{tilavuusvirta} $\dot{V}$ on

\begin{equation}
 \dot{V} = \frac{\Delta V}{\Delta T}
\end{equation}

ja hetkellinen

\begin{equation}
 \dot{V} = \frac{dV}{dt}
\end{equation}

Tilavuusvirran SI-yksikkö on

\begin{equation}
 [\dot{V}] = \left[\frac{dV}{dt}\right] = \left[\frac{V}{t}\right] = \frac{m^3}{s}
\end{equation}

Tämänkaltaisista aikaderivoiduista suureista, jotka eivät ole nopeutta, kiihtyvyyttä eivätkä mekaanista tai
lämpötehoa on tapana käyttää \textbf{virta}-nimitystä.

\subsection{Moolivirta \.{n}}

Pinnan läpi aikayksikössä menevä ainemäärä on \textbf{moolivirta} $\dot{n}$:

\begin{align}
 &\dot{n} = \frac{\Delta n}{\Delta t}\\ 
 &\dot{n} = \frac{dn}{dt}\\
 &[\dot{n}] = \left[\frac{dn}{dt}\right] = \left[\frac{n}{t}\right] = \frac{mol}{s}
\end{align}

\subsection{Massavirta \.{m}}

Pinnan läpi aikayksikössä menevä massa on \textbf{massavirta} $\dot{m}$:

\begin{align}
 &\dot{m} = \frac{\Delta m}{\Delta t}\\ 
 &\dot{m} = \frac{dm}{dt}\\
 &[\dot{m}] = \left[\frac{dm}{dt}\right] = \left[\frac{m}{t}\right] = \frac{kg}{s}
\end{align}

\section{Ekstensiivi- ja intensiivisuureet} %...............................................

\textbf{Ekstensiivisuureet} ovat suureita, joiden arvo riippuu systeemin koosta\footnote{``extent''} eli 
massasta tai ainemäärästä. Tyypillinen ekstensiivisuure on tilavuus $V$. \textbf{Intensiivisuureiden} arvot 
taas eivät riipu systeemin koosta. Tyypillisiä intensiivisuureita ovat paine $p$ ja lämpötila $T$.

\subsection{Ominaissuureet}

Intensiivisuureet ovat siinä mielessä toivottavampia, että niiden käyttö ei vaadi systeemin koon 
selvittämistä tai kiinnittämistä. Niistä saadaan jopa skalaarikenttiä (esim. $T(\vec{r}, t)$). 

Onneksi 
ekstensiivisuureet voidaan muuttaa intensiivisuureiksi jakamalla ne systeemin massalla. Näin 
syntyviä intensiivisuureita kutsutaan \textbf{ominaissuureiksi} ja merkitään vastaavaa ekstensiivisuureen 
suurta vastaavalla pienellä kirjaimella. Esimerkiksi ominaistilavuus on

\begin{align}
 &v = \frac{V}{m}\\
 &[v] = \left[\frac{V}{m}\right] = \frac{m^3}{kg}
\end{align}

ja ominaissisäenergia

\begin{align}
 &u = \frac{U}{m}\\
 &[u] = \left[\frac{U}{m}\right] = \frac{J}{kg}
\end{align}

\subsection{Molaariset ominaissuureet}

Toinen vaihtoehto ekstensiivisuureiden muuntamiseksi intensiivisiksi on niiden jakaminen systeemin 
ainemäärällä sen massan sijaan. Näin saadaan \textbf{molaarisia ominaissuureita}, joita merkitään 
alaindeksillä $m$. Esimerkiksi moolitilavuus on

\begin{align}
 &V_m = \frac{V}{n}\\
 &[V_m] = \left[\frac{V}{n}\right] = \frac{m^3}{mol}
\end{align}

ja molaarinen sisäenergia

\begin{align}
 &U_m = \frac{U}{n}\\
 &[U_m] = \left[\frac{U}{n}\right] = \frac{J}{mol}
\end{align}

\chapter{Monimutkaisempia johdannaissuureita} % ------------------------------------------------------------------

\section{Voima \={F}} % ...........................................................

Olemme tottuneet ajattelemaan voimaa jonkinlaisena perussuureena, mutta itse asiassa se on vain hyvin kätevä 
johdannaissuure, joka on määritelty Newtonin II lain \footnote{Newtonin I laki on II lain erikoistapaus.} perusteella:

\begin{equation}
 \vec{F} = \frac{d\vec{p}}{dt} = \frac{d(m\vec{v})}{dt}
\end{equation}

Niinpä sen yksiköksi tulee:

\begin{equation}
 [F] = [|\vec{F}|] = \left[\frac{d(m|\vec{v}|)}{dt}\right] = \left[\frac{mv}{t}\right] 
  = \left[\frac{mL}{t^2}\right] = \frac{kgm}{s^2}
\end{equation}

Tämä on edelleen nimetty\footnote{ilmeisistä syistä} Newtoniksi:

\begin{equation}
 [F] = \frac{kgm}{s^2} = N
\end{equation}

Voimia ei liene todellisuudessa olemassakaan. Ne ovat vain yksi ihmiskunnan historian hyödyllisimmistä abstraktioista. 
Tämä voiman eksakti muoto kuvaa vain sitä mitä arkikielen voima-sanakin: ``voimaa'' tarvitaan sitä enemmän mitä enemmän 
ja mitä nopeammin materiaa joudutaan kiihdyttämään (tai hidastamaan, $ a < 0$).

\section{Paine p} %....................................................................

\textbf{Paineella} $p$ tarkoitetaan yksinkertaisimmillaan voimaa jaettuna pinta-alalle, jolle se kohdistuu.
Paineen SI-yksikkö on Pascal $Pa$.

\begin{align}
 &p = \frac{|\vec{F}|}{A}\\
 &[p] = \left[\frac{|\vec{F}|}{A}\right] = \frac{N}{m^2} = Pa
\end{align}

Virtausaineissa tilanne ei kuitenkaan ole näin yksinkertainen. Paine voidaan nimittäin määrittää 
mille tahansa virtausaineen reunoilla tai sen sisällä olevalle todelliselle tai kuvitteelliselle pinnalle. 
Itse asiassa virtausopissa paine voidaan (infinitesimaalisten kontrollitilavuuksien $dV$ avulla) määrittää
virtausaineen jokaiselle pisteelle eli $p = p(\vec{r}, t)$).

\section{Työ W} \label{def:W}%......................................................................

Mitä työ on? Mekaniikassa \textbf{työn W} yleinen määritelmä on

\begin{equation}
\label{eq:W}
 W = \int_S \vec{F} \cdot \vec{ds}
\end{equation}

Mitä tämä sitten tarkoittaa? 

Arkisestikin voimme todeta, että jonkin kappaleen siirtämisen ``työläys'' on suoraan verrannollinen

\begin{enumerate}
 \item Voimaan $F$, joka tarvitaan kappaleen liikuttamiseksi
 \item Matkaan $s$, joka kappaletta siirretään
\end{enumerate}

Kun voima on vakio ja reitti koko ajan voiman suuntainen, nämä verrannollisuudet voidaan yhdistää tuloksi ja 
(valitsemalla määritelmässä verrannollisuuskertoimeksi 1) määritellä työ

\begin{equation}
 W = |\vec{F}|s
\end{equation}

Yleisessä tapauksessa ei päästä näin helpolla, sillä kappaleeseen vaikuttavan voiman suuruus ja suunta voivat 
riippua esimerkiksi ajasta ja kappaleen paikasta eikä reittikään ole välttämättä lähelläkään suoraa. 

Onneksi 
mikä tahansa infinitesimaalisen lyhyt reitin pätkä $\vec{ds}$ voidaan katsoa hyvällä tarkkuudella suoraksi ja 
voiman reitin suuntainen komponentti saadaan pistetulolla eli

\begin{equation}
 dW = \vec{F} \cdot \vec{ds}
\end{equation}

Kun nämä infinitesimaalisen lyhyet reitin pätkät sitten summataan eli integroidaan saadaan työn yleinen määritelmä 
\ref{eq:W}.

Työn määritelmä voitaisiin avata sanallisesti vaikka seuraavasti:

\begin{quotation}
``Kun kappale, johon voima $\vec{F}$ vaikuttaa, kulkee 
reitin $S$ tekee voima kaavasta \ref{eq:W} laskettavissa olevan määrän työtä.''
\end{quotation}

 Huomioi, että tämä ei vaadi, että 
juuri voima $\vec{F}$ aiheuttaisi kappaleen liikkeen.

\section{Teho \.W}

Jossain ajassa tehty työ tai hetkellisenä työn aikaderivaatta on \textbf{teho} $\dot{W}$. 
Tehon SI-yksikkö on Watti $W$.

\begin{align}
 &\dot{W} = \frac{\Delta W}{\Delta t}\\ 
 &\dot{W} = \frac{dW}{dt}\\
 &[\dot{W}] = \left[\frac{dW}{dt}\right] = \left[\frac{W}{t}\right] = \frac{J}{s} = W
\end{align}

Tehosta käytetään useimmiten merkintää $P$. Itse käytän kuitenkin merkintää $\dot{W}$ sekaannusten 
välttämiseksi paineen kanssa, jota joskus myös merkitään pienen sijaan isolla p:llä\footnote{
Erityisesti paine ei ole ominaisteho $p \neq \frac{P}{m}$.}. Toisaalta näin 
korostan myös tehon yhteyttä työhön (lämmön sijasta).

\part{Virtausoppi} %__________________________________________________________________________________

\chapter{Virtausoppi mekaniikan alana}
\textbf{Virtausoppi} (``Fluid Mechanics'') on mekaniikan ala, joka käsittelee virtausaineita.

\section{Virtausaine}
%Fluidi, (neste). Leikkausjännitys-määritelmä. Muu kuin kiinteä (neste, kaasu, ylikriittinen [neste], plasma)

Virtausaine (``fluid'') on mikä tahansa aine, joka virtaa. Eksaktimmin tämä voidaan ilmaista niin, 
että 'virtausaineet deformoituvat jatkuvasti leikkausjännityksen vaikutuksesta'. Käytännössä esiintyvistä 
aineen olomuodoista virtausaineita ovat \textbf{nesteet} ja \textbf{kaasut}. Plasmatkin ovat toki 
virtausaineita mutta toistaiseksi harvinaisia käytännön sovelluksissa.

\section{Partikkeli- ja kontinuumimekaniikka}
Teknillinen mekaniikka voidaan jakaa partikkeli- ja jatkumomekaniikkaan. 

\textbf{Partikkelimekaniikassa} oletetaan, 
että massa\footnote{Tässä tarkoitetaan lähinnä \textit{hidasta massaa}, ks. \nameref{def:massa}.} on 
keskittynyt \textit{partikkeleiksi} eli pistemassoiksi. Nämä pistemassat voivat sitten muodostaa 
suurempia systeemejä, joista hyödyllisin lienee \textit{jäykkä kappale}.

\textbf{Kontinuumimekaniikassa} taas ajatellaan, että massa on jakautunut avaruuteen jatkuvasti. 
Tätä voidaan tietenkin kuvata tiheyskentällä \(\rho(\vec{r}, t)\). Tällöin muutkin ominaisuudet kuten 
esimerkiksi nopeus \(\vec{v}(\vec{r}, t)\), paine \(p(\vec{r}, t)\) tai lämpötila \(T(\vec{r}, t)\) 
ovat jatkuvia vektori- tai skalaarikenttiä, siis paikan (1-3 ulottuvuudessa) ja mahdollisesti 
ajan funktioita. Niinpä virtausopin teoria nojaa vahvasti \textbf{differentiaali- ja integraalilaskentaan} 
ja \textbf{vektorianalyysiin}.

Nykyään tiedämme, että aine ei ole mikrotasolla jatkuvaa. Virtausopin teoria on tätä tietoa vanhempi. 
Vaikka kontinuumimekaniikka ei nykytietämyksen mukaan olekaan fysikaalisesti oikeellista, sillä 
saadaan hyviä tuloksia. Aineen epäjatkuvuus nimittäin ilmenee vasta niin pienessä mittakaavassa, 
että jatkumo-oletus toimii vielä makroskooppisesta näkökulmasta differentiaalisina näyttäytyville 
tilavuuksille.

\section{Statiikka ja dynamiikka}
Virtausoppi voidaan jakaa \textbf{hydrostatiikkaan} (``Hydrostatics'') ja \textbf{virtausdynamiikkaan} 
(``Fluid Dynamics''). Koska virtausaineet kuitenkin yleensä virtaavat ja niitä käytetään ja niistä 
ollaan kiinnostuneita juuri sen takia on suurin osa virtausopista käytännössä virtausdynamiikkaa.
\footnote{``Hydrostatiikka''-termin käyttö johtuu varmaankin siitä, että virtausaineiden statiikkaa 
tarvitaan lähinnä erilaisten nestealtaiden ja patojen suunnittelussa. Suomeksi ``virtausstatiikka'' 
on myös itseristiriita (``oksymoroni''), mutta ``Fluid Statics'' ei olisi ollenkaan niin paha.}
Käytännössä englanniksi käytetään yleensä termiä ``Fluid Dynamics'' ja suomeksi ``virtauslaskenta''.
\footnote{Virtausoppi on käytännössä hyvin laskennallista, kuten tulet huomaamaan.}

\subsection{Nopeuskentän keskeisyys}
Yleensä virtausopissa pyritään ratkaisemaan nopeuskenttä \(\vec{v}(\vec{r}, t)\) eli virtausaineen 
nopeusvektori jokaisessa tarkasteltavan tilavuuden pisteessä ja tarkasteltavan ajanjakson 
ajanhetkenä. Nopeuskentän avulla on sitten melko suoraviivaista laskea suureita kuten painekenttä 
\(p(\vec{r}, t)\) tai tilavuus, massa- tai energiavirta (teho) jonkin pinnan läpi.

\chapter{Virtaustyypit}
Virtauksia voidaan jaotella eri tavoin. Kaikki nämä jaottelut ovat tietenkin liukuvia (esimerkiksi 
transitioalue laminaarista turbulenttiin virtaukseen) ja yhdistettävissä (esimerkiksi aluksi on 
helpointa keskittyä \underline{\smash{sisäpuolisen}} \underline{laminaarin} 
\underline{\smash{kokoonpuristumattoman}} \underline{kitkallisen} virtauksen tarkasteluun).

\section{Kitkallinen ja kitkaton}
\textbf{Kitkallisessa virtauksessa} kitka aiheuttaa virtausta vastustavan \textbf{leikkausvoiman},
joka pitää ohuen kerroksen virtausainetta kiinni virtausaineen tilavuutta rajoittavissa seinämissä. 
Tämän kerroksen läheisyydessä olevaan virtausaineeseen aiheutuu myös leikkausvoima, jota vastaan 
virtaus joutuu tekemään työtä. 

Tämä kuluttaa virtauskentän koordinoitunutta liike-energiaa 
molekyylien lämpöliikkeen liike-energiaksi eli sisäenergiaksi jolloin virtausaineen nopeus 
pienenee ja entropia kasvaa. Virtauksen sisäenergiaksi kuluvaa liike-energiaa kutsutaan virtauksen 
\textbf{kitkahäviöksi}. Kitka aiheutuu tietenkin virtausaineen ja myös sen kanssa tekemisissä olevan 
kiinteän aineen molekylien voimavuorovaikutuksista.

\textbf{Kitkattomassa virtauksessa} ei nimensä mukaisesti ole kitkaa, leikkausvoimaa tai kitkahäviöitä. 
Varsinkin koska kitka vaikuttaa myös nopeuskentän muotoon ja sitä kautta suunnilleen kaikkiin tuloksiin, 
kitkaton virtaus on pätevä malli lähinnä \textbf{vapaalle virtaukselle}, joka tapahtuu kaukana 
kaikista pinnoista. Tähän vaaditaan yleensä ulkopuolinen virtaus, mutta rajakerrosten 
ohuuden ansiosta ``kaukana'' voi tarkoittaa yläilmakehän lisäksi vaikkapa ``huoneessa yli sentin 
päässä kiinteistä pinnoista''. Ideaalikaasu olisi määritelmällisesti kitkatonta,
koska sen molekyylit vuorovaikuttavat vain törmäämällä kimmoisasti. Sellaista ei kuitenkaan oikeasti 
ole olemassa.\footnote{Suprajuoksevat (``superfluid'') virtausaineet, kuten 
\href{http://www.youtube.com/watch?v=2Z6UJbwxBZI}{nestemäinen helium} sen sijaan virtaavat kitkattomasti. 
Valitettavasti suprajuoksevuuden saavuttamiseen vaaditaan vielä kylmempiä lämpötiloja kuin suprajohtavuuden.}

\section{Laminaari ja turbulentti}
\textbf{Laminaarin} eli pyörteettömän virtauksen nopeuskenttä muuttuu loivasti ja sulavasti ajan 
ja paikan funktiona\footnote{Niinpä 
\href{http://www.newscientist.com/blogs/nstv/2011/08/born-to-be-viral-how-to-unmix-a-mixed-fluid.html}
{tämä} on mahdollista.} ja kitkahäviöitä aiheutuu ainoastaan leikkausvoimasta. \textbf{Turbulentissa} 
eli pyörteisessä virtauksessa sen sijaan nopeuskenttä vaihtelee voimakkaasti ajan ja paikan funktiona 
moudostaen myös pyörteitä. Tarpeeksi pienten pyörteiden liike-energian voidaan katsoa muuttuvan 
sisäenergiaksi ja kasvattavan täten entropiaa. Turbulenssi aiheuttaa siis ylimääräisiä kitkahäviöitä 
mutta toisaalta tehostaa virtausaineiden sekoittumista ja lämmönsiirtoa.

Turbulentin virtauksen nopeuskentästä pyritään ratkaisemaan sen aikakeskiarvo. Turbulenssin 
vaikutusta nopeuskenttään ja turbulenssihäviöitä kuvataan usein näennäisen leikkausvoiman avulla. 
Turbulenssin mallintaminen eli käytännössä tuon näennäisen leikkausvoiman funktion määrittäminen 
on hyvin haastavaa.

Laminaarin virtauksen muuttuminen turbulentiksi eli \textbf{transitio} ei ole ollenkaan suoraviivaista 
ja tätä toiminta-aluetta pyritään sen mallinnuksen vaikeuden ja käytännöllisen arvaamattomuuden vuoksi 
välttämään suunniteltaessa virtausaineita hyödyntäviä laitteita.

\section{Kokoonpuristuva ja kokoonpuristumaton}
%Hydrauliikka ja kaasudynamiikka}
Virtausaineet ovat tietenkin periaatteessa aina \textbf{kokoonpuristuvia} eli niiden 
tiheys muuttuu paineen funktiona:

\begin{equation}
 \frac{\partial \rho}{\partial p} \neq 0
\end{equation}

Käytännössä kuitenkin nesteiden tiheys muuttuu harvoin merkittävästi paineen funktiona eli

\begin{equation}
 \frac{\partial \rho_{neste}}{\partial p} \rightarrow 0
\end{equation}

Kaasujen tiheys toisaalta muuttuu voimakkaasti paineen funktiona ja tämä on otettava huomioon:

\begin{equation}
 \left|\frac{\partial \rho_{kaasu}}{\partial p}\right| >> 0
\end{equation}

Nesteiden ja kaasujen tyypillisen käyttäytymisen johdosta oppia kokoonpuristumattomasta virtauksesta 
kutsutaan \textbf{hydrauliikaksi}\footnote{kreikan kielestä ``hydor'' = vesi} (``Hydraulics'') ja 
kokoonpuristuvasta \textbf{kaasudynamiikaksi} (``Gas Dynamics'').

\section{Sisäpuolinen ja ulkopuolinen}
\textbf{Sisäpuolinen virtaus} on virtausaineen liikettä jossakin usealta puolelta rajatussa kanavassa. 
Tyypillisesti nämä ovat jonkinlaisia putkia. Eksaktimmin ja matemaattisen käsittelyn kannalta 
sisäpuolinen virtaus on virtausta, jossa kitkan ja mahdollisesti turbulenssin aiheuttama 
leikkausvoima vaikuttaa suurimpaan osaan virtauksen nopeuskentästä.

\textbf{Ulkopuolisessa virtauksessa} virtausaine pääsee liikkumaan enimmäkseen vapaasti. Mikäli 
seinämiä esiintyy, ne aiheuttavan leikkausvoiman kautta vaikutusta virtauksen nopeuskenttään 
vain (yleensä hyvin ohuessa) \textbf{rajakerroksessa}. 

%Newtoninen ja epänewtoninen?

\chapter{Käsittelytavat}
\section{Kontrollitilavuus}
\section{Kenttäteoria}
\subsection{Nopeuskentän keskeisyys}
\section{Dimensioanalyysi}
\section{Teoria, kokeellisuus ja tietokoneet}

\chapter{Säilymislait}
\section{Noetherin teoreema}
\section{Massan säilyminen}
\section{Liikemäärän säilyminen}
\section{Kulmaliikemäärän säilyminen}
\section{Energian säilyminen}

\chapter{Pintavoimat ja tilavuusvoimat}
\section{Pintavoimat}
\subsection{Kitka}
\subsubsection{Leikkausjännitys}
\subsubsection{Nopeuden reunaehto}
\subsubsection{Viskositeetti}
\section{Tilavuusvoimat}
\subsection{Painovoima}
\subsection{Sähköiset voimat}

\chapter{Kontrollitilavuus}

Kontrollitilavuuden menetelmissä tarkastelemme virtauksen suureita äärellisen kokoisessa\footnote{siis ei 
äärettömän suuressa muttei myöskään olemattomassa tai diffrentiaalisen pienessä} kontrollitilavuudessa.

Yleisessä tapauksessa kontrollitilavuuden koko ja muoto ovat ajasta riippuvia. Eksaktimmin tämä voidaan sanoa niin, 
että kontrollitilavuus $V$ ja sen kontrollipinta $S$ ovat ajan funktioita:

\begin{align}
 &V = V(t)\\
 &S = S(t)
\end{align}

Kontrollitilavuutta voidaan tietenkin ajatella avoimena systeeminä, jolla on erinäisiä ekstensiivisuureita, 
esimerkiksi massa $m$ ja entalpia $H$. Haluamme tietenkin pysyä perillä siitä, mitkä näiden ekstensiivisuureiden 
ja niitä vastaavien intensiivisuureiden arvot ovat milläkin ajanhetkellä. Olkoon $B$ mielivaltainen 
kontrollitilavuuden ekstensiivisuure\footnote{Kuten pian näemme, $B$ voi olla myös vektori.} ja $b$ sitä vastaava 
intensiivisuure ja nämä ajan funktioita:

\begin{equation}
 dB(\vec{r}, t) = b(\vec{r}, t)dm(\vec{r}, t) = b(\vec{r}, t)\rho(\vec{r}, t)dV(\vec{r}, t)
\end{equation}

Eli differentiaaliselle tilavuuselementille kohdassa $\vec{r}$ ajanhetkellä $t$ ekstensiivisuureen $B$ arvo on 
intensiivisuureen $b$ arvo tuossa kohdassa tuolloin kerrottuna tuon differentiaalisen tilavuuselementin massalla.
Differentiaalinen massa on vuorostaan tiheys tuossa kohdassa tuolloin kerottuna differentiaalisen tilavuuselementin
tilavuudella. 
Koska meitä kuitenkin kiinnostaa koko kontrollitilavuuden $B$, laskemme sen integroimalla:

\begin{equation}
 B(t) = \int_{B(t)} dB(\vec{r}, t) = \iiint_{V(t)} b(\vec{r}, t)\rho(\vec{r}, t)dV(\vec{r}, t)
\end{equation}

Voimme tutkia ekstensiivisuureiden aikariippuvuutta aikaderivoimalla ne:

\begin{equation}
 \label{eq:dBdt}
 \frac{dB(t)}{dt} = \frac{d}{dt}\iiint_{V(t)} b(\vec{r}, t)\rho(\vec{r}, t)dV(\vec{r}, t)
\end{equation}

Tästä on ennen kaikkea se hyöty, että erilaiset säilymislait on helppo ilmaista matemaattisesti aikaderivaattojen avulla. 
Jatkossa kirjoitan selkeyden vuoksi ``$b(\vec{r}, t)$'':n sijaan vain $b$, mutta pidä mielessäsi suureiden 
paikka- ja aikariipppuvuus.

\subsection{Reynoldsin siirtoteoreema}

Nyt meillä on enää sellainen ongelma, että sekä integroitavat funktiot, että kontrollitilavuus, jonka 
yli integroidaan ovat ajan funktioita. Olisi sekä käsitteellisesti että laskennallisesti kätevää, jos 
saisimme kontrollitilavuuden $V(t)$ muodon ja koon sekä intensiivisuureen $b(\vec{r}, t)$ muutosten vaikutukset 
intensiivisuureen $B(t)$ arvoon eriytettyä omiksi termeikseen. Tämä onnistuu \textbf{Reynoldsin siirtoteoreeman} 
avulla. Sen mukaan:

\begin{framed}
 \begin{equation}
    \label{eq:RTT}
    \frac{dB(t)}{dt} = \frac{d}{dt}\iiint_{V(t)} b\rho dV
      = \iiint_{V(t)}\frac{d( b\rho)}{dt}dV 
	+ \iint_{S(t)} b\rho(\vec{v}_s \cdot \vec{n})dA
 \end{equation}
\end{framed}

Missä $\vec{v}_s(\vec{r}, t)$ on kontrollipinnan nopeus ja $\vec{n}(\vec{r}, t)$ siitä ulospäin osoittava 
yksikkönormaali.

Reynoldsin siirtoteoreema on tietenkin matemaattisesti todistettu. Toisaalta se on varsin intuitiivinen tulos;
$B$:n muutos on sen muutoksen kontrollitilavuudessa ja kontrollipinnan ylittävän $B$:n summa

\section{Säilymislait kontrollitilavuudelle}

Makroskooppiselle systeemille massan, liikemäärän, kulmaliikemäärän ja energian säilymislait ovat

\begin{align}
 \label{eq:V-sailymislait1}
 &\frac{dm(t)}{dt} = 0\\
 &\frac{d\vec{p}(t)}{dt} = \sum \vec{F}\\
 &\frac{d\vec{L}(t)}{dt} = \sum \vec{M}\\
 \label{eq:V-sailymislait4}
 &\frac{dE(t)}{dt} = \dot{Q} + \dot{W}
\end{align}

Massa $m$ ei siis muutu\footnote{Oletamme edelleen että ydinreaktioita ei tapahdu.}. Liikemäärän $\vec{p}$ muutoksen 
aiheuttaa voima $\vec{F}$, joka on itse asiassa määritelty liikemäärän aikaderivaatan suuruiseksi. Täysin vastaavasti 
kulmaliikemäärän $\vec{L}$ muutos on momentin $\vec{M}$ suuruinen. Termodynamiikasta tuttu energian säilyminen, 
$\Delta E = Q + W$ eli systeemin energian muutos on työn ja lämmön summa, sulautuu joukkoon aikaderivoidussa muodossaan.

Systeemin massa $m$, liikemäärä $\vec{p}$, kulmaliikemäärä $\vec{L}$ ja energia $E$ ovat kaikki ekstensiivisuureita. 
Niitä vastaavat ``intensiivisuureet'' ovat

\begin{align}
 &\frac{m}{m} = 1\\
 &\frac{\vec{p}}{m} = \frac{m\vec{v}}{m} = \vec{v}\\
 &\frac{\vec{L}}{m} = \frac{\vec{r} \times m \vec{v}}{m} = \vec{r} \times \vec{v}\\
 &\frac{E}{m} = e 
\end{align}

Yllä ``intensiivisuureet'' on heittomerkeissä, koska näin voidaan kyllä matemaattisessa mielessä sanoa, mutta 
fysikaalisesti emme miellä nopeutta tai varsinkaan ykköstä kovin ``intensiivisuuremaisiksi'' siinä mielessä kuin 
esimerkiksi ominaisenergian $e$. Joka tapauksessa näiden avulla voimme nyt sijoittaa kunkin neljästä 
ekstensiivisuureestamme Reynoldsin siirtoteoreemaan \ref{eq:RTT}:

\begin{align}
 &\frac{dm(t)}{dt}
  = \iiint_{V(t)}\frac{d( \rho)}{dt}dV 
    + \iint_{S(t)} \rho(\vec{v}_s \cdot \vec{n})dA\\
 &\frac{d\vec{p}(t)}{dt}
  = \iiint_{V(t)}\frac{d( \vec{v}\rho)}{dt}dV 
    + \iint_{S(t)} \vec{v}\rho(\vec{v}_s \cdot \vec{n})dA\\
 &\frac{d\vec{L}(t)}{dt}
  = \iiint_{V(t)}\frac{d( \vec{r} \times \vec{v}\rho)}{dt}dV 
    + \iint_{S(t)} \vec{r} \times \vec{v}\rho(\vec{v}_s \cdot \vec{n})dA\\
 &\frac{dE(t)}{dt}
  = \iiint_{V(t)}\frac{d( e\rho)}{dt}dV 
    + \iint_{S(t)} e\rho(\vec{v}_s \cdot \vec{n})dA
\end{align}

Kun sijoitamme nämä säilymislakeihin \ref{eq:V-sailymislait1}-\ref{eq:V-sailymislait4}, saamme 
säilymislaeiksi

\begin{framed}
  \begin{align}
  &\frac{dm(t)}{dt}
    = \iiint_{V(t)}\frac{d( \rho)}{dt}dV 
      + \iint_{S(t)} \rho(\vec{v}_s \cdot \vec{n})dA = 0\\
  &\frac{d\vec{p}(t)}{dt}
    = \iiint_{V(t)}\frac{d( \vec{v}\rho)}{dt}dV 
      + \iint_{S(t)} \vec{v}\rho(\vec{v}_s \cdot \vec{n})dA = \sum \vec{F} \\
  &\frac{d\vec{L}(t)}{dt}
    = \iiint_{V(t)}\frac{d( \vec{r} \times \vec{v}\rho)}{dt}dV 
      + \iint_{S(t)} \vec{r} \times \vec{v}\rho(\vec{v}_s \cdot \vec{n})dA = \sum \vec{M} \\
  &\frac{dE(t)}{dt}
    = \iiint_{V(t)}\frac{d( e\rho)}{dt}dV 
      + \iint_{S(t)} e\rho(\vec{v}_s \cdot \vec{n})dA = \dot{Q} + \dot{W}
  \end{align}
\end{framed}

\section{Nopeita vastauksia}
\section{CFD}

\chapter{Kenttäteoria}
\section{Säilymislait}
\subsection{Navier-Stokesin yhtälöt}
\section{Kitkaton virtaus}
\subsection{Potentiaalivirtaus}

\chapter{Dimensioanalyysi}
\section{Suhteelliset suureet}
\section{Dimensiottomat luvut}

\part*{Liitteet} %___________________________________________________________________________________
\addcontentsline{toc}{part}{Liitteet}

% \chapter{Lähdeluettelo} %----------------------------------------------------------------------------
% 
% \bibliography{pruju}{}
% \bibliographystyle{plain}

\end{document}

