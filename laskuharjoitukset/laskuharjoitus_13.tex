% Copyright 2013 Pauli Jaakkola
% 
% This program is free software: you can redistribute it and/or modify
% it under the terms of the GNU General Public License as published by
% the Free Software Foundation, either version 3 of the License, or
% (at your option) any later version.
% 
% This program is distributed in the hope that it will be useful,
% but WITHOUT ANY WARRANTY; without even the implied warranty of
% MERCHANTABILITY or FITNESS FOR A PARTICULAR PURPOSE.  See the
% GNU General Public License for more details.
% 
% You should have received a copy of the GNU General Public License
% along with this program.  If not, see <http://www.gnu.org/licenses/>.

\documentclass[12pt,a4paper,finnish]{article}

\usepackage[utf8]{inputenc}                 % Tekstiasetuksia, sisältää ääkköset
\usepackage[T1]{fontenc}                    % Tekstiasetuksia, T1-koodatut fontit
\usepackage{ae,aecompl}                     % Paremman näköiset fontit
\usepackage[finnish]{babel}                 % Suomenkielinen tavutus ja otsikot
% \usepackage{a4wide}
\usepackage[intlimits]{amsmath}             % Lisää kaavavoimaa!
\usepackage{amssymb} 
\usepackage{fixltx2e}                       % \textsubscript
\usepackage{enumerate}
\usepackage{framed}
\usepackage{hyperref}
\makeatother
\hypersetup{
  colorlinks=true,
  linkcolor=blue,
}

\renewcommand{\thesection}{}
\renewcommand{\thesubsection}{}
\renewcommand{\thesubsubsection}{}
\makeatletter
\def\@seccntformat#1{\csname #1ignore\expandafter\endcsname\csname the#1\endcsname\quad}
\let\sectionignore\@gobbletwo
\let\latex@numberline\numberline
\def\numberline#1{\if\relax#1\relax\else\latex@numberline{#1}\fi}
\makeatother

\title{LTP++: Lämmönsiirto\\Laskuharjoituksen 13 ratkaisut}
\date{\today}
\author{Pauli Jaakkola}

\begin{document}

\maketitle
\tableofcontents
\newpage

\section{Tehtävä 28}

\begin{math}
 \begin{array}{l l l l}
  L = 0,03m & d = 0,0025m & b = 0,006m & A_L = 1 m^2\\
  T_s = 100\,^{\circ}C & T_{\infty} = 20\,^{\circ}C & k = 380 \frac{W}{mK} & h = 35\frac{W}{m^2K}
 \end{array}
\end{math}

\subsection{a}

\begin{align}
 &\frac{\theta(L)}{\theta_0} = \frac{T(L) - T_{\infty}}{T_s - T_{\infty}} = \frac{\cosh(m(L - L))}{\cosh(mL)}\\
 &T(L) = T_{\infty} + \frac{\cosh(0)}{\cosh(mL)}(T_s - T_{\infty}) = T_{\infty} + \frac{T_s - T_{\infty}}{\cosh(mL)}\\
 &m = \sqrt{\frac{4h}{kd}} \approx 12,14 \frac{1}{m}\\
 &T(L) \approx \underline{95\,^{\circ}C}
\end{align}

\subsection{b}

\begin{align}
 &\dot{Q}_R = mkA_c\tanh(mL)\theta_0 = mk\frac{\pi d^2}{4}\tanh(mL)(T_s - T_{\infty})\\
 &\dot{Q}_R \approx \underline{0,632 W}
\end{align}

\subsection{c}

\begin{align}
 &N = \frac{A_L}{A_b} = \frac{A_L}{b^2} \approx 27780\\
 &\sum \dot{Q}_R = N\dot{Q}_R \approx \underline{17,6 kW}\\
 &\dot{Q}_L = (A_L - NA_c)h(T_s - T_{\infty}) = (A_L - N\frac{\pi d^2}{4})h(T_s - T_{\infty})\\
 &\dot{Q}_L \approx 2,4 kW\\
 &\dot{Q} = \sum \dot{Q}_R + \dot{Q}_L \approx \underline{20 kW}
\end{align}

\section{Tehtävä 25}

\subsection{Kaava}

\begin{align}
 &\frac{\theta_{A,i}}{\theta_{B,i}} = \frac{T_A - T_i}{T_B - T_i} = \frac{\cosh(m_i(L - L))}{\cosh(m_iL)} = \frac{1}{\cosh(m_iL)}\\
 &T_A = T_i + \frac{T_B - T_i}{\cosh(m_iL)}
\end{align}

Ripaparametrille $m$ löytyy valmis kaava, mutta $T_B$ joudutaan ratkaisemaan energian säilymislaista:

\begin{align}
 &\dot{Q}_B = \dot{Q}_i = -\dot{Q}_v\\
 &m_ikA_c\tanh(m_iL)\theta_{i,0} = -m_vkA_c\tanh(m_vL)\theta_{v,0}\\
 &m_i\tanh(m_iL)(T_B - T_i) = -m_v\tanh(m_vL)(T_B - T_v)\\
 &T_B = \frac{m_v\tanh(m_vL)T_v + m_i\tanh(m_iL)T_i}{m_v\tanh(m_vL) + m_i\tanh(m_iL)}
\end{align}

\begin{framed}
  \begin{align}
  &T_B = \frac{m_v\tanh(m_vL)T_v + m_i\tanh(m_iL)T_i}{m_v\tanh(m_vL) + m_i\tanh(m_iL)}\\
  &T_A = T_i + \frac{T_B - T_i}{\cosh(m_iL)}
  \end{align}
\end{framed}

\subsection{Arvot}

\begin{math}
 \begin{array}{l l l}
  L = 0,075m & s = 0,005m & t = 0,0015m\\
  A_c = st = 7,5\cdot10^{-6}m & P = 2(s + t) = 0,013m & \\
  T_i = 20\,^{\circ}C & T_v = 90\,^{\circ}C & \\
  h_i = 10 \frac{W}{m^2K} & h_v = 100 \frac{W}{m^2K} & \\
  k_h = 417 \frac{W}{mK} & k_t = 16 \frac{W}{mK}
 \end{array}
\end{math}

\subsubsection{Hopea}

\begin{align}
 &m_{i,h} = \sqrt{\frac{h_iP}{k_hA_c}} \approx 6,447 \frac{1}{m}\\
 &m_{v,h} = \sqrt{\frac{h_vP}{k_hA_c}} \approx 20,388 \frac{1}{m}\\
 &T_{B,h} \approx 80,533\,^{\circ}C\\
 &T_{A,h} \approx 74,1\,^{\circ}C
\end{align}

\subsubsection{Teräs}

\begin{align}
 &m_{i,t} = \sqrt{\frac{h_iP}{k_tA_c}} \approx 32,914 \frac{1}{m}\\
 &m_{v,t} = \sqrt{\frac{h_vP}{k_tA_c}} \approx 104,083 \frac{1}{m}\\
 &T_{B,t} \approx 73,365\,^{\circ}C\\
 &T_{A,t} \approx 29,0\,^{\circ}C
\end{align}

\section{Tehtävä 34}

\subsection{HUOM!}

\begin{align}
 &\overline{Nu_L} = \frac{1}{L}\int_0^L Nu_x dx\\
 &\frac{\overline{h}L}{k} = \frac{1}{L}\int_0^L \frac{h_x x}{k} dx\\
 &\frac{L}{k}\frac{1}{L}\int_0^L h_x dx = \frac{1}{kL}\int_0^L h_x x dx\\
 &L\int_0^L h_x dx \neq \int_0^L \frac{h_x x}{k} dx
\end{align}

(Sisäpuolisessa virtauksessa tätä ongelmaa ei tule, koska $Nu_D = \frac{hD}{k}$.)

\begin{align}
 &\left\{
 \begin{aligned}
  &\overline{Nu}_L = \frac{\overline{h}L}{k} = \frac{L}{k}\frac{1}{L}\int_0^L h_x dx = \frac{1}{k}\int_0^L h_x dx\\
  &Nu_x = \frac{h_x x}{k} \Leftrightarrow h_x = \frac{kNu_x}{x}
 \end{aligned}\right.\\
 &\overline{Nu}_L = \frac{1}{k}\int_0^L \frac{kNu_x}{x} dx = \int_0^L \frac{Nu_x}{x} dx
\end{align}

\subsection{a}

\begin{framed}
 \begin{equation}
  Nu_x = 0,332Re_x^{1/2}Pr^{1/3} 
 \end{equation}
\end{framed}

\begin{align}
 &Nu_x = 0,332\left(\frac{U_{\infty}x}{\nu}\right)^{1/2}Pr^{1/3}
   = 0,332Pr^{1/3}\left(\frac{U_{\infty}}{\nu}\right)^{1/2}x^{1/2}\\
 &\overline{Nu}_L = \int_0^L \frac{Nu_x}{x} dx = \int_0^L 0,332Pr^{1/3}\left(\frac{U_{\infty}}{\nu}\right)^{1/2}x^{-1/2} dx\\
 &\overline{Nu}_L = 0,332Pr^{1/3}\left(\frac{U_{\infty}}{\nu}\right)^{1/2}\int_0^L x^{-1/2} dx\\
 &\overline{Nu}_L = 0,332Pr^{1/3}\left(\frac{U_{\infty}}{\nu}\right)^{1/2}\bigg/_0^L 2x^{1/2} dx\\
 &\overline{Nu}_L = 0,664Pr^{1/3}\left(\frac{U_{\infty}}{\nu}\right)^{1/2}L^{1/2} = \underline{0,664Re_L^{1/2}Pr^{1/3}}
\end{align}

\subsection{b}

\begin{framed}
 \begin{equation}
  Nu_x = 0,0297Re_x^{0,8}Pr^{0,6}
 \end{equation}
\end{framed}

\begin{align}
 &Nu_x = 0,0297\left(\frac{U_{\infty}x}{\nu}\right)^{0,8}Pr^{0,6}
  = 0,0297Pr^{0,6}\left(\frac{U_{\infty}}{\nu}\right)^{0,8}x^{0,8}\\
 &\overline{Nu}_L = \int_0^L \frac{Nu_x}{x} dx = \int_0^L 0,0297Pr^{0,6}\left(\frac{U_{\infty}}{\nu}\right)^{0,8}x^{-0,2} dx\\
 &\overline{Nu}_L = 0,0297Pr^{0,6}\left(\frac{U_{\infty}}{\nu}\right)^{0,8}\int_0^L x^{-1/5} dx\\
 &\overline{Nu}_L = 0,0297Pr^{0,6}\left(\frac{U_{\infty}}{\nu}\right)^{0,8}\bigg/_0^L \frac{5}{4}x^{4/5} dx\\
 &\overline{Nu}_L = 0,0297Pr^{0,6}\left(\frac{U_{\infty}}{\nu}\right)^{0,8}\frac{5}{4}L^{0,8} = \underline{0,037Re_L^{0,8}Pr^{0,6}}
\end{align}

\end{document}