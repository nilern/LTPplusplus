% Copyright 2013 Pauli Jaakkola
% 
% This program is free software: you can redistribute it and/or modify
% it under the terms of the GNU General Public License as published by
% the Free Software Foundation, either version 3 of the License, or
% (at your option) any later version.
% 
% This program is distributed in the hope that it will be useful,
% but WITHOUT ANY WARRANTY; without even the implied warranty of
% MERCHANTABILITY or FITNESS FOR A PARTICULAR PURPOSE.  See the
% GNU General Public License for more details.
% 
% You should have received a copy of the GNU General Public License
% along with this program.  If not, see <http://www.gnu.org/licenses/>.

\documentclass[12pt,a4paper,finnish]{article}

\usepackage[utf8]{inputenc}                 % Tekstiasetuksia, sisältää ääkköset
\usepackage[T1]{fontenc}                    % Tekstiasetuksia, T1-koodatut fontit
\usepackage{ae,aecompl}                     % Paremman näköiset fontit
\usepackage[finnish]{babel}                 % Suomenkielinen tavutus ja otsikot
% \usepackage{a4wide}
\usepackage[intlimits]{amsmath}             % Lisää kaavavoimaa!
\usepackage{amssymb} 
\usepackage{fixltx2e}                       % \textsubscript
\usepackage{enumerate}
\usepackage{framed}
\usepackage{hyperref}
\makeatother
\hypersetup{
  colorlinks=true,
  linkcolor=blue,
}

\renewcommand{\thesection}{}
\renewcommand{\thesubsection}{}
\renewcommand{\thesubsubsection}{}
\makeatletter
\def\@seccntformat#1{\csname #1ignore\expandafter\endcsname\csname the#1\endcsname\quad}
\let\sectionignore\@gobbletwo
\let\latex@numberline\numberline
\def\numberline#1{\if\relax#1\relax\else\latex@numberline{#1}\fi}
\makeatother

\title{LTP++: Lämmönsiirto\\Laskuharjoituksen 15 ratkaisut}
\date{\today}
\author{Pauli Jaakkola}

\begin{document}

\maketitle
\tableofcontents
\newpage

\section{Tehtävä 49}

\begin{math}
 \begin{array}{l l l}
  T_1 = 5800 K & T_2 = 5700 K & \sigma = 5,6697 \cdot 10^{-8} \frac{W}{m^2K^4}
 \end{array}
\end{math}

\begin{align}
  &\left\{
  \begin{aligned}
    &G_a = \sigma T^4 \quad \bigg|\bigg| \quad [G] = \frac{W}{m^2} = [q] \\
    &\dot{Q}_a = A_aG_a = A_{mk}G_{mk}
  \end{aligned}\right.\\
  &G_{mk} = \frac{A_aG_a}{A_{mk}} = \frac{A_a}{A_{mk}}\sigma T^4 \\
  &\frac{\Delta G_{mk}}{G_{mk1}} = \frac{G_{mk2} - G_{mk1}}{G_{mk1}} 
    = \frac{\frac{A_a}{A_{mk}}\sigma T_2^4 - \frac{A_a}{A_{mk}}\sigma T_1^4}{\frac{A_a}{A_{mk}}\sigma T_1^4} \\
  &\frac{\Delta G_{mk}}{G_{mk1}} = \frac{T_2^4 - T_1^4}{ T_1^4} \approx 0,067 \approx \underline{7\%}
\end{align}

\section{Tehtävä 42}

\begin{math}
 \begin{array}{l l l l}
  d_1 = 0,010 m & r_1 = \frac{d_1}{2} = 0,005 m & d_2 = 0,020 m & r_2 = \frac{d_2}{2} = 0,010 m \\
  \dot{Q}' = -200 \frac{W}{m} & T_{\infty} = 293,15 K & & \\
  k = 1,6 \frac{W}{mK} & h_c = 20 \frac{W}{m^2K} & \varepsilon = 0,9 & \sigma = 5,6697 \cdot 10^{-8} \frac{W}{m^2K^4}
 \end{array}
\end{math}

\subsection{a}

Eristeen korkein lämpötila on tietenkin vastuksen ja eristeen kohtauspinnassa. Kyseessä on 
johtuminen sylinterikuoren läpi:

\begin{align}
 &\dot{Q}' = \frac{\dot{Q}}{L} = \frac{1}{L}\dot{Q} = \frac{1}{L}\frac{T_2 - T_1}{R}
  \quad \bigg|\bigg| \quad R = \frac{\ln \frac{r_2}{r_1}}{2\pi kL} \\
 &\dot{Q}' = \frac{1}{L}\frac{T_2 - T_1}{\frac{\ln \frac{r_2}{r_1}}{2\pi kL}} 
  = \frac{2\pi k}{\ln \frac{r_2}{r_1}}(T_2 - T_1) \\
 &T_1 = T_2 - \frac{\ln \frac{r_2}{r_1}}{2\pi k}\dot{Q}'\\
\end{align}

Mutta tähän tarvitaan $T_2$. Eristeestä poistuva lämpövirta pituutta kohden on jatkuvuustilassa
sama kuin vastuksesta lähtevä:

\begin{align}
 &\dot{Q}' = \frac{\dot{Q}}{L} = \frac{A_v}{L}q = \frac{2\pi r_2L}{L}(q_c + q_r)\\
 &\dot{Q}' = \frac{\dot{Q}}{L} = \frac{A_v}{L}q 
  = \frac{2\pi r_2L}{L}\left(h_c(T_{\infty} - T_2) + \varepsilon\sigma(T_{\infty}^4 - T_2^4)\right)\\
 &\dot{Q}' = 2\pi r_2\left(h_c(T_{\infty} - T_2) + \varepsilon\sigma(T_{\infty}^4 - T_2^4)\right)
\end{align}

Säteilyn johdosta joudutaan neljännen asteen yhtälöön. Se voidaan ratkaista ratkaisukaavalla, iteroimalla 
tai laskimella/laskentaohjelmilla\footnote{MATLAB/Octave/Scilab, Wolfram Alpha/Mathematica, Maxima, Maple, EES...
Luultavasti symboliset laskentaohjelmat käyttävät ratkaisukaavaa ja numeeriset iteroivat.}.

Ratkaisukaava, laskin ja laskentaohjelmat haluavat syödä standardimuotoisia polynomiyhtälöitä:

\begin{equation}
  \varepsilon\sigma T_2^4 + h_cT_2 
    - \left(\varepsilon\sigma T_{\infty}^4 + h_cT_{\infty} - \frac{\dot{Q}'}{2\pi r_2}\right) = 0
\end{equation}

Ratkaistaan kuitenkin nyt iteroimalla, sillä sitä seuraa jatkossa lisää.

(Kiertopiste)iterointiin tarvitaan $T_{2,(i+1)} = f(T_{2,i})$:

\begin{equation}
 T_{2,(i+1)} = \frac{\varepsilon\sigma}{h_c}(T_{\infty}^4 - T_{2,i}^4) + T_{\infty}
  - \frac{\dot{Q}'}{2\pi r_2h_c}
\end{equation}

Prosessin aloittamiseen tarvitaan tietenkin alkuarvaus, olkoon se nyt 

\begin{equation}
  T_{2, 0} = 100 \,^{\circ}C = 373,15K
\end{equation}

Iterointia jatketaan kunnes se 'konvergoi' johonkin arvoon eli tulos ei enää muutu halutuissa rajoissa. 
Olkoon konvergoimisehto nyt

\begin{equation}
 |T_{2,(i+1)} - T_{2,i}| < 0,01
\end{equation}

Sitten toimeen:

\begin{equation}
  \begin{array}{l l l}
    T_{2,0} = 373,15 K &  T_{2,8} = 402,36 K &  T_{2,16} = 403,46 K \\
    T_{2,1} = 421,68 K &  T_{2,9} = 404,28 K &  T_{2,17} = 403,54 K \\
    T_{2,2} = 390,48 K &  T_{2,10} = 402,99 K &  T_{2,18} = 403,49 K \\
    T_{2,3} = 411,83 K &  T_{2,11} = 403,86 K &  T_{2,19} = 403,52 K \\
    T_{2,4} = 397,75 K &  T_{2,12} = 403,28 K &  T_{2,20} = 403,50 K \\
    T_{2,5} = 407,29 K &  T_{2,13} = 403,67 K &  T_{2,21} = 403,51 K \\
    T_{2,6} = 400,94 K &  T_{2,14} = 403,41 K &  T_{2,22} = 403,51 K \\
    T_{2,7} = 405,22 K &  T_{2,15} = 403,58 K &  T_{2,23} = 403,51 K
  \end{array}
\end{equation}

\begin{equation}
 T_2 \approx 403,5 K
\end{equation}

Riippuen funktion $f$ muodosta sekä alkuarvauksen ja tuloksen arvoista iterointi saattaa konvergoinnin 
sijaan hajaantua. Tämän välttämiseen on keinoja, joista yksinkertaisin lienee alkuarvauksen muuttaminen ja 
monimutkaisin jokin paljon kiertopisteiterointia kehittyneempi algoritmi. Näihin ei tässä kuitenkaan ole 
tarvetta mennä.

Enää on sijoitettava tämä eristeen maksimilämpötilan kaavaan:

\begin{equation}
 T_1 = T_2 - \frac{\ln \frac{r_2}{r_1}}{2\pi k}\dot{Q}' \approx 417,4 K \approx \underline{144\,^{\circ}C}
\end{equation}

\subsection{b}

\begin{math}
 g = 9,81 \frac{m}{s^2}
\end{math}\\

Johtumisen kannalta tilanne ei muutu:

\begin{equation}
 T_1 = T_2 - \frac{\ln \frac{r_2}{r_1}}{2\pi k}\dot{Q}'
\end{equation}

Pinnasta lämpö siirtyy kuitenkin vain konvektiolla, jolloin

\begin{align}
 &\dot{Q}' = \frac{\dot{Q}}{L} = \frac{A_v}{L}q = \frac{2\pi r_2L}{L}q_c = 2\pi r_2\bar{h}_c(T_{\infty} - T_2)\\
 &T_2 =  T_{\infty} - \frac{\dot{Q}'}{2\pi r_2\bar{h}_c}
\end{align}

Nusseltin luvun avulla saadaan lämmönsiirtokerroin:

\begin{align}
 &\overline{Nu}_{d_e} = \frac{\bar{h}_cd_e}{k_a}\\
 &\bar{h}_c = \frac{k_a}{d_e}\overline{Nu}_{d_e}
\end{align}

Avataan tähän vielä Nusseltin luku annetusta kaavasta:

\begin{align}
 &\bar{h}_c = \frac{k_a}{d_e}\left(0,36 + 
  \frac{0,518Ra_{d_e}^{1/4}}{\left(1 + \left(\frac{0,559}{Pr}\right)^{9/16}\right)^{4/9}}\right)\\
 &\bar{h}_c = \frac{k_a}{d_e}\left(0,36 + 
  \frac{0,518(Gr_{d_e}Pr)^{1/4}}{\left(1 + \left(\frac{0,559}{Pr}\right)^{9/16}\right)^{4/9}}\right)\\
 &\bar{h}_c = \frac{k_a}{d_e}\left(0,36 + 
  \frac{0,518\left(\frac{g\beta(T_2 - T_{\infty})d_e^3}{\nu^2}Pr\right)^{1/4}}
    {\left(1 + \left(\frac{0,559}{Pr}\right)^{9/16}\right)^{4/9}}\right)
\end{align}

Grashofin luvusta tuli lämmönsiirtokertoimen kaavaan $T_2$, jota olemme ratkaisemassa. Joudutaan siis iteroimaan.
Myös ilman aineominaisuudet $k_a$, $Pr$, $\beta$ ja $\nu$ ovat lämpötilariippuvaisia. 
Iterointi käsittää nyt seuraavan kaavasarjan:

\begin{align}
 &T_{r,i + 1} = \frac{T_{2,i} + T_{\infty}}{2}\\
 &\begin{array}{l l l l}
  k_{a,i+1} = k_a(T_{r, i + 1}) & \beta_{i+1} = \beta(T_{r, i + 1}) \\
    Pr_{i+1} = Pr(T_{r, i + 1}) & \nu_{i+1} = \nu(T_{r, i + 1})
  \end{array}\\
 &\bar{h}_{c, i+1} = \frac{k_{a,i+1}}{d_e}\left(0,36 + 
  \frac{0,518\left(\frac{g\beta_{i+1}(T_{2, i} - T_{\infty})d_e^3}{\nu_{i+1}^2}Pr_{i+1}\right)^{1/4}}
    {\left(1 + \left(\frac{0,559}{Pr_{i+1}}\right)^{9/16}\right)^{4/9}}\right)\\
 &T_{2, i+1} =  T_{\infty} - \frac{\dot{Q}'}{2\pi r_2\bar{h}_{c, i+1}}
\end{align}

Otetaan alkuarvaukseksi $T_{2,0} = 180\,^{\circ}C$ ja toistetaan ylläolevaa iterointiprosessia kunnes
$|T_{2, i+1} - T_{2,i}| < 0,1 K$. Iterointi konvergoi arvoon $T_2 \approx 325\,^{\circ}C$.

\begin{equation}
  T_1 = T_2 - \frac{\ln \frac{r_2}{r_1}}{2\pi k}\dot{Q}' \approx \underline{339\,^{\circ}C}
\end{equation}

\newpage

\section{Tehtävä 43}

\subsection{a}

\begin{math}
 \begin{array}{l l l}
  T_w = 2500 K & \lambda_1 = 0,4\cdot10^{-6}m & \lambda_2 = 0,7\cdot10^{-6}m
 \end{array}
\end{math}

\begin{align}
 &\lambda_1 T_w = 1000 \mu mK\\
 &\lambda_2 T_w = 1750 \mu mK\\
 &F_{\lambda_1-\lambda 2} =  F_{0-\lambda 2} - F_{0-\lambda 1}
  =  F_{0-\lambda}(\lambda_2 T_w) - F_{0-\lambda}(\lambda_1 T_w)\\
 &F_{\lambda_1-\lambda 2} \approx 0,07 - 0 = 0,07 = \underline{7\%}
\end{align}

\subsection{b}

\begin{align}
 &\lambda_{max} T_w = 0,28978\cdot10^{-2}mK\\
 &\lambda_{max} = \frac{0,28978\cdot10^{-2}mK}{T_w} \approx 1,159\cdot10^{-6}m \approx \underline{1,16\mu m}
\end{align}

\subsection{c}

\begin{math}
 \begin{array}{l l l l}
  D = 0,05 m & L^+ = P = \pi D & \dot{Q} = -60 W & \lambda_1 = 2,7\mu m \\
  \tau = 
    \left\{\begin{aligned}
	    &1, \text{kun} \lambda \leq \lambda_1 \\
	    &0, \text{kun} \lambda > \lambda_1
	   \end{aligned}\right. &
  \varepsilon = 0,92 & g = 9,81 \frac{m}{s^2} &
 \end{array}
\end{math}

\begin{align}
 &F_{0 - \lambda_1} = F_{0-\lambda}(\lambda_2r T_w) \approx 0,8 \\
 &\dot{Q}_L = F_{\lambda_1 - \infty}\dot{Q} = (F_{0 - \infty} - F_{0 - \lambda_1})\dot{Q} 
  = (1 - F_{0 - \lambda_1})\dot{Q} \approx -48 W \\
 &q_L = \frac{\dot{Q}_L}{A_L} = \frac{\dot{Q}_L}{4\pi\left(\frac{D}{2}\right)^2} \approx -1746,2 \frac{W}{m^2} \\
 &q_L = q_c + q_r = \bar{h}_c(T_{\infty} - T_L) + \varepsilon\sigma(T_{\infty}^4 - T_L^4)
\end{align}

Saatiin neljännen asteen yhtälö, jossa lisäksi on lämpötilariippuvainen lämmönsiirtokerroin. Kuten 
tehtävässä 42 opimme, näistä molemmista haasteista selviää iteroimalla. Iteroitavat yhtälöt ovat:

\begin{align}
 &T_{r,i + 1} = \frac{T_{L,i} + T_{\infty}}{2}\\
 &\begin{array}{l l l l}
  k_{i+1} = k(T_{r,i + 1}) & \beta_{i+1} = \beta(T_{r,i + 1}) \\
    Pr_{i+1} = Pr(T_{r,i + 1}) & \nu_{i+1} = \nu(T_{r,i + 1})
  \end{array}\\
 &\bar{h}_{c, i+1} = \frac{k_{i+1}}{L^+}\overline{Nu}_{L^+,i+1} \\
 &= \frac{k_{i+1}}{\pi D}\left(5,75 + 
  0,75\left(\frac{\frac{g\beta_{i+1}(T_{L, i} - T_{\infty})(\pi D)^3}{\nu_{i+1}^2}Pr_{i+1}}
    {\left(1 + \left(\frac{0,49}{Pr_{i+1}}\right)^{9/16}\right)^{16/9}}\right)^{0,252}\right)\\
 &T_{L, i+1} =  T_{\infty} + \frac{\varepsilon\sigma}{\bar{h}_{c, i+1}}(T_{\infty}^4 - T_L^4) 
  - \frac{q_L}{\bar{h}_{c, i+1}}
\end{align}

Olkoon nyt alkuarvaus lasin lämpötilaksi $T_{L, 0} = 80\,^{\circ}C = 353,15 K$ ja iteroinnin lopettamisehto 
$|T_{L, i + 1} - T_{L, i}| < 0,1 K$. Tulokseksi suppenee

\begin{equation}
 T_L \approx 112\,^{\circ}C \approx \underline{110\,^{\circ}C}
\end{equation}


\end{document}