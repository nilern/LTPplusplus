% Copyright 2013 Pauli Jaakkola
% 
% This program is free software: you can redistribute it and/or modify
% it under the terms of the GNU General Public License as published by
% the Free Software Foundation, either version 3 of the License, or
% (at your option) any later version.
% 
% This program is distributed in the hope that it will be useful,
% but WITHOUT ANY WARRANTY; without even the implied warranty of
% MERCHANTABILITY or FITNESS FOR A PARTICULAR PURPOSE.  See the
% GNU General Public License for more details.
% 
% You should have received a copy of the GNU General Public License
% along with this program.  If not, see <http://www.gnu.org/licenses/>.

\documentclass[12pt,a4paper,finnish]{article}

\usepackage[utf8]{inputenc}                 % Tekstiasetuksia, sisältää ääkköset
\usepackage[T1]{fontenc}                    % Tekstiasetuksia, T1-koodatut fontit
\usepackage{ae,aecompl}                     % Paremman näköiset fontit
\usepackage[finnish]{babel}                 % Suomenkielinen tavutus ja otsikot
% \usepackage{a4wide}
\usepackage[intlimits]{amsmath}             % Lisää kaavavoimaa!
\usepackage{amssymb} 
\usepackage{fixltx2e}                       % \textsubscript
\usepackage{enumerate}
\usepackage{framed}
\usepackage{hyperref}
\makeatother
\hypersetup{
  colorlinks=true,
  linkcolor=blue,
}

\renewcommand{\thesection}{}
\renewcommand{\thesubsection}{}
\makeatletter
\def\@seccntformat#1{\csname #1ignore\expandafter\endcsname\csname the#1\endcsname\quad}
\let\sectionignore\@gobbletwo
\let\latex@numberline\numberline
\def\numberline#1{\if\relax#1\relax\else\latex@numberline{#1}\fi}
\makeatother

\title{LTP++: Termodynamiikka\\Laskuharjoituksen 1 ratkaisut}
\date{\today}
\author{Pauli Jaakkola}

\begin{document}

\maketitle
\tableofcontents
\newpage

\section{Tehtävä 2}

\begin{math}
\begin{array}{ll}
 p = 1bar = 10^5 Pa & R_u = 8,314 \frac{J}{molK}\\
 T = 20\,^{\circ}C = 293(,15)K & M = 28,965 \frac{g}{mol}\\
 & R = \frac{R_u}{M} \approx 287,036 \frac{J}{kgK}
\end{array}
\end{math}

\begin{align}
 &pV = Nk_BT \quad\bigg|\bigg|\quad Nk_B = nN_Ak_B = nR_u\\
 &\underline{pV = nR_uT} \quad\bigg|\bigg|\quad M = \frac{m}{n} \Leftrightarrow n = \frac{m}{M}\\
 &pV = \frac{m}{M}R_uT \quad\bigg|\bigg|\quad R = \frac{R_u}{M}\\
 &\underline{pV = mRT}\\
 &p\frac{V}{m} = RT \quad\bigg|\bigg|\quad v = \frac{V}{m}\\
 &\underline{pv = RT} \quad\bigg|\bigg|\quad v = \frac{V}{m} = \frac{1}{\frac{m}{V}} = \frac{1}{\rho}\\
 &\frac{p}{\rho} = RT\\
 &\rho = \frac{p}{RT}\\
 &\rho = \frac{p}{RT} \approx \underline{1,189 \frac{kg}{m^3}}
\end{align}

\section{Tehtävä 3}

\subsection{a}
\begin{enumerate}
 \item Ne johtuvat eri asioista:
 \begin{itemize}
  \item Työtä siirtyy systeemin ja ympäristön välillä, koska niiden välillä on nettovoima
    eli ne eivät ole \textit{mekaanisessa tasapainossa}
  \item Lämpöä siirtyy systeemin ja ympäristön välillä, koska niillä on lämpötilaero
    eli ne eivät ole \textit{termisessä tasapainossa}
 \end{itemize}
 \item Työ saadaan muutettua kokonaan lämmöksi, mutta vain osa lämmöstä saadaan muutettua työksi.
  (Termodynamiikan II pääsääntö.)
\end{enumerate}

\subsection{b}

\begin{math}
 \begin{array}{lll}
  m = 1 kg & c_p = 4200 \frac{J}{kgK} & \Delta T = 1\,^{\circ}C = 1 K
 \end{array}
\end{math}

\begin{align}
 &\Delta E = Q + W = Q\\
 &Q = \Delta E = \Delta U  \quad\bigg|\bigg|\quad  \Delta U = m\Delta u = mc_v\Delta T\\
 &c_v = c_p \quad\bigg|\bigg|\quad \text{(Katso \nameref{t:5})}\\
 &Q = mc_p\Delta T \approx 4200 J = \underline{4,2 kJ}
\end{align}

\subsection{c}

\begin{math}
 \begin{array}{lll}
  m = 1 kg & W = Q_{b-kohta} = 4200 J & \vec{v_1} = \vec{0}\\
 \end{array}
\end{math}

\begin{align}
 &\Delta E = Q + W = W\\
 &\Delta E = \Delta E_k = W \quad\bigg|\bigg|\quad 
 \Delta E_k = \Delta \left(\frac{1}{2}m|\vec{v}|^2\right) = \frac{1}{2}m\Delta\left(|\vec{v}|^2\right)\\
 &\frac{1}{2}m\left(|\vec{v_2}|^2 -|\vec{v_1}|^2\right) = W\\
 &\frac{1}{2}m\left(|\vec{v_2}|^2 -|\vec{0}|^2\right)
  = W\\
 &\frac{1}{2}m|\vec{v_2}|^2= W\\
 &|\vec{v_2}| = \sqrt{\frac{2W}{m}} \approx \underline{91,65 \frac{m}{s}}
\end{align}

\subsection{d}

\begin{math}
 \begin{array}{lll}
  m = 1 kg & W = W_{c-kohta} = 4200 J & g = 9,81 \frac{m}{s^2}
 \end{array}
\end{math}

\begin{align}
 &\Delta E = Q + W = W\\
 &\Delta E = \Delta E_p = W \quad\bigg|\bigg|\quad \Delta E_p = mg\Delta h\\
 &mg\Delta h = W\\
 &\Delta h = \frac{W}{mg} \approx \underline{428,3 m}
\end{align}

\section{Tehtävä 4}

\begin{math}
 \begin{array}{lll}
  L = 0,2m & T = 20\,^{\circ}C = 293(,15K )
    & R_u = 8,314\frac{J}{molK}\\
  D = 0,2m&p_\infty = 1 bar = 10^5 pa & M = 28,965\frac{g}{mol}\\
  m = 200kg  & g = 9,81 \frac{m}{s^2}& R = \frac{R_u}{M} \approx 287,036 \frac{J}{kgK}\\
 \end{array}
\end{math}

\begin{align}
 \downarrow: \; &p_\infty A + mg - pA = 0\\
 &pA = p_\infty A + mg\\
 &p = p_\infty + \frac{mg}{A} \quad\bigg|\bigg|\quad A = \pi R^2 = \pi \left(\frac{D}{2}\right)^2 = \frac{\pi}{4}D^2\\
 &p = p_\infty + \frac{mg}{\frac{\pi}{4}D^2} = p_\infty + \frac{4mg}{\pi D^2} 
  \approx 1,624\cdot10^5Pa = \underline{1,624 bar}
\end{align}
\begin{align}
 &pV = m_iRT\\
 &m_i = \frac{pV}{RT} \quad\bigg|\bigg|\quad V = AL = \frac{\pi}{4}D^2L\\
 &m_i = \frac{p\pi D^2L}{4RT} \approx 0,012kg = \underline{12g}
\end{align}

\section{Tehtävä 5} \label{t:5}

Todistetaan, että kokoonpuristumattomalle aineelle

\begin{framed}
 \begin{equation}
  c_p = c_v
 \end{equation}
\end{framed}

\begin{align}
 &c_v = \left(\frac{\partial u}{\partial T}\right)_v\\
 &c_p = \left(\frac{\partial h}{\partial T}\right)_p\\
 &h = u + pv\\
 &\rho = vakio\\
 &v = \frac{1}{\rho} = vakio \Leftrightarrow \frac{\partial v}{\partial T} = 0
\end{align}

\begin{align}
 &c_p = \left(\frac{\partial h}{\partial T}\right)_p 
  = \left(\frac{\partial (u + pv)}{\partial T}\right)_p
   = \left(\frac{\partial u}{\partial T} + \frac{\partial (pv)}{\partial T}\right)_p\\
 &c_p
  = \left(\frac{\partial u}{\partial T} + p\frac{\partial v}{\partial T} 
    + v\frac{\partial p}{\partial T}\right)_p \quad\bigg|\bigg|\quad 
  \frac{\partial v}{\partial T} = 0, \left(\frac{\partial p}{\partial T}\right)_p = 0\\
 &c_p = \left(\frac{\partial u}{\partial T}\right)_p \quad\bigg|\bigg|\quad 
  u = u(T) \Rightarrow \left(\frac{\partial u}{\partial T}\right)_p 
     = \frac{du}{dT} = \left(\frac{\partial u}{\partial T}\right)_v\\
 &c_p = \left(\frac{\partial u}{\partial T}\right)_v = c_v \; \square
\end{align}

\section{Tehtävä 10}

\subsection{a}

Otetaan vettä ja tuhkaa ja mitataan niiden massat sekä lämpötilat. Sitten sekoitetaan ne termospullossa 
(voidaan olettaa adiabaattiseksi) ja mitataan seoksen loppulämpötila. Koska veden 
ominaislämpökapasiteetti tunnetaan, jää tuhkan ominaislämpökapasiteetti ainoaksi tuntemattomaksi.

\subsection{b}

\begin{math}
\begin{array}{lll}
  m_t = 0,4 kg & T_t = 80\;^{\circ}C & T_l = 20\;^{\circ}C\\
  m_v = 0,5 kg & T_v = 10\;^{\circ}C & c_{pv} = 4200 \frac{J}{kgK}\\
\end{array}
\end{math}

\begin{align}
 &Q_t = -Q_v \quad\bigg|\bigg|\quad \Delta U = Q + W = Q\\
 &\Delta U_t = -\Delta U_v \quad\bigg|\bigg|\quad \Delta U = m\Delta u = mc_v\Delta T = mc_p\Delta T\\
 &m_tc_{pt}\Delta T_t = -m_vc_{pv}\Delta T_v \quad\bigg|\bigg|\quad
  \begin{aligned}
    &\Delta T_t = T_l - T_t\\
    &\Delta T_v = T_l - T_v
  \end{aligned}\\
 &m_tc_{pt}(T_l - T_t) = -m_vc_{pv}(T_l - T_v)\\
 &c_{pt} = \frac{m_v(T_l - T_v)}{m_t(T_t - T_l)}c_{pv} \approx \underline{875\frac{J}{kgK}}
\end{align}

\end{document}