% Copyright 2013 Pauli Jaakkola
% 
% This program is free software: you can redistribute it and/or modify
% it under the terms of the GNU General Public License as published by
% the Free Software Foundation, either version 3 of the License, or
% (at your option) any later version.
% 
% This program is distributed in the hope that it will be useful,
% but WITHOUT ANY WARRANTY; without even the implied warranty of
% MERCHANTABILITY or FITNESS FOR A PARTICULAR PURPOSE.  See the
% GNU General Public License for more details.
% 
% You should have received a copy of the GNU General Public License
% along with this program.  If not, see <http://www.gnu.org/licenses/>.

\documentclass[12pt,a4paper,finnish]{article}

\usepackage[utf8]{inputenc}                 % Tekstiasetuksia, sisältää ääkköset
\usepackage[T1]{fontenc}                    % Tekstiasetuksia, T1-koodatut fontit
\usepackage{ae,aecompl}                     % Paremman näköiset fontit
\usepackage[finnish]{babel}                 % Suomenkielinen tavutus ja otsikot
% \usepackage{a4wide}
\usepackage[intlimits]{amsmath}             % Lisää kaavavoimaa!
\usepackage{amssymb} 
\usepackage{fixltx2e}                       % \textsubscript
\usepackage{enumerate}
\usepackage{framed}
\usepackage{hyperref}
\makeatother
\hypersetup{
  colorlinks=true,
  linkcolor=blue,
}

\renewcommand{\thesection}{}
\renewcommand{\thesubsection}{}
\makeatletter
\def\@seccntformat#1{\csname #1ignore\expandafter\endcsname\csname the#1\endcsname\quad}
\let\sectionignore\@gobbletwo
\let\latex@numberline\numberline
\def\numberline#1{\if\relax#1\relax\else\latex@numberline{#1}\fi}
\makeatother

\title{LTP++: Lämmönsiirto\\Laskuharjoituksen 12 ratkaisut}
\date{\today}
\author{Pauli Jaakkola}

\begin{document}

\maketitle
\tableofcontents
\newpage

\section{Tehtävä 17}

(Tämä tehtävä vastaa monilta osin edellisten harjoitusten tehtävää 47. Ratkaisu on 
kuitenkin käsitteellistetty eri tavalla.)

Jäätyvän veden ja ilman välinen \textbf{lämpötilaero} $\Delta T$ aiheuttaa \textbf{lämpövirran} 
$\dot{Q}$. Koska jään lämmönjohtavuus ja jään pinnan lämmönsiirtokerroin ovat äärellisiä, 
lämpövirta $\dot{Q}$ kohtaa kuitenkin \textbf{lämpövastuksen} $R$:

\begin{equation}
 \dot{Q} = \frac{\Delta T}{R}
\end{equation}

Johtumisen lämpövastus on:

\begin{equation}
 R_j = \frac{x}{kA}
\end{equation}

Lämmönsiirtokerroin (konvektio ja mahdollisesti säteily) taas aiheuttaa lämpövastuksen

\begin{equation}
 R_h = \frac{1}{hA}
\end{equation}

Tässä vastukset ovat sarjassa, joten

\begin{equation}
 R = \sum R_i = R_j + R_h = \frac{x}{kA} + \frac{1}{hA}
\end{equation}

Eli

\begin{equation}
 \dot{Q} = \frac{\Delta T}{R} = \frac{T_{\infty} - T_{sf}}{R} = \frac{T_{\infty} 
  - T_{sf}}{\frac{x}{kA} + \frac{1}{hA}} = A\frac{T_{\infty} 
  - T_{sf}}{\frac{x}{k} + \frac{1}{h}}
\end{equation}

Jään jäätyminen taas vapauttaa lämpötehon (johto tehtävässä 47)

\begin{equation}
 \dot{Q} = -\rho Ah_{sf}\frac{dx}{dt}
\end{equation}

Vapautuva lämpöteho siirtyy ilmaan vastusten kautta:

\begin{equation}
 \dot{Q} = -\rho Ah_{sf}\frac{dx}{dt} = A\frac{T_{\infty} - T_{sf}}{\frac{x}{k} + \frac{1}{h}}
\end{equation}

\textbf{Huom!} $h$ on lämmönsiirtokerroin $\left(\frac{W}{m^2K}\right)$, $h_{sf}$ sulamisentalpia 
$\left(\frac{J}{kg}\right)$...

Ratkaistaan differentiaaliyhtälö separoimalla ja integroimalla määrätysti (vastaavasti kuin tehtävässä 47):

\begin{align}
 &\rho h_{sf}\frac{dx}{dt} = \frac{T_{sf} - T_{\infty}}{\frac{x}{k} + \frac{1}{h}}\\
 &\left(\frac{x}{k} + \frac{1}{h}\right) \frac{dx}{dt} = \frac{T_{sf} - T_{\infty}}{\rho h_{sf}}\\
 &\int_0^{x(t)} \left(\frac{x}{k} + \frac{1}{h}\right) dx = \int_0^t \frac{T_{sf} - T_{\infty}}{\rho h_{sf}}dt\\
 &\int_0^{x(t)} \left(\frac{x}{k} + \frac{1}{h}\right) dx = \frac{T_{sf} - T_{\infty}}{\rho h_{sf}} \int_0^t dt\\
 &\bigg/_0^{x(t)} \left(\frac{x^2}{2k} + \frac{x}{h}\right) = \frac{T_{sf} - T_{\infty}}{\rho h_{sf}} \bigg/_0^t t\\
 &\frac{x^2(t)}{2k} + \frac{x(t)}{h} = \frac{T_{sf} - T_{\infty}}{\rho h_{sf}}t
\end{align}

Saatiin toisen asteen yhtälö. Saatetaan se ``standardiin'' polynomimuotoon ja käytetään ratkaisukaavaa:

\begin{align}
 &\frac{1}{2k}x^2(t) + \frac{1}{h}x(t) - \frac{T_{sf} - T_{\infty}}{\rho h_{sf}}t = 0\\
 &x(t) = \frac{ -\frac{1}{h} \pm \sqrt{ \left(\frac{1}{h}\right)^2 
  - 4\frac{1}{2k}\left(-\frac{T_{sf} - T_{\infty}}{\rho h_{sf}}t\right)}}{2\frac{1}{2k}}\\
 &x(t) = \frac{ -\frac{1}{h} \pm \sqrt{ \frac{1}{h^2} 
  + \frac{2}{k\rho h_{sf}}(T_{sf} - T_{\infty})t}}{\frac{1}{k}}\\
 &x(t) = \frac{k}{h}\left( -1 \pm \sqrt{ 1 
  + \frac{2h^2}{k\rho h_{sf}}(T_{sf} - T_{\infty})t}\right)\\
\end{align}

Miinusmerkki neliöjuuren edessä tuottaa negatiivisia paksuuksia, joten lopullinen kaava on

\begin{framed}
 \begin{equation}
  x(t) = \frac{k}{h}\left( -1 + \sqrt{ 1 + \frac{2h^2}{k\rho h_{sf}}(T_{sf} - T_{\infty})t}\right)
 \end{equation}
\end{framed}


Tehtävänannon parametrit:\\

\begin{math}
 \begin{array}{l l l l}
  t = 10 h = 36000 s & T_{sf} = 0\,^{\circ}C & T_{\infty} = -10\,^{\circ}C &\\
  k = 2,2 \frac{W}{m^2K} & h = 10 \frac{W}{m^2K} & h_{sf} = 330 \frac{kJ}{kg} & \rho = 913 \frac{kg}{m^3}
 \end{array}
\end{math}\\

Sijoitus:

\begin{equation}
 x(t) = \frac{k}{h}\left( -1 + \sqrt{ 1 + \frac{2h^2}{k\rho h_{sf}}(T_{sf} - T_{\infty})t}\right) \approx \underline{12 mm}
\end{equation}

\section{Tehtävä 21}

\subsection{a}

\begin{align}
 &\rho c_p \frac{\partial T}{\partial t} = k\nabla^2T + \dot{Q}_v\\
 &\rho c \frac{\partial T}{\partial t} = k\nabla^2T\\
\end{align}

Sylinterikoordinaatistossa:

\begin{equation}
 \nabla^2 = \left(\frac{1}{r}\frac{\partial}{\partial r}\left(r\frac{\partial}{\partial r}\right)
   + \frac{1}{r^2}\frac{\partial^2}{\partial \varphi^2} + \frac{\partial^2}{\partial z^2}\right)
\end{equation}

\begin{align}
 &\rho c \frac{\partial T}{\partial t} = k\left(\frac{1}{r}\frac{\partial}{\partial r}
 \left(r\frac{\partial T}{\partial r}\right) 
  + \frac{1}{r^2}\frac{\partial^2 T}{\partial \varphi^2} + \frac{\partial^2 T}{\partial z^2}\right)\\
 &\frac{\partial T}{\partial t} = \frac{k}{\rho c}\left(\frac{1}{r}\frac{\partial}{\partial r}
 \left(r\frac{\partial T}{\partial r}\right)\right)
\end{align}

$\frac{k}{\rho c}$ on \textit{terminen diffusiviteetti} $\alpha$. Saadaan yhtälöksi

\begin{equation}
 \frac{\partial T(r, t)}{\partial t} = \frac{\alpha}{r}\frac{\partial}{\partial r}
 \left(r\frac{\partial T(r, t)}{\partial r}\right)
\end{equation}

Toisen asteen differentiaaliyhtälö $r$:n suhteen $\Rightarrow$ kaksi reunaehtoa:

\begin{equation}
 \left\{
 \begin{aligned}
  &\frac{\partial T}{\partial r}\bigg|_{r = 0} = 0\\
  &q = -k\frac{\partial T}{\partial r}\bigg|_{r = R} = -h_c(T_{\infty} - T(R, t))
 \end{aligned}\right.
\end{equation}

Eli lämpötilakenttä on symmetrinen x-akselin suhteen ja energian on säilyttävä tangon pinnassa.

Ajan suhteen differentiaaliyhtälö on ensimmäistä astetta eli tarvitaan yksi alkuehto:

\begin{equation}
 T(r, 0) = T_0
\end{equation}

Ratkaisu on siis

\begin{framed}
 \begin{align}
  &\frac{\partial T(r, t)}{\partial t} = \frac{\alpha}{r}\frac{\partial}{\partial r}
    \left(r\frac{\partial T(r, t)}{\partial r}\right)\\
  &\frac{\partial T}{\partial r}\bigg|_{r = 0} = 0\\
  &q = -k\frac{\partial T}{\partial r}\bigg|_{r = R} = -h_c(T_{\infty} - T(R, t))\\
  &T(r, 0) = T_0
 \end{align}
\end{framed}

\subsection{b}

\begin{math}
 \begin{array}{l l l l}
  T(t) = 100\,^{\circ}C & T_{\infty} = 20\,^{\circ}C & R = 0,005 m & \\
  k = 380 \frac{W}{mK} & h_c = 20 \frac{W}{m^2K} & c_p = 380 \frac{J}{kgK} & \rho 8900 \frac{kg}{m^3}
 \end{array}
\end{math}

\begin{align}
 &\Delta H = Q\\
 &c_pm\Delta T = qA_v\Delta t\\
 &c_pmdT = qA_v dt\\
 &\frac{dT}{dt} = \frac{qA_v}{c_pm}
\end{align}

\begin{align}
 &q = -h_c(T_{\infty} - T(t)) = h_c(T(t) - T_{\infty})\\
 &A_v = 2\pi RL\\
 &m = \rho V = \rho A_pL = \rho \pi R^2 L
\end{align}

\begin{align}
 &\frac{dT}{dt} = \frac{h_c(T(t) - T_{\infty})2\pi RL}{c_p\rho \pi R^2 L}\\
 &\frac{dT}{dt} = \frac{2h_c}{c_p\rho R}(T(t) - T_{\infty}) \approx \underline{0,189 \frac{K}{s}}
\end{align}

\section{Tehtävä 23}

\begin{math}
 \begin{array}{l l l}
  L = 0,5m & D = 0,01m & V = \bar{u} = 5 \frac{m}{s}\\
  T_0 = 20\,^{\circ}C & T_s = 100\,^{\circ}C
 \end{array}
\end{math}

\begin{align}
 &Nu_D = \frac{h_cD}{k}\\
 &h_c(x) = \frac{k}{D}Nu_D(x)
\end{align}

\begin{align}
 &Nu_D(x) = Nu_D(x^*)\\
 &x^* = \frac{x}{DRe_DPr}\\
 &Re_D = \frac{VD}{\nu}\\
 &Pr = \frac{\nu}{\alpha}
\end{align}

Aineominaisuudet ovat lämpötilan funktioita. Putkessa lämpötilakenttä ei kuitenkaan ole vakio. 
Keskimääräiset (virtausaineen!) aineominaisuudet saadaan usein riittävän tarkasti ottamalla ne 
referenssilämpötilassa 

\begin{equation}
 T_r = \frac{T_0 + T_s}{2} = 60\,^{\circ}C
\end{equation}

\begin{math}
 \begin{array}{l l l}
  k(T_r) \approx 0,028 & \nu(T_r) \approx 1,89\cdot10^{-5}\frac{m^2}{s} &
  Pr(T_r) \left(= \frac{\nu(T_r)}{\alpha(T_r)}\right) \approx 0,708
 \end{array}
\end{math}

Näillä tiedoilla

\begin{equation}
 Re_D = \frac{VD}{\nu} \approx 2646
\end{equation}

\subsection{a}

\begin{align}
 &x^*(L) = \frac{L}{DRe_DPr} \approx 2,669\cdot 10^{-2}\\
 &Nu_D(L) = Nu_D(x^*(L)) \approx 4\\
 &h_c(L) = \frac{k}{D}Nu_D(L) \approx \underline{11,2\frac{W}{m^2K}}
\end{align}

\subsection{b}

\begin{align}
 &\overline{Nu}_D(L) = \overline{Nu}_D(L(x^*(L)) \approx 5\\
 &\bar{h}_c(L) = \frac{k}{D}\overline{Nu}_D(L) \approx \underline{14\frac{W}{m^2K}}
\end{align}

\subsection{c}

\begin{equation}
 \frac{T_m(L) - T_s}{T_0 - T_s} = \exp\left(-\frac{4KL}{\rho c_pVD}\right)
\end{equation}

Tässä $K$ on lämmönläpäisykerroin ('lämpökonduktanssi', voidaan usein merkitä myös $U$ tai $H$):

\begin{equation}
 q = K\Delta T
\end{equation}

Verrattaessa lämpövastukseen ('lämpöresistanssi'), havaitaan että

\begin{align}
 &\dot{Q} = \frac{\Delta T}{R} = Aq = AK\Delta T\\
 &K = \frac{1}{RA}
\end{align}

Koska tässä ilmavirtauksen ja putken pinnan välissä lämpö siirtyy vain konvektiolla, jolloin 
$q = h_c\Delta T$, niin nyt

\begin{align}
 &K = h_c(L) \approx 11,2\frac{W}{m^2K}\\
 &\frac{T_m(L) - T_s}{T_0 - T_s} = \exp\left(-\frac{4h_c(L)L}{\rho c_pVD}\right)
\end{align}

Tarvitaan muutama aineominaisuus lisää, otetaan nämäkin lämpötilassa $T_r$:\\

\begin{math}
  \begin{array}{l l}
  c_p(T_r) = 1000 \frac{J}{kgK} & \rho(T_r) = 1,1 \frac{kg}{m^3}
  \end{array}
\end{math}\\

Ratkaistaan keskilämpötila putken päässä

\begin{equation}
 T_m(L) = T_s + \exp\left(-\frac{4h_c(L)L}{\rho c_pVD}\right)(T_0 - T_s) \approx \underline{51,917\,^{\circ}C}
\end{equation}

\subsection{d}

Menee transitioalueeseen, joten voidaan käyttää vain muutenkin epätarkkaa Dittus-Bolterin sovitetta 
(virhe voi olla jopa 30-40\%):

\begin{align}
 &\overline{Nu}_D = \overline{Nu}_D(Re_D, Pr) = 0,023Re_D^{0,8}Pr^{0,4} \approx 10,958\\
 &\bar{h}_c = \frac{k}{D}\overline{Nu}_D \approx \underline{30,684 \frac{W}{m^2K}}
\end{align}

No, ainakin se oli helppo laskea ja ainakin suuruusluokka verrattuna laminaariin lienee oikein.

\end{document}