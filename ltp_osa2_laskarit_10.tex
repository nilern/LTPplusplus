% Copyright 2013 Pauli Jaakkola
% 
% This program is free software: you can redistribute it and/or modify
% it under the terms of the GNU General Public License as published by
% the Free Software Foundation, either version 3 of the License, or
% (at your option) any later version.
% 
% This program is distributed in the hope that it will be useful,
% but WITHOUT ANY WARRANTY; without even the implied warranty of
% MERCHANTABILITY or FITNESS FOR A PARTICULAR PURPOSE.  See the
% GNU General Public License for more details.
% 
% You should have received a copy of the GNU General Public License
% along with this program.  If not, see <http://www.gnu.org/licenses/>.

\documentclass[12pt,a4paper,finnish]{article}

\usepackage[utf8]{inputenc}                 % Tekstiasetuksia, sisältää ääkköset
\usepackage[T1]{fontenc}                    % Tekstiasetuksia, T1-koodatut fontit
\usepackage{ae,aecompl}                     % Paremman näköiset fontit
\usepackage[finnish]{babel}                 % Suomenkielinen tavutus ja otsikot
% \usepackage{a4wide}
\usepackage[intlimits]{amsmath}             % Lisää kaavavoimaa!
\usepackage{amssymb} 
\usepackage{fixltx2e}                       % \textsubscript
\usepackage{enumerate}
\usepackage{framed}
\usepackage{hyperref}
\makeatother
\hypersetup{
  colorlinks=true,
  linkcolor=blue,
}

\renewcommand{\thesection}{}
\renewcommand{\thesubsection}{}
\makeatletter
\def\@seccntformat#1{\csname #1ignore\expandafter\endcsname\csname the#1\endcsname\quad}
\let\sectionignore\@gobbletwo
\let\latex@numberline\numberline
\def\numberline#1{\if\relax#1\relax\else\latex@numberline{#1}\fi}
\makeatother

\title{LTP++: Virtausoppi\\Laskuharjoituksen 10 ratkaisut}
\date{\today}
\author{Pauli Jaakkola}

\begin{document}

\maketitle
\tableofcontents
\newpage

\section{Tehtävä 13}

\begin{math}
 \begin{array}{l l}
  U_{\infty} = 10 \frac{m}{s} & L = 1m\\
  \rho = 1,2 \frac{kg}{m^3} & \nu = 1,6\cdot10^{-5}\frac{m^2}{s}
 \end{array}
\end{math}

\subsection{a}

Dimensioton paikka $\eta(x, y)$ ja nopeuskentän ratkaisu sen funktiona $u(\eta)$:

\begin{align}
 &\eta = \eta(x, y) = \sqrt{U_{\infty}}\frac{y}{\sqrt{\nu x}}\\
 &u(x, y) = u(\eta) = U_{\infty}f'(\eta)
\end{align}

Yksi tapa määritellä rajakerroksen paksuus on seuraava: rajakerros ulottuu pinnasta kohtisuoraan 
(y-koordinaatin suuntaan) siihen pisteeseen asti, jossa $u(x, y) = 0,99U_{\infty}$ eli virtauksen 
nopeus $u(x,y)$ on 99\% sen nopeudesta vapaassa virtauksessa $U_{\infty}$ (matemaattis-teoreettisesti 
'vapaa virtaus' on äärettömän kaukana). Eli siis:

\begin{equation}
 u(x, \delta) = 0,99U_{\infty}
\end{equation}

Rajakerroksen reunalla ($y = \delta$):

\begin{align}
 &\eta(x, \delta) = \sqrt{U_{\infty}}\frac{\delta}{\sqrt{\nu x}}\\
 &u(x, \delta) = u(\eta(x, \delta)) = U_{\infty}f'(\eta(x, \delta))\\
 &u(x, \delta) = 0,99U_{\infty} 
\end{align}

\begin{align}
 &u(x, \delta) = U_{\infty}f'(\eta(x, \delta)) = 0,99U_{\infty}\\
 &f'(\eta(x, \delta)) = 0,99\\
 &\eta(x, \delta) = \eta(f') = \eta(0,99) \approx 5
\end{align}

Levyn jättöreunalla $x = L$:

\begin{align}
 &\eta(L, \delta) = \sqrt{U_{\infty}}\frac{\delta}{\sqrt{\nu L}} = \eta(x, \delta)\\
 &\delta(L) = \sqrt{\frac{\nu L}{U_{\infty}}}\eta(x, \delta) \approx 0,0063 m = \underline{6,3 mm}
\end{align}

\subsection{b} \label{13b}

\begin{align}
 &F - F_{\tau} = 0\\
 &F = F_{\tau} = \bar{\tau}_sA = \bar{\tau}_sLW\\
 &F' = \frac{F}{W} = \bar{\tau}_sL
\end{align}

Keskimääräinen Fanningin kitkakerroin $\bar{f}$ (ei siis sama kuin a-kohdan $f$, joka on dimensioton virtafunktio) 
on määritelty (vrt. Darcyn kitkakerroin):

\begin{align}
 &\bar{f} = \frac{\bar{\tau}_s}{\frac{1}{2}\rho U_{\infty}^2}\\
 &\bar{\tau}_s = \frac{1}{2}\bar{f}\rho U_{\infty}^2
\end{align}

Paikalliselle Fanningin kitkakertoimelle on tässä tapauksessa ratkaisu:

\begin{align}
 &f_x = 2f''(0)Re_x^{-\frac{1}{2}}\\
 &f''(0) = 0,332\\
 &Re_x = \frac{U_{\infty}x}{\nu}
\end{align}

Todella typerästi nyt $f''(0)$ merkitsee kuitenkin \textit{dimensiottoman virtafunktion} toista derivaattaa 
kohdassa $0$. Keskimääräinen kitkakerroin $\bar{f}$ saadaan seuraavasti:

\begin{align}
 &\bar{f} = \frac{1}{L}\int_0^Lf_xdx = \frac{1}{L}\int_0^L2f''(0)Re_x^{-\frac{1}{2}}dx 
  = \frac{1}{L}\int_0^L0,664\left(\frac{U_{\infty}x}{\nu}\right)^{-\frac{1}{2}}dx\\
 &\bar{f} = \frac{0,664}{L}\left(\frac{U_{\infty}}{\nu}\right)^{-\frac{1}{2}}\int_0^Lx^{-\frac{1}{2}}dx
  = \frac{0,664}{L}\left(\frac{U_{\infty}}{\nu}\right)^{-\frac{1}{2}}\bigg/_0^L2x^{\frac{1}{2}}\\
 &\bar{f} = \frac{0,664}{L}\left(\frac{U_{\infty}}{\nu}\right)^{-\frac{1}{2}}2L^{\frac{1}{2}}
  = 1,328\left(\frac{U_{\infty}L}{\nu}\right)^{-\frac{1}{2}} = 1,328Re_L^{-\frac{1}{2}}
\end{align}

\begin{align}
 &Re_L = \frac{U_{\infty}L}{\nu} \approx 6,25\cdot10^5\\
 &\bar{f} = 1,328Re_L^{-\frac{1}{2}} \approx 1,68\cdot10^{-3}\\
 &\bar{\tau}_s = \frac{1}{2}\bar{f}\rho U_{\infty}^2 \approx 0,101 Pa\\
 &F' = \bar{\tau}_s L \approx \underline{0,101 \frac{N}{m}}
\end{align}

\section{Tehtävä 16}

\begin{math}
 \begin{array}{l l}
  L = 200 m & W = P = 45 m\\
  &u_{laiva} = U_{\infty} = 21 \text{solmua} = 21\cdot1,852\frac{km}{h} = 21\cdot1,852\frac{1000m}{3600s} \approx 10,803\frac{m}{s}\\
  \rho = 1000 \frac{kg}{m^3} & \nu = 1,7\cdot10^{-6}\frac{m^2}{s}
 \end{array}
\end{math}

\begin{align}
 &\dot{W} = \frac{dW}{dt} = \frac{d(Fs)}{dt} = F\frac{ds}{dt} = Fu_{laiva}\\
 &F = \bar{\tau}_sA_p = \bar{\tau}_sLW\\
 &\bar{\tau}_s = \frac{1}{2}\bar{f}\rho U_{\infty}^2 
  \quad \bigg|\bigg| \quad \bar{f} = \frac{\bar{\tau}_s}{\frac{1}{2}\rho U_{\infty}^2}\\
 &\bar{f} = \bar{f}(Re_L) \quad \bigg|\bigg| \quad \text{kaavakokoelma sivu 8}\\
 &Re_L = \frac{U_{\infty}L}{\nu} \approx 1,271\cdot10^9\\
 &\bar{f} = \bar{f}(Re_L) \approx 1,5\cdot10^{-3}\\
 &\bar{\tau}_s = \frac{1}{2}\bar{f}\rho U_{\infty}^2 \approx 87,534 Pa\\
 &F = \bar{\tau}_sLW \approx 787,806 kN\\
 &\dot{W} = Fu_{laiva} \approx \underline{8,5 MW}
\end{align}

Vaihtoehtoisesti keskimääräisen Fanningin kitkakertoimen saa integroimalla paikallista 
$f_x = 0,0592Re_x^{-\frac{5}{5}}$ vastaavasti kuin tehtävän 13 \nameref{13b}-kohdassa. 
Tulos on samassa suuruusluokassa mutta eroaa merkittävästi.

\section{Tehtävä 10}

\begin{math}
 \begin{array}{l l l}
  L = 2,5 m & h = 0,005m & \bar{u}_{\infty} = 10\frac{m}{s}\\ 
  \rho = 1,2 \frac{kg}{m^2} & \eta = 18\cdot10^{-6}\frac{kg}{ms}
 \end{array}
\end{math}

\subsection{a}

\begin{align}
 \rightarrow: \quad &F_p(x) - F_p(x + dx) - F_{\tau}(y) + F_{\tau}(y + dy) = 0\\
 &p(x)A_x - p(x + dx)A_x - \tau(y)A_y + \tau(y + dy)A_y = 0\\
 &p(x)dydz - p(x + dx)dydz - \tau(y)dxdz + \tau(y + dy)dxdz = 0\\
 &p(x)dy - \left(p(x) + \frac{\partial p}{\partial x}dx\right)dy - 
  \tau(y)dx + \left(\tau(y) + \frac{\partial \tau}{\partial y}dy\right)dx = 0\\
 &-\frac{\partial p}{\partial x}dxdy +\frac{\partial \tau}{\partial y}dydx = 0\\
 &\frac{\partial \tau}{\partial y} = \frac{\partial p}{\partial x}
   \quad \bigg|\bigg| \quad \tau = \eta\frac{\partial u}{\partial y}\\
 &\frac{\partial}{\partial y}\left(\eta\frac{\partial u}{\partial y}\right) = \frac{\partial p}{\partial x}\\
 &\frac{\partial^2 u}{\partial y^2} = \frac{1}{\eta}\frac{\partial p}{\partial x}\\
 &\int\left(\frac{\partial u}{\partial y}\right) = \int\frac{1}{\eta}\frac{\partial p}{\partial x}dy 
  = \frac{1}{\eta}\frac{\partial p}{\partial x}\int dy \\
 &\frac{\partial u}{\partial y} = \frac{1}{\eta}\frac{\partial p}{\partial x}y + C_1\\
 &\int du = \int\left(\frac{1}{\eta}\frac{\partial p}{\partial x}y + C_1\right)dy\\
 &u(y) = \frac{1}{2\eta}\frac{\partial p}{\partial x}y^2 + C_1y + C_2
\end{align}

Kaksi integroimisvakiota eli tarvitaan kaksi reunaehtoa:

\begin{equation}
 \left\{
 \begin{aligned}
  &u(h) = 0\\
  &u(-h) = 0
 \end{aligned}\right.
\end{equation}

\begin{align}
 &\left\{
 \begin{aligned}
  &u(h) = \frac{1}{2\eta}\frac{\partial p}{\partial x}h^2 + C_1h + C_2 = 0\\
  &u(-h) = \frac{1}{2\eta}\frac{\partial p}{\partial x}h^2 - C_1h + C_2 = 0
 \end{aligned}\right.\\
 &\left\{
 \begin{aligned}
  &C_2 = -\frac{1}{2\eta}\frac{\partial p}{\partial x}h^2 - C_1h\\
  &C_1 = \frac{1}{2\eta}\frac{\partial p}{\partial x}h + \frac{C_2}{h}
 \end{aligned}\right.\\
 &C_1 = \frac{1}{2\eta}\frac{\partial p}{\partial x}h -\frac{1}{2\eta}\frac{\partial p}{\partial x}h - C_1\\
 &C_1 = -C_1 \Leftrightarrow C_1 = 0\\
 &C_2 = -\frac{1}{2\eta}\frac{\partial p}{\partial x}h^2
\end{align}

\begin{equation}
 u(y) = \frac{1}{2\eta}\frac{\partial p}{\partial x}y^2 -\frac{1}{2\eta}\frac{\partial p}{\partial x}h^2
\end{equation}

\begin{framed}
 \begin{equation}
  u(y) = \frac{1}{2\eta}\frac{\partial p}{\partial x}(y^2 - h^2)
 \end{equation}
\end{framed}

\subsection{b}

\begin{equation}
 \dot{m} = \rho_{\infty}\dot{V}_{\infty} = \rho_L\dot{V}_L
\end{equation}

Koska nopeus on pieni, oletetaan virtaus kokoonpuristumattomaksi $\rho_{\infty} = \rho_L = \rho$:

\begin{align}
 &\dot{m} = \rho\dot{V}_{\infty} = \rho\dot{V}_L\\
 &\dot{V}_{\infty} = \dot{V}_L\\
 &\bar{u}_{\infty}A_{\infty} = \bar{u}_LA_L
\end{align}

Oletetaan, että lasilevyt ovat hyvin ohuita rakoihin verrattuna $A_{\infty} = A_L + t \approx A_L$:

\begin{align}
 &\bar{u}_{\infty}A_L = \bar{u}_LA_L\\
 &\bar{u}_{\infty} = \bar{u}_L
\end{align}

\begin{align}
 &\bar{u}_{\infty} = \bar{u}_L = \frac{1}{A_L}\int_{-h}^{h}u(y)dA = \frac{1}{2hW}\int_{-h}^{h}u(y)Wdy
  = \frac{1}{2h}\int_{-h}^{h}u(y)dy\\
 &\bar{u}_{\infty} = \frac{1}{2h}\int_{-h}^{h}\frac{1}{2\eta}\frac{\partial p}{\partial x}(y^2 - h^2)dy
  = \frac{1}{4\eta h}\frac{\partial p}{\partial x}\int_{-h}^{h}(y^2 - h^2)dAy\\
 &\bar{u}_{\infty} = \frac{1}{4\eta h}\frac{\partial p}{\partial x}
  \bigg/_{-h}^{h}\left(\frac{1}{3}y^3 - h^2y\right)dy\\
 &\bar{u}_{\infty} = \frac{1}{4\eta h}\frac{\partial p}{\partial x}
  \left( \left(\frac{1}{3}h^3 - h^2h\right) - \left(\frac{1}{3}(-h)^3 - h^2(-h)\right) \right)\\
 &\bar{u}_{\infty} = \frac{1}{4\eta h}\frac{\partial p}{\partial x}
  \left( \left(\frac{1}{3}h^3 - h^3\right) - \left(-\frac{1}{3}h^3 + h^3\right) \right)\\
 &\bar{u}_{\infty} = \frac{1}{4\eta h}\frac{\partial p}{\partial x}
  \left( \frac{1}{3}h^3 - h^3 + \frac{1}{3}h^3 - h^3\right)
  = \frac{1}{4\eta h}\frac{\partial p}{\partial x}\left( \frac{2}{3} - 2 \right)h^3\\
 &\bar{u}_{\infty} = \frac{1}{4\eta}\frac{\partial p}{\partial x} \left( \frac{2}{3} - \frac{6}{3} \right)h^2
  = -\frac{1}{3\eta}\frac{\partial p}{\partial x}h^2
\end{align}

\begin{align}
 &\bar{u}_{\infty} = -\frac{1}{3\eta}\frac{\partial p}{\partial x}h^2 \approx 
  -\frac{1}{3\eta}\frac{\Delta p}{L}h^2 = -\frac{1}{3\eta}\frac{p_2 - p_1}{L}h^2 = \frac{1}{3\eta}\frac{p_1 - p_2}{L}h^2\\
 &p_1 - p_2 = \frac{3\eta L}{h^2}\bar{u}_{\infty} \approx \underline{54,3 Pa}
\end{align}

\subsection{c}

\begin{align}
 &\xi = \frac{\Delta p/L}{\frac{1}{2}\rho \bar{u}_L^2/d_h}\\
 &\Delta p = \xi \frac{1}{2}\rho \bar{u}_L^2\frac{L}{d_h} = 4f \frac{1}{2}\rho \bar{u}_L^2\frac{L}{d_h}
\end{align}

\begin{align}
 &d_h = 4h \approx 0,02 m\\
 &f = \frac{24}{Re_{d_h}} \quad \bigg|\bigg| \quad \text{Rakovirtaukselle, ei prujussa}\\
 &Re_{d_h} = \frac{\bar{u}_Ld_h}{\nu} \quad \bigg|\bigg| \quad \eta = \rho\nu\\
 &Re_{d_h} = \frac{\rho\bar{u}_Ld_h}{\eta} \approx 1,326\cdot 10^4\\
 &f = \frac{24}{Re_{d_h}} \approx 1,810\cdot10^{-3}\\
 &\Delta p = 4f \frac{1}{2}\rho \bar{u}_L^2\frac{L}{d_h} \approx \underline{54,3 Pa}
\end{align}

\subsection{d}

\begin{align}
 &f = f(Re_{d_h}) \approx 0,029 \quad \bigg|\bigg| \quad \text{kaavakokoelman sivu 6}\\
 &\Delta p = 4f \frac{1}{2}\rho \bar{u}_L^2\frac{L}{d_h} \approx \underline{220 Pa}
\end{align}

\end{document}