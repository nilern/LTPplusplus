% Copyright 2013 Pauli Jaakkola
% 
% This program is free software: you can redistribute it and/or modify
% it under the terms of the GNU General Public License as published by
% the Free Software Foundation, either version 3 of the License, or
% (at your option) any later version.
% 
% This program is distributed in the hope that it will be useful,
% but WITHOUT ANY WARRANTY; without even the implied warranty of
% MERCHANTABILITY or FITNESS FOR A PARTICULAR PURPOSE.  See the
% GNU General Public License for more details.
% 
% You should have received a copy of the GNU General Public License
% along with this program.  If not, see <http://www.gnu.org/licenses/>.

\documentclass[12pt,a4paper,finnish]{book}

\usepackage[utf8]{inputenc}                 % Tekstiasetuksia, sisältää ääkköset
\usepackage[T1]{fontenc}                    % Tekstiasetuksia, T1-koodatut fontit
\usepackage{ae,aecompl}                     % Paremman näköiset fontit
\usepackage[finnish]{babel}                 % Suomenkielinen tavutus ja otsikot
% \usepackage{a4wide}
\usepackage{shorttoc}
\usepackage[intlimits]{amsmath}             % Lisää kaavavoimaa!
\usepackage{fixltx2e}                       % \textsubscript
\usepackage{enumerate}
\usepackage{framed}
\usepackage{hyperref}
\makeatother
\hypersetup{
  colorlinks=true,
  linkcolor=blue,
  urlcolor=blue
}



\title{LTP++\\Lämmönsiirron perusteet}
\date{\today}
\author{Pauli Jaakkola}

\begin{document}

\maketitle
% GPL, LaTeX-maininnat tähän
\newpage
\pagenumbering{Roman}
\shorttoc{Sisältö lyhyesti}{0}
\setcounter{tocdepth}{3}
\tableofcontents
\newpage
\pagenumbering{arabic}
\setcounter{secnumdepth}{3}

\chapter*{Johdanto} %------------------------------------------------------------------------------------------
\addcontentsline{toc}{chapter}{Johdanto}
\renewcommand{\thesection}{\arabic{section}}

\section{Lämpötieteet ja lämpötekniikka} %..............................................

On helppo ajatella suoraviivaisesti, että \textbf{tieteet} lähtevät liikkeelle kiinnostuksesta johonkin 
luonnossa tapahtuvaan \textbf{ilmiöön}. 
Sitten tehdään \textbf{perustutkimusta} -- laaditaan \textbf{teorioita} ja \textbf{testataan niitä kokeellisesti}. 
Lopulta kun teoria on riittävän \textbf{yleispätevä}, joku käyttää sitä ja luovuuttaan teknisen tai muun \textbf{käytännön 
sovelluksen} luomiseen. Ollaan edetty \textbf{soveltavaan tutkimukseen}.

\textbf{Lämpötieteet}, kuten tässä kirjassa käsiteltävät

\begin{itemize}
 \item Termodynamiikka
 \item Virtausoppi
 \item Lämmönsiirto
\end{itemize}

ovat kuitenkin suurelta osin ns. \textbf{teknisiä tieteitä} eli insinöörien työkaluja. Ne ovat 
syntyneet pikemminkin tutkittaessa, miten \textbf{lämpöteknisiä} sovelluksia, kuten 

\begin{itemize}
 \item Lämpövoimakoneita (voimalaitokset)
 \item Lämpöpumppuja (jäähdytys ja lämmitys)
 \item Virtauskoneita (pumppuja, puhaltimia, kompressoreja ja turbiineja)
 \item Lämmönsiirtimiä (monien prosessien osana)
\end{itemize}

voitaisiin parantaa. Vaikka lämpötieteet ovat sittemmin monin osin kehittyneet maailmaa syleilevän yleispäteviksi, 
niiden keskeisin tai ainakin hyvödyllisin sovellusalue on edelleen juuri lämpötekniikka.

\section{Suureet, luonnonlait ja kaavat; ymmärrys ja tulokset} %...............................................

Mielikuvamme luonnontieteistä ja teknisistä tieteistä on usein sellainen, että ne koostuvat pääosin \textbf{kaavoista}. 
Itse asiassa kaavat ovat kuitenkin vain korkealle abstraktiotasolle jalostettuja yhteenvetoja siitä ymmärryksestä, 
joka on saavutettu \textbf{teoreettisten mallien} ja \textbf{kokeellisen tutkimuksen} vuorovaikutuksessa.
Kaavoja suositaan luonnontieteissä myös sen takia, että ne ovat \textbf{kvantitatiivisia} tieteitä, jotka 
pyrkivät tarkkuuteen ja yksiselitteisyyteen eli \textbf{eksaktiuteen}\footnote{Luonnontieteilijät pitävät 
joskus -- tai useinkin -- itseään jotenkin ihmistieteilijöitä parempina tällä perusteella. Tämä näkyy 
teekkarien ja humanistien välisessä vastakkainasettelussa mutta myös siinä, että englannin kielen tiedettä 
tarkoittava sana \textit{science} voi yksinään tarkoittaa nimenomaan luonnontiedettä, jopa erotuksena 
ihmistieteistä. Todellisuudessa luonnontieteiden kvantitatiivisuus ja eksaktius johtuu kuitenkin siitä, että 
tarkasteltavat ilmiöt ovat oikeastaan hyvin yksinkertaisia verrattuna vaikkapa ihmisen käyttäytymiseen.}

Useimmiten kaavat kertovat joidenkin \textbf{suureiden} välisen yhteyden matemaattisessa muodossa. Yhteys sinänsä 
saattaa olla syvällinen ajatus, jopa \textbf{luonnonlaki} -- monet kaavat kuvaavat esimerkiksi energian säilymistä:

\begin{align}
 \frac{\dot{Q} + \dot{W}}{\dot{m}} &= \Delta\left(h + \frac{1}{2}\vec{V}^2 + gz\right)\\
 \rho c_v\frac{dT}{dt} &= k\nabla^2T + \dot{Q}
\end{align}

Kuitenkin ehkä suurempi osa luonnontieteen oivalluksista on käsitteellistetty itse suureisiin. Monet kaavatkin 
ovat itse asiassa vain suureiden määritelmiä:

\begin{align}
 H &= U + pV\\
 G &= H - T\Delta S 
\end{align}

Niinpä jos tämän kirjan ``punainen lanka'' ovat luonnonlait, niin ehkä suureet ovat toinen yhtälailla tärkeä
``vihreä lanka''\footnote{Myös erään puolueen lehti. Tämän alaviitteen tarkoitus on kuitenkin huomauttaa, 
ettei tässä ole kyse tuotesijoittelusta tai poliittisesta propagandasta.}. Kaavat ovat toki tärkeitä, sillä 
niillä saadaan \textbf{tuloksia}, mutta vasta niiden taustalla vaikuttavien suureiden ja luonnonlakien 
ymmärtäminen mahdollistaa \textbf{luovuuden}. Tai edes oikeiden kaavojen käytön oikeassa tilanteessa ja siten 
tulosten \textbf{oikeellisuuden}.

\section{Miksi nämä tieteet?}

Lämpötieteellisten ilmiöiden ja -teknisten laitteiden analyysi on käytännössä useimmiten monitieteellistä. 
Mietitäänpä vaikkapa \textbf{lämmönsiirrintä}, joka siirtää \textbf{lämpötehoa $\dot{Q}$} vesivirtauksesta $a$ 
vesivirtaukseen $b$. Seuraavassa esiintyviä kaavoja ei tietenkään tarvitse tässä vaiheessa vielä ymmärtää.

Ensinnäkin meitä tietenkin kiinnostaa lämmönsiirron suunta. \textbf{Termodynamiikan toisen pääsäännön} mukaan 
lämpö siirtyy spontaanisti\footnote{``itsestään, luonnostaan''} korkeammasta lämpötilasta matalampaan. Tämä on 
kokeellinen havainto, mutta klassinen termodynamiikka selittää sen niin, että \textbf{entropian} täytyy kasvaa. 
Tilastollinen termodynamiikka selittää, miksi näin on. Matemaattisesti:

\begin{align}
 &T_a > T_b\\
 &\Rightarrow \dot{Q}_a < 0\\
 &\Rightarrow \dot{Q}_b > 0
\end{align}

Kun lämmönsiirron suunta on nyt selvillä, meitä tietenkin kiinnostaa kummankin vesivirran lämpötilan muutos. 
\textbf{Termodynamiikan ensimmäisen pääsäännön} mukaan energia säilyy eli virtausten entalpiat muuttuvat 
lämmön verran. Oletetaan että kaikki lämpö siirtyy $a$:sta $b$:hen (eikä esim. lämmönsiirtimen rakenteisiin):

\begin{align}
 &\dot{Q}_b = -\dot{Q}_a\\
 &\Delta \dot{H_a} = \dot{Q_a}\\
 &\Delta \dot{H_b} = \dot{Q_b}
\end{align}

Käyttämällä entalpiavirran ja ominaisentalpian ($\dot{H} = \dot{m}h$) sekä ominaisentalpian ja lämpötilan 
($\Delta h = c_p \Delta T$) välisiä yhteyksiä saadaan virtausten lämpötilojen muutoksen lämmönsiirtimessä:

\begin{align}
 &\Delta T_a = \frac{\dot{Q_a}}{c_p\dot{m}_a}\\
 &\Delta T_b = \frac{\dot{Q_b}}{c_p\dot{m}_b}
\end{align}

Tässä vaiheessa ongelmaksi tulee tietenkin sen määrittäminen, miten suuri siirtyvä lämpöteho on. Tähän 
tarvitaan \textbf{lämmönsiirtoa}. Lämpö siirtyy virtauksissa \textbf{konvektiolla} ja \textbf{johtumalla} 
ja putkien läpi johtumalla. Lämmön johtumisen teoria on melko yksinkertainen ja tarkka. Konvektiivisesta 
lämmönsiirrosta saadaan kohtuullinen arvio dimensiottomien lukujen avulla ilmaistuilla kokeellisilla 
korrelaatioilla.

Konvektiivisen lämmönsiirron tarkempi määrittäminen vaatisi \textbf{virtausopin} tuntemusta. Sitä 
tarvitaan myös sen määrittämiseen, miten suuren \textbf{mekaanisen tehon} $\dot{W}$ virtauksen 
pumppaaminen lämmönsiirtimen läpi vaatisi.

Nämä kolme ovat siis keskeisimmät lämpötekniikassa tarvittavat tieteet. Tietenkään poikkitieteellisyys 
ei välttämättä lopu vielä tähän. Esimerkiksi lämmönsiirtimen rakenteiden mitoittamiseen 
lämpötilaeroista johtuvat mekaaniset rasitukset kestäviksi tarvittaisiin lujuuslaskentaa. Usein 
voimalaitoksissa lämpö saadaan joko polttoprosessista tai ydinreaktiosta, joiden analysoimiseen 
tarvitaan fysikaalista kemiaa tai ydinfysiikkaa jne.

\section{Merkinnöistä} %................................................................

Lämpötieteiden kirjallinen perinne on vanha ja julkaisujen määrä valtava. Tämän seikan valossa on täysin 
ymmärrettävää, että \textbf{käytetyt merkinnätkin vaihtelevat melkoisesti}:

\begin{itemize}
 \item Esimerkiksi $q$:lla voidaan merkitä ominaislämpöä ($J/kg$), lämpövirran tiheyttä ($W/m^2$) tai jopa 
  tilavuusvirtaa ($m^3/s$). 
 \item Samaten $u$, $v$ ja $h$ voivat merkitä sisäenergiaa, ominaistilavuutta ja entalpiaa -- tai 
  sitten nopeusvektorin x- ja y-komponentteja sekä lämmönsiirtokerrointa.
\end{itemize}
  
Tämän tilanteen syntyä on edesauttanut myös se, että jo lämpötieteiden sisällä -- saati sitten fysiikassa 
yleensä -- on käytössä niin monta suuretta, että latinalaiset tai kreikkalaisetkaan aakkoset eivät tahdo riittää.

Kun teoria on hyvin hallussa, suureet menevät harvoin sekaisin sekalaisista merkinnöistä huolimatta. 
Laskentatilanteesta, kaavojen muodosta ja yksiköistä näkee, mistä suureista on kyse. Mutta tätä kirjaa 
lukevat ainakin toivottavasti ne, joilla teoria ei ole vielä juuri ollenkaan hallussa.

Niinpä \textbf{olen pyrkinyt yksiselitteiseen merkintätapaan}, jossa eri suureita ei merkitä samalla merkinnällä. 
Mikäli eri kirjaimen käyttäminen olisi täysin yleisen käytännön vastaista käytän vektorimerkkejä 
tai aikaderivaattoja  suureiden erottelemiseksi:
 
\begin{itemize}
 \item ominaistilavuus $v$, vauhti eli nopeusvektorin pituus $|\vec{v}|$
 \item tilavuus $V$, tilavuusvirta $\dot{V}$
\end{itemize}

\renewcommand{\thepart}{\arabic{part}}
\setcounter{part}{-1}
\part{Suureita} % _____________________________________________________________________________________________
\setcounter{chapter}{0}
\renewcommand{\thesection}{\arabic{chapter}.\arabic{section}}

Ennen kuin alamme varsinaisesti käsitellä termodynamiikkaa tai muitakaan lämpötieteitä on syytä palauttaa mieleen 
muutama perussuure yksiköineen ja määritelmineen. Luultavasti suureet ovat ennestään tuttuja etkä halua käyttää 
niihin juurikaan aikaa mutta perusteelliseen ymmärrykseen on hyvä pyrkiä -- pidemmällä tähtäimellä sitä kautta 
pääsee vähemmällä. Ja mistäpä muualta perusteellinen ymmärrys lähtisi kuin perusteista, perusasoista.

% \section{Puhtaista luvuista vektoreihin}
% 
% \begin{itemize}
%  \item \textbf{Puhtailla luvuilla} ei ole yksikköä. Meillä voi olla esimerkiksi 2 pakastinta, 3 turbiinia ja 1 lämmönvaihdin.
%  \item \textbf{Skalaarisuureilla} on yksikkö eli suuruus: pituudella metri, massalla kilogramma jne.
%  \item \textbf{Vektorisuureilla} on suuruus ja suunta, esimerkiksi voiman yksikkö on newton mutta on välttämätöntä 
%  tietää myös mihin suuntaan voima vetää tai työntää.
% \end{itemize}

% kunkin suureen kohdalla, onko se mikä näistä!

\chapter{Perussuureita} %--------------------------------------------------------------------------------------------

\section{Avaruus ja aika} % ................................................................

Lämpötieteissä ulottuvuuksia käsitellään klassisen fysiikan tapaan eli avaruusulottuvuuksia on kolme ja ne ovat 
toisistaan sekä ajan yhdestä ulottuvuudesta erillisiä. Syitä tähän on pohjimmiltaan kaksi:

\begin{enumerate}
 \item Lämpötieteet syntyivät ennen suhteellisuusteoriaa ja muuta modernia fysiikkaa.
 \item Käytännön sovelluksissa on harvinaista joutua käsittelemään tilanteita joissa tarvittaisiin 
 suhteellisuusteorian aika-avaruutta\footnote{Miksi tämä on englanniksi ``spacetime'' ja suomeksi ``aika-avaruus''?}
 tai ylimääräisiä avaruusulottuvuuksia.
\end{enumerate}

\subsection{Avaruus}

\subsubsection{Pituus L}

\textbf{Pituuden} (usein \textbf{L}) yksikkönä käytetään SI-perusyksikkö \textbf{metriä}:

\begin{equation}
 [L] = m
\end{equation}

(``Metri on sellaisen matkan pituus, jonka valo kulkee tyhjiössä aikavälissä 1/299 792 458 sekuntia 
(17. CGPM, 1983).'')

\subsubsection{Pinta-ala A}

\textbf{Pinta-alan} (usein \textbf{A}) yksikkönä käytetään \textbf{neliömetriä}:

\begin{equation}
 [A] = m^2
\end{equation}

(Suorakaiteen pinta-alaa voi kuvata kertomalla sen sivujen pituudet toisillaan. Koska minkä muotoisen tasokuvion 
tahansa voi ajatella muodostuvan esimerkiksi mielivaltaisen pienistä neliöistä, on neliömetri pätevä mittaamaan 
mielivaltaisen muotoisia pinta-aloja).

\subsubsection{Tilavuus V}

\textbf{Tilavuuden} (usein \textbf{V}) yksikkönä käytetään \textbf{kuutiometriä}:

\begin{equation}
 [V] = m^3
\end{equation}

(Suorakulmaisen särmiön tilavuutta voi kuvata kertomalla sen sivujen pituudet toisillaan. Koska minkä muotoisen avaruuskappaleen 
tahansa voi ajatella muodostuvan esimerkiksi mielivaltaisen pienistä kuutioista, on kuutiometri pätevä mittaamaan 
mielivaltaisen muotoisia tilavuuksia).

\subsubsection{Yksiulotteinen sijainti s}

Ensimmäisenä on syytä mainita, että jos tilannetta voidaan kuvata yksiulotteisena\footnote{ajattele vaikkapa 
raiteillaan pysyvää junaa}, käytetään joskus ainoana avaruuskoordinaattina \textbf{sijaintia s}.

\subsubsection{Kolmiulotteinen sijainti \={r}}

Avaruuden ulottuvuuksia on siis kolme ja ne muodostavan kolmiulotteisen avaruuden. 

Mikä tahansa piste tässä avaruudessa voidaan määrittää \textbf{kolmen koordinaatin} avulla. 
Ensin koordinaatit täytyy kalibroida määrittämällä niille nollakohdat sekä yksiköt. ``Kartesiolainen'' eli 
suorakulmainen $(x, y, z)$-koordinaatisto on yleisin, mutta lämpötieteissä eivät ole erityisen harvinaisia tilanteet 
joissa esimerkiksi sylinterikoordinaatisto $(z, r, \theta)$ tai pallokoordinaatisto $(r, \theta, \phi)$ on kätevämpi.

\textbf{Origo} sijaitsee koordinaattien nollakohtien leikkauspisteessä $(0, 0, 0)$. Minkä tahansa pisteen sijainti 
voidaan ilmoittaa \textbf{paikkavektorilla $\vec{r}$} origosta kyseiseen pisteeseen. Kartesiolaisessa koordinaatistossa

\begin{equation}
 \vec{r} = \begin{bmatrix}
            x\\ y\\ z
           \end{bmatrix}
\end{equation}

\subsection{Aika t}

\textbf{Ajan t} yksikkönä käytetään SI-perusyksikkö \textbf{sekuntia}:

\begin{equation}
 [t] = s
\end{equation}

(``Sekunti on 9 192 631 770 kertaa sellaisen säteilyn jaksonaika, joka vastaa cesium 133 -atomin siirtymää 
perustilan ylihienorakenteen kahden energiatason välillä (13. CGPM, 1967).'')

% Kenttäsuureet

\section{Aineen määrä} % ..............................................................

\subsection{Ainemäärä n}

\textbf{Ainemäärä n} kertoo kuinka monta kappaletta jotain hiukkasta (yleensä molekyylia) on. Se on siis itse 
asiassa puhdas luku, mutta koska yleensä käsitellään niin suuria molekyylimääriä, on sille määritetty 
SI-perusyksikkö \textbf{mooli}:

\begin{equation}
 [n] = mol
\end{equation}

``Mooli on sellaisen systeemin ainemäärä, joka sisältää yhtä monta keskenään samanlaista perusosasta kuin 0,012 
kilogrammassa hiili 12:ta on atomeja. Perusosaset voivat olla atomeja, molekyylejä, ioneja, elektroneja,
muita hiukkasia tai sellaisten hiukkasten määriteltyjä ryhmiä. (14. CGPM, 1971)''

0,012 kilogrammassa hiili-12:ta eli yhdessä moolissa olevien hiukkasten lukumäärä on \textbf{Avogadron luku $N_A$}:

\begin{equation}
 [N_A] \approx (6,02214129 \pm 0,00000027)\cdot10^{23}
\end{equation}

Ainemäärä on hyödyllinen yleensä kemiassa (koska reaktioissa väliä on molekyylien määrällä) ja kaasuja käsiteltäessä 
(koska mm. tilavuudet ja paineet riippuvat molekyylien määristä).

\subsection{Massa m} \label{def:massa}

Mekaniikassa meitä kiinnostaa kuitenkin yleensä pikemminkin se, miten ``painava'' tai ``hidas'' käsiteltävä systeemi 
on. Tätä mitataan \textbf{massalla m}. Klassisessa fysiikassa esiintyy itse asiassa kahdenlaista massaa:

\begin{itemize}
 \item \textbf{Hidas massa} on Newtonin II laissa esiintyvä massa. Se mittaa siis sitä miten suuri voima tarvitaan 
 kappaleen kiihdyttämiseen.
 \item \textbf{Painava massa} taas on Newtonin gravitaatiolaissa esiintyvä massa. Se mittaa siis sitä miten suuren 
 voiman gravitaatiokenttä aiheuttaa kappaleeseen\footnote{Vrt. varaus sähkömagneettisissa kentissä.}.
\end{itemize}

Hidas ja painava massa ovat kuitenkin saman suuruiset, mikä ei ollut klassisen fysiikan teorioiden perusteella  
mitenkään itsestään selvää. Kuitenkin jo Galileo Galilei huomasi kokeellisesti, että kaikkien kappaleiden 
kappaleen \textbf{putoamiskiihtyvyys g} on sama. Näin voi olla vain, mikäli hidas ja painava massa ovat yhtä suuret.

% Todistus

Suppea suhteellisuusteoria pätee vain vakionopeudella liikkuville koordinaatistoille (arkisemmin ``tarkkailijoille''). 
Se sai alkunsa sähkömagneettisten aaltojen teoriassa tehdystä havainnosta että valon nopeus tyhjiössä on 
koordinaatiston nopeudesta riippumaton vakio.

Yleinen suhteellisuusteoria pätee myös kiihtyvässä liikkeessä oleville koordinaatistoille. Sen perustava oivallus 
oli nimenomaan se, että hidas ja painava massa tuskin ovat sattumalta täsmälleen yhtä suuret. Putoamiskiihtyvyys 
on kiihtyvyys, joka aiheutuu aika-avaruuden kaareutumisesta massan ympärillä.

Massan SI-perusyksikkö on \textbf{kilogramma kg}:

\begin{equation}
 [m] = kg
\end{equation}

``Kilogramma on yhtä suuri kuin kansainvälisen kilogramman prototyypin massa (1. ja 3. CGPM, 1889 ja 1901).''

Kilogramma on ainoa SI-perusyksikkö, joka vielä perustuu tällaiseen prototyyppiin. Tämä on ongelmallista ensinnäkin 
siksi, että prototyyppi ei ole toistettavissa ja toisekseen siksi että - kauhistus sentään - prototyypin massa 
ei mittausten mukaan ole vakio. Alun perin kilogramma piti määritellä ``1 litra vettä on massaltaan kilogramman 
4 $^{\circ}$C:n lämpötilassa''. Vesipohjaiseen määritelmään siirtymistä on myöhemminkin ehdotettu, joskin niin että 
määritelmä vastaisi nykyistä kilogramman määritelmää paremmalla tarkkuudella.

\subsection{Moolimassa M}

Systeemin massa ja ainemäärä riippuvat toisistaan \textbf{moolimassan M} kautta:

\begin{equation}
 M = \frac{m}{n}
\end{equation}

Moolimassan SI-yksiköksi tulee

\begin{equation}
 [M] = \left[\frac{m}{n}\right] = \frac{[m]}{[n]} = \frac{kg}{mol}
\end{equation}

Tämä on kuitenkin niin suuri yksikkö että helpommin käsiteltäviä lukuja saadaan käyttämällä yksikkönä joko 
g/mol (yleisin) tai kg/kmol.

\chapter{Yksinkertaisia johdannaissuureita} % --------------------------------------------------------------------

\section{Aikaderivaattasuureet} % ................................................................

% Suureen muutos / ajan muutos - muutos tietyssä ajassa - aika mieliv. lyhyeksi - aikaderivointi

% Ajan ja tilan kytkentä - vaikka itsenäisiä, muuttuvat yhtä aikaa

\subsection{Nopeus \={v}}

Yksiulotteisen sijainnin muutoksen suhde ajan muutokseen on \textbf{vauhti $|\vec{v}|$}:

\begin{equation}
 v = \frac{\Delta s}{\Delta t}
\end{equation}

Jos vauhti halutaan hetkellisesti eli mielivaltaisen lyhyenä ajanhetkenä tämä lähestyy aikaderivaattaa:

\begin{equation}
 v = \frac{ds}{dt}
\end{equation}

Kun tämä siirretään kolmiulotteiseen avaruuteen paikkavektorin $\vec{r}$ derivaataksi saadaan \textbf{nopeus $\vec{v}$} 
joka on siis myös vektorisuure:

\begin{equation}
 \vec{v} = \frac{d\vec{r}}{dt}
\end{equation}

Aikaderivaattaa on usein tapana merkitä pisteellä:

\begin{equation}
 \vec{v} = \frac{d\vec{r}}{dt} = \dot{\vec{r}}
\end{equation}

Nopeuden yksiköksi tulee sama kuin vauhdinkin eli

\begin{equation}
 [v] = \left[\frac{ds}{dt}\right] = \left[\frac{s}{t}\right] = \frac{m}{s}
\end{equation}

\subsection{Kiihtyvyys \={a}}

Vauhdin muutos ajan suhteen on \textbf{kiihtyvyys $|\vec{a}|$}:

\begin{equation}
 a = \frac{\Delta v}{\Delta t}
\end{equation}

Jos vauhti halutaan hetkellisesti eli mielivaltaisen lyhyenä ajanhetkenä tämä lähestyy aikaderivaattaa:

\begin{equation}
 a = \frac{dv}{dt}
\end{equation}

Kolmiulotteisessa avaruudessa nopeusvektorin $\vec{v}$ derivaataksi saadaan \textbf{kiihtyvyysvektori $\vec{a}$}:

\begin{equation}
 \vec{a} = \frac{d\vec{v}}{dt} = \dot{\vec{v}} = \ddot{\vec{r}}
\end{equation}

Kiihtyvyyden yksikkö on

\begin{equation}
 [a] = \left[\frac{dv}{dt}\right] = \left[\frac{d^2s}{dt^2}\right] = \left[\frac{s}{t^2}\right] = \frac{m}{s^2}
\end{equation}

\subsection{Tilavuusvirta \.{V}}

Kun halutaan tietää, kuinka suuri tilavuus kulkee jonkin pinnan läpi aikayksikössä voidaan 
se määrittää vastaavalla menettelyllä kuin nopeus. Keskimääräinen \textbf{tilavuusvirta} $\dot{V}$ on

\begin{equation}
 \dot{V} = \frac{\Delta V}{\Delta T}
\end{equation}

ja hetkellinen

\begin{equation}
 \dot{V} = \frac{dV}{dt}
\end{equation}

Tilavuusvirran SI-yksikkö on

\begin{equation}
 [\dot{V}] = \left[\frac{dV}{dt}\right] = \left[\frac{V}{t}\right] = \frac{m^3}{s}
\end{equation}

Tämänkaltaisista aikaderivoiduista suureista, jotka eivät ole nopeutta, kiihtyvyyttä eivätkä mekaanista tai
lämpötehoa on tapana käyttää \textbf{virta}-nimitystä.

\subsection{Moolivirta \.{n}}

Pinnan läpi aikayksikössä menevä ainemäärä on \textbf{moolivirta} $\dot{n}$:

\begin{align}
 &\dot{n} = \frac{\Delta n}{\Delta t}\\ 
 &\dot{n} = \frac{dn}{dt}\\
 &[\dot{n}] = \left[\frac{dn}{dt}\right] = \left[\frac{n}{t}\right] = \frac{mol}{s}
\end{align}

\subsection{Massavirta \.{m}}

Pinnan läpi aikayksikössä menevä massa on \textbf{massavirta} $\dot{m}$:

\begin{align}
 &\dot{m} = \frac{\Delta m}{\Delta t}\\ 
 &\dot{m} = \frac{dm}{dt}\\
 &[\dot{m}] = \left[\frac{dm}{dt}\right] = \left[\frac{m}{t}\right] = \frac{kg}{s}
\end{align}

\section{Ekstensiivi- ja intensiivisuureet} %...............................................

\textbf{Ekstensiivisuureet} ovat suureita, joiden arvo riippuu systeemin koosta\footnote{``extent''} eli 
massasta tai ainemäärästä. Tyypillinen ekstensiivisuure on tilavuus $V$. \textbf{Intensiivisuureiden} arvot 
taas eivät riipu systeemin koosta. Tyypillisiä intensiivisuureita ovat paine $p$ ja lämpötila $T$.

\subsection{Ominaissuureet}

Intensiivisuureet ovat siinä mielessä toivottavampia, että niiden käyttö ei vaadi systeemin koon 
selvittämistä tai kiinnittämistä. Niistä saadaan jopa skalaarikenttiä (esim. $T(\vec{r}, t)$). 

Onneksi 
ekstensiivisuureet voidaan muuttaa intensiivisuureiksi jakamalla ne systeemin massalla. Näin 
syntyviä intensiivisuureita kutsutaan \textbf{ominaissuureiksi} ja merkitään vastaavaa ekstensiivisuureen 
suurta vastaavalla pienellä kirjaimella. Esimerkiksi ominaistilavuus on

\begin{align}
 &v = \frac{V}{m}\\
 &[v] = \left[\frac{V}{m}\right] = \frac{m^3}{kg}
\end{align}

ja ominaissisäenergia

\begin{align}
 &u = \frac{U}{m}\\
 &[u] = \left[\frac{U}{m}\right] = \frac{J}{kg}
\end{align}

\subsection{Molaariset ominaissuureet}

Toinen vaihtoehto ekstensiivisuureiden muuntamiseksi intensiivisiksi on niiden jakaminen systeemin 
ainemäärällä sen massan sijaan. Näin saadaan \textbf{molaarisia ominaissuureita}, joita merkitään 
alaindeksillä $m$. Esimerkiksi moolitilavuus on

\begin{align}
 &V_m = \frac{V}{n}\\
 &[V_m] = \left[\frac{V}{n}\right] = \frac{m^3}{mol}
\end{align}

ja molaarinen sisäenergia

\begin{align}
 &U_m = \frac{U}{n}\\
 &[U_m] = \left[\frac{U}{n}\right] = \frac{J}{mol}
\end{align}

\chapter{Monimutkaisempia johdannaissuureita} % ------------------------------------------------------------------

\section{Voima \={F}} % ...........................................................

Olemme tottuneet ajattelemaan voimaa jonkinlaisena perussuureena, mutta itse asiassa se on vain hyvin kätevä 
johdannaissuure, joka on määritelty Newtonin II lain \footnote{Newtonin I laki on II lain erikoistapaus.} perusteella:

\begin{equation}
 \vec{F} = \frac{d\vec{p}}{dt} = \frac{d(m\vec{v})}{dt}
\end{equation}

Niinpä sen yksiköksi tulee:

\begin{equation}
 [F] = [|\vec{F}|] = \left[\frac{d(m|\vec{v}|)}{dt}\right] = \left[\frac{mv}{t}\right] 
  = \left[\frac{mL}{t^2}\right] = \frac{kgm}{s^2}
\end{equation}

Tämä on edelleen nimetty\footnote{ilmeisistä syistä} Newtoniksi:

\begin{equation}
 [F] = \frac{kgm}{s^2} = N
\end{equation}

Voimia ei liene todellisuudessa olemassakaan. Ne ovat vain yksi ihmiskunnan historian hyödyllisimmistä abstraktioista. 
Tämä voiman eksakti muoto kuvaa vain sitä mitä arkikielen voima-sanakin: ``voimaa'' tarvitaan sitä enemmän mitä enemmän 
ja mitä nopeammin materiaa joudutaan kiihdyttämään (tai hidastamaan, $ a < 0$).

\section{Paine p} %....................................................................

\textbf{Paineella} $p$ tarkoitetaan yksinkertaisimmillaan voimaa jaettuna pinta-alalle, jolle se kohdistuu.
Paineen SI-yksikkö on Pascal $Pa$.

\begin{align}
 &p = \frac{|\vec{F}|}{A}\\
 &[p] = \left[\frac{|\vec{F}|}{A}\right] = \frac{N}{m^2} = Pa
\end{align}

Virtausaineissa tilanne ei kuitenkaan ole näin yksinkertainen. Paine voidaan nimittäin määrittää 
mille tahansa virtausaineen reunoilla tai sen sisällä olevalle todelliselle tai kuvitteelliselle pinnalle. 
Itse asiassa virtausopissa paine voidaan (infinitesimaalisten kontrollitilavuuksien $dV$ avulla) määrittää
virtausaineen jokaiselle pisteelle eli $p = p(\vec{r}, t)$).

\section{Työ W} \label{def:W}%......................................................................

Mitä työ on? Mekaniikassa \textbf{työn W} yleinen määritelmä on

\begin{equation}
\label{eq:W}
 W = \int_S \vec{F} \cdot \vec{ds}
\end{equation}

Mitä tämä sitten tarkoittaa? 

Arkisestikin voimme todeta, että jonkin kappaleen siirtämisen ``työläys'' on suoraan verrannollinen

\begin{enumerate}
 \item Voimaan $F$, joka tarvitaan kappaleen liikuttamiseksi
 \item Matkaan $s$, joka kappaletta siirretään
\end{enumerate}

Kun voima on vakio ja reitti koko ajan voiman suuntainen, nämä verrannollisuudet voidaan yhdistää tuloksi ja 
(valitsemalla määritelmässä verrannollisuuskertoimeksi 1) määritellä työ

\begin{equation}
 W = |\vec{F}|s
\end{equation}

Yleisessä tapauksessa ei päästä näin helpolla, sillä kappaleeseen vaikuttavan voiman suuruus ja suunta voivat 
riippua esimerkiksi ajasta ja kappaleen paikasta eikä reittikään ole välttämättä lähelläkään suoraa. 

Onneksi 
mikä tahansa infinitesimaalisen lyhyt reitin pätkä $\vec{ds}$ voidaan katsoa hyvällä tarkkuudella suoraksi ja 
voiman reitin suuntainen komponentti saadaan pistetulolla eli

\begin{equation}
 dW = \vec{F} \cdot \vec{ds}
\end{equation}

Kun nämä infinitesimaalisen lyhyet reitin pätkät sitten summataan eli integroidaan saadaan työn yleinen määritelmä 
\ref{eq:W}.

Työn määritelmä voitaisiin avata sanallisesti vaikka seuraavasti:

\begin{quotation}
``Kun kappale, johon voima $\vec{F}$ vaikuttaa, kulkee 
reitin $S$ tekee voima kaavasta \ref{eq:W} laskettavissa olevan määrän työtä.''
\end{quotation}

 Huomioi, että tämä ei vaadi, että 
juuri voima $\vec{F}$ aiheuttaisi kappaleen liikkeen.

\section{Teho \.W}

Jossain ajassa tehty työ tai hetkellisenä työn aikaderivaatta on \textbf{teho} $\dot{W}$. 
Tehon SI-yksikkö on Watti $W$.

\begin{align}
 &\dot{W} = \frac{\Delta W}{\Delta t}\\ 
 &\dot{W} = \frac{dW}{dt}\\
 &[\dot{W}] = \left[\frac{dW}{dt}\right] = \left[\frac{W}{t}\right] = \frac{J}{s} = W
\end{align}

Tehosta käytetään useimmiten merkintää $P$. Itse käytän kuitenkin merkintää $\dot{W}$ sekaannusten 
välttämiseksi paineen kanssa, jota joskus myös merkitään pienen sijaan isolla p:llä\footnote{
Erityisesti paine ei ole ominaisteho $p \neq \frac{P}{m}$.}. Toisaalta näin 
korostan myös tehon yhteyttä työhön (lämmön sijasta).

\part{Lämmönsiirto} %____________________________________________________________________

Termodynamiikassa energia voi siirtyä kahdella tavalla: työnä tai \textbf{lämpönä}. 

Energiaa siirtyy työnä systeemin rajojen yli kun systeemin ja ympäristön välillä vaikuttaa 
nettovoima. Työn osaamme laskea erinäisissä tapauksissa mekaniikasta mahdollisesti 
hyödyntäen myös esim. virtausoppia mekaniikan erityisalana tai sähkömagnetiikkaa 
toisena fysiikan alana.

Lämpönä energiaa siirtyy systeemin rajojen yli kun systeemin ja ympäristön välillä on 
\textbf{lämpötilaero}. Termodynamiikan ensimmäisellä pääsäännöllä kykenemme selvittämään 
siirtyvän lämpömäärän \textbf{vaikutuksen} systeemiin ja toisella sen, \textbf{miksi} lämpöä 
siirtyy ja että se siirtyy spontaanisti vain \textbf{korkeammasta lämpötilasta matalampaan}. 
Siirtyvän lämmön tärkeimmän ominaisuuden eli \textbf{määrän} laskemiseen tietyn geometrian ja 
olosuhteiden vallitessa tarvitaan kuitenkin oma tieteensä eli \textbf{lämmönsiirto}.

\chapter{Lämmönsiirtotavat}

Mitä suurempi on systeemin lämpötila, sitä suuremmalla todennäköisyydellä sen molekyylit 
ovat merkittävästi perustilaa korkeammalla energialla. Kun korkeammassa lämpötilassa 
olevat molekyylit vuorovaikuttavat matalammilla energiatasoilla olevien kanssa, ne 
vaihtavat energiaa ja pitkällä aikavälillä lämpötilaero pyrkii termodynamiikan 
toisen pääsäännön mukaisesti tasoittumaan.

Lämmön siirtyminen johtuu aina lämpötilaerosta, mutta lämpö voi siirtyä kolmella eri 
tavalla: \textbf{johtumalla} (``conduction''), \textbf{konvektiolla eli kulkeutumalla} 
(``convection'') ja \textbf{säteilemällä} (``radiation'').

\section{Johtuminen}
Kun molekyylit vuorovaikuttavat suoraan toisiinsa esimerkiksi törmäilemällä virtausaineessa 
tai hilavärähtelyiden kautta kiinteässä aineessa, siirtyy lämpö johtumalla.

\section{Konvektio}
Kun sisäenergiaa kulkeutuu virtausaineen molekyylien mukana niiden virratessa siirtyy 
lämpö konvektiolla. Tähänhän ei välttämättä vaadita lämpötilaeroa ja virtausaineen 
siirtäminen systeemin rajojen yli voidaan käsitellä suoraan termodynaamisesti:

\begin{equation}
 \dot{Q} = \dot{m}h
\end{equation}

Lämmönsiirrossa \textbf{konvektiivisella lämmönsiirrolla} tarkoitetaankin yleensä 
tilannetta, jossa virtausaine virtaa kiinteän aineen yli jolloin se kuljettaa 
mukanaan sisäenergiaa. Systeemin rajojen yli (oli systeemi sitten kiinteä aine tai 
virtausaine) lämpö kuitenkin siirtyy itse asiassa johtumalla.

\section{Säteily}
Kaikki kiihtyvässä liikkessä olevat varaukset lähettävät sähkömagneettista säteilyä. 
Jos varauksilla on keskimäärin suuri liike-energia, on niillä tilastollisen 
termodynamiikan mukaisesti myös korkea lämpötila. Itse asiassa kaikki, minkä 
lämpötila on yli $0 K$ lähettää lämpösäteilyksi kutsuttua sähkömagneettista säteilyä. 

Arjessa kohtaamme yleensä lähinnä kappaleita, joiden lähettämä lämpösäteily on 
aallonpituudeltaan enimmäkseen infrapunasäteilyä. Riittävän kuumat kappaleet kuten 
taottava rauta tai aurinko kuitenkin voivat lämpösäteillä myös näkyvää valoa tai 
auringon tapauksessa ultraviolettisäteilyä jne. Itse asiassahan me vain näemme ne 
aallonpituudet, joiden intensiteetti auringonvalon aallonpituusjakaumassa on korkein.

Vaikka säteilylämmönsiirrossa molekyylit vuorovaikuttavatkin kaukaa sähkömagneettisen 
säteilyn kautta, pätee termodynamiikan toinen pääsääntö myös sille. Eli vaihtaessaan 
energiaa lämpösäteilyn muodossa molekyylit lähestyvät toistensa lämpötiloja.

\chapter{Johtuminen}

\chapter{Konvektiivinen lämmönsiirto}
\section{Johtuminen ja konvektio}

\chapter{Säteilylämmönsiirto}

\part*{Liitteet} %___________________________________________________________________________________
\addcontentsline{toc}{part}{Liitteet}

\chapter{Lähdeluettelo} %----------------------------------------------------------------------------

\bibliography{pruju}{}
\bibliographystyle{plain}

\end{document}

