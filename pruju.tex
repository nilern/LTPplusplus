% Copyright 2013 Pauli Jaakkola
% 
% This program is free software: you can redistribute it and/or modify
% it under the terms of the GNU General Public License as published by
% the Free Software Foundation, either version 3 of the License, or
% (at your option) any later version.
% 
% This program is distributed in the hope that it will be useful,
% but WITHOUT ANY WARRANTY; without even the implied warranty of
% MERCHANTABILITY or FITNESS FOR A PARTICULAR PURPOSE.  See the
% GNU General Public License for more details.
% 
% You should have received a copy of the GNU General Public License
% along with this program.  If not, see <http://www.gnu.org/licenses/>.

\documentclass[12pt,a4paper,finnish]{book}

\usepackage[utf8]{inputenc}               % Tekstiasetuksia, sisältää ääkköset
\usepackage[T1]{fontenc}                    % Tekstiasetuksia, T1-koodatut fontit
\usepackage[finnish]{babel}                 % Suomenkielinen tavutus ja otsikot
\usepackage{a4wide}
\usepackage[intlimits]{amsmath}                        % Lisää kaavavoimaa!
\usepackage{enumerate}
% \usepackage{framed}
\usepackage{hyperref}
\makeatother
\hypersetup{
  colorlinks=true,
  linkcolor=blue,
}

\title{LTP++\\Lämpötekniikan perusteita}
\date{\today}
\author{Pauli Jaakkola}

\begin{document}

\maketitle
% Creative Commons, LaTeX-maininnat tähän
\newpage
\pagenumbering{Roman}
\setcounter{tocdepth}{3}
\tableofcontents
\newpage
\pagenumbering{arabic}
\setcounter{secnumdepth}{3}

% \chapter{Johdanto}
% 
\chapter{Systeemi} % -----------------------------------------------
% % Avoin, suljettu, adiabaattinen

Systeemin käsite on lämpötieteissä hyvin keskeinen ja hyödyllinen.

\section{Kontrollitilavuus} \label{section:kontrollitilavuus}

\textbf{Kontrollitilavuus} on mielivaltaisen avaruudessa sijaitsevan \textit{kontrollipinnan} sisältämä tilavuus. 
Pinnan sijainti ja muoto voi myös riippua ajasta. 

Kontrollitilavuus voidaan siis valita täysin vapaasti, 
mutta usein oikein valittu kontrollitilavuus helpottaa haluttujen tulosten saamista tai jopa ylipäätään mahdollistaa sen.

\section{Systeemin käsite}

\textbf{Systeemin} käsite on erityisesti termodynamiikassa keskeinen. Systeemiksi voidaan valita mikä tahansa 
\nameref{section:kontrollitilavuus}.

Vaikka automaatiolle keskeisessä systeemiteoriassa systeemin käsite on suomennettu \textit{järjestelmäksi},
lämpötieteissä puhutaan anglisistisesti \textit{systeemeistä}. Mahdollisesti sellaiset asiat kuin 
sisäenergia ja entropia merkityksineen luovat sellaista kuvaa, että lämpötieteelliset systeemit eivät 
yleensä ole pohjimmiltaan erityisen ``järjestelmällisiä'' tai ``järjestyksessä''.

\section{Ympäristön käsite}

\textbf{Ympäristö} käsittää termodynamiikassa kaiken systeemin ulkopuolella olevan. Yhdessä systeemi ja ympäristö 
muodostavat siis \textbf{maailmankaikkeuden}.

\section{Avoin ja suljettu systeemi}

\textbf{Avoimen systeemin} kontrollipinta on avoin eli sen läpi voi kulkea ainetta. Tyypillisiä esimerkkejä 
ovat \textit{lämmönsiirrin} ja \textit{turbiini}.

\textbf{Suljetun systeemin} kontrollipinta on suljettu eli sen läpi ei voi kulkea ainetta. Tyypillisiä 
esimerkkejä ovat \textit{suljettu kaasusäiliö} ja \textit{metallikuutio}.

\section{Eristämätön ja eristetty systeemi}

\textbf{Eristämättömän systeemin} kontrollipinnan läpi voi vapaasti siirtyä lämpöä. 
Tällainen on esimerkiksi \textit{ilmatilavuus keskellä muuta ilmaa}.

\textbf{Täydellisesti eristetyn eli adiabaattisen systeemin} kontrollipinnan läpi ei siirry lämpöä.

Täydellisesti eristettyjä systeemejä ei tietenkään ole todellisuudessa olemassa, vaan \textit{eristeillä} 
on jokin äärellinen lämpövastus, joka vähentää systeemistä poistuvan lämmön määrää. Hyvin eristettyjä 
systeemejä ovat esimerkiksi \textit{termospullo} ja \textit{passiivienergiatalo}.

\chapter{Suureita} % -----------------------------------------------

Ennen kuin alamme varsinaisesti käsitellä termodynamiikkaa tai muitakaan lämpötieteitä on syytä palauttaa mieleen 
muutama suure yksiköineen ja määritelmineen. Luultavasti suureet ovat ennestään tuttuja etkä halua käyttää 
niihin juurikaan aikaa mutta yllätyksiäkin lienee luvassa ja syvälliseen ymmärrykseen on hyvä pyrkiä - pidemmällä 
tähtäimellä sitä kautta pääsee vähemmällä.

\section{Puhtaista luvuista vektoreihin}

\begin{itemize}
 \item \textbf{Puhtailla luvuilla} ei ole yksikköä. Meillä voi olla esimerkiksi 2 pakastinta, 3 turbiinia ja 1 lämmönvaihdin.
 \item \textbf{Skalaarisuureilla} on yksikkö eli suuruus: pituudella metri, massalla kilogramma jne.
 \item \textbf{Vektorisuureilla} on suuruus ja suunta, esimerkiksi voiman yksikkö on newton mutta on välttämätöntä 
 tietää myös mihin suuntaan voima vetää tai työntää.
\end{itemize}

% kunkin suureen kohdalla, onko se mikä näistä!

\section{Avaruus ja aika} % ...................................................

Lämpötieteissä ulottuvuuksia käsitellään klassisen fysiikan tapaan eli avaruusulottuvuuksia on kolme ja ne ovat 
toisistaan sekä ajan yhdestä ulottuvuudesta erillisiä. Syitä tähän on pohjimmiltaan kaksi:

\begin{enumerate}
 \item Lämpötieteet syntyivät ennen suhteellisuusteoriaa ja muuta modernia fysiikkaa.
 \item Käytännön sovelluksissa on harvinaista joutua käsittelemään tilanteita joissa tarvittaisiin 
 suhteellisuusteorian aika-avaruutta\footnote{Miksi tämä on englanniksi ``spacetime'' ja suomeksi ``aika-avaruus''?}
 tai ylimääräisiä avaruusulottuvuuksia.
\end{enumerate}

\subsection{Avaruus}

\subsubsection{Yksiulotteinen sijainti s}

Ensimmäisenä on syytä mainita, että jos tilannetta voidaan kuvata yksiulotteisena\footnote{ajattele vaikkapa 
raiteillaan pysyvää junaa}, käytetään joskus ainoana avaruuskoordinaattina \textbf{sijaintia s}.

\subsubsection{Kolmiulotteinen sijainti r}

Avaruuden ulottuvuuksia on siis kolme ja ne muodostavan kolmiulotteisen avaruuden. 

Mikä tahansa piste tässä avaruudessa voidaan määrittää \textbf{kolmen koordinaatin} avulla. 
Ensin koordinaatit täytyy kalibroida määrittämällä niille nollakohdat sekä yksiköt. ``Kartesiolainen'' eli 
suorakulmainen $(x, y, z)$-koordinaatisto on yleisin, mutta lämpötieteissä eivät ole erityisen harvinaisia tilanteet 
joissa esimerkiksi sylinterikoordinaatisto $(z, r, \theta)$ tai pallokoordinaatisto $(r, \theta, \phi)$ on kätevämpi.

\textbf{Origo} sijaitsee koordinaattien nollakohtien leikkauspisteessä $(0, 0, 0)$. Minkä tahansa pisteen sijainti 
voidaan ilmoittaa \textbf{paikkavektorilla $\vec{r}$} origosta kyseiseen pisteeseen. Kartesiolaisessa koordinaatistossa

\begin{equation}
 \vec{r} = \begin{bmatrix}
            x\\ y\\ z
           \end{bmatrix}
\end{equation}

\subsubsection{Pituus L}

\textbf{Pituuden} (usein \textbf{L}) yksikkönä käytetään SI-perusyksikkö \textbf{metriä}:

\begin{equation}
 [L] = m
\end{equation}

(``Metri on sellaisen matkan pituus, jonka valo kulkee tyhjiössä aikavälissä 1/299 792 458 sekuntia 
(17. CGPM, 1983).'')

\subsubsection{Pinta-ala A}

\textbf{Pinta-alan} (usein \textbf{A}) yksikkönä käytetään \textbf{neliömetriä}:

\begin{equation}
 [A] = m^2
\end{equation}

(Suorakaiteen pinta-alaa voi kuvata kertomalla sen sivujen pituudet toisillaan. Koska minkä muotoisen tasokuvion 
tahansa voi ajatella muodostuvan esimerkiksi mielivaltaisen pienistä neliöistä, on neliömetri pätevä mittaamaan 
mielivaltaisen muotoisia pinta-aloja).

\subsubsection{Tilavuus V}

\textbf{Tilavuuden} (usein \textbf{V}) yksikkönä käytetään \textbf{kuutiometriä}:

\begin{equation}
 [V] = m^3
\end{equation}

(Suorakulmaisen särmiön voi kuvata kertomalla sen sivujen pituudet toisillaan. Koska minkä muotoisen avaruuskappaleen 
tahansa voi ajatella muodostuvan esimerkiksi mielivaltaisen pienistä kuutioista, on kuutiometri pätevä mittaamaan 
mielivaltaisen muotoisia tilavuuksia).

\subsection{Aika t}

\textbf{Ajan t} yksikkönä käytetään SI-perusyksikkö \textbf{sekuntia}:

\begin{equation}
 [t] = s
\end{equation}

(``Sekunti on 9 192 631 770 kertaa sellaisen säteilyn jaksonaika, joka vastaa cesium 133 -atomin siirtymää 
perustilan ylihienorakenteen kahden energiatason välillä (13. CGPM, 1967).'')

% massa ja moolimäärä

\section{Aikaderivaattasuureet} % ...................................................

% Suureen muutos / ajan muutos - muutos tietyssä ajassa - aika mieliv. lyhyeksi - aikaderivointi

% Ajan ja tilan kytkentä - vaikka itsenäisiä, muuttuvat yhtä aikaa

\subsection{Nopeus v}

Yksiulotteisen sijainnin muutoksen suhde ajan muutokseen on \textbf{vauhti $|\vec{v}|$}:

\begin{equation}
 v = \frac{\Delta s}{\Delta t}
\end{equation}

Jos vauhti halutaan mielivaltaisen lyhyenä ajanhetkenä tämä lähestyy aikaderivaattaa:

\begin{equation}
 v = \frac{ds}{dt}
\end{equation}

Kun tämä siiretään kolmiulotteiseen avaruuteen paikkavektorin $\vec{r}$ derivaataksi saadaan \textbf{nopeus $\vec{v}$} 
joka on siis myös vektorisuure:

\begin{equation}
 \vec{v} = \frac{d\vec{r}}{dt}
\end{equation}

Nopeuden yksiköksi tulee sama kuin vauhdinkin eli

\begin{equation}
 [v] = \left[\frac{ds}{dt}\right] = \left[\frac{s}{t}\right] = \frac{m}{s}
\end{equation}

% aikaderivaatan pistemerkintä & ``virta''
% kiihtyvyys, massa- mooli, tilavuusvirrat

\section{Voima F} % ...................................................

Olemme tottuneet ajattelemaan voimaa jonkinlaisena perussuureena, mutta itse asiassa se on vain hyvin kätevä 
johdannaissuure, joka on määritelty Newtonin II lain \footnote{Newtonin I laki on II lain erikoistapaus.} perusteella:

\begin{equation}
 \vec{F} = \frac{d\vec{p}}{dt} = \frac{d(m\vec{v})}{dt}
\end{equation}

Niinpä sen yksiköksi tulee:

\begin{equation}
 [F] = [|\vec{F}|] = \left[\frac{d(m|\vec{v}|)}{dt}\right] = \left[\frac{mv}{t}\right] 
  = \left[\frac{mL}{t^2}\right] = \frac{kgm}{s^2}
\end{equation}

Tämä on edelleen nimetty\footnote{ilmeisistä syistä} Newtoniksi:

\begin{equation}
 [F] = \frac{kgm}{s^2} = N
\end{equation}

Voimia ei liene todellisuudessa olemassakaan. Ne ovat vain yksi ihmiskunnan historian hyödyllisimmistä abstraktioista. 
Tämä voiman eksakti muoto kuvaa vain sitä mitä arkikielen voima-sanakin: ``voimaa'' tarvitaan sitä enemmän mitä enemmän 
ja mitä nopeammin materiaa joudutaan kiihdyttämään (tai hidastamaan, $ a < 0$).


\part{Termodynamiikka} % ___________________________________________________________________

% O:s pääsääntö ja lämpötila
% Tilasuureet
% Ideaalikaasun tilanyhtälö

\chapter{Energian tiede} % -----------------------------------------------

Termodynamiikan ydin on energia. Termodynamiikan pääsäännötkin käsittelevät energiaa; sen määrää, 
muotoa, laatua, jakautumista ja niin edelleen.

\section{Klassinen ja tilastollinen termodynamiikka}

\textbf{Klassinen termodynamiikka} käsittelee makroskooppisia systeemejä, ilmiöitä ja suureita. Se kehitettiin 
olennaisilta osiltaan valmiiksi ennen kuin molekyylien olemassaolo oli yleisesti hyväksyttyä tai todennettua. 

Esimerkiksi lämpötekniikassa keskeiset lämpötekniset laitteet kuten lämpövoimakoneet, 
lämpöpumput, pumput, puhaltimet, turbiinit sekä lämmönvaihtimet ovat makroskooppisia systeemejä, joiden 
tutkimuksessa klassisella termodynamiikalla saadaan suhteellisen helposti kiinnostavia ja hyödyllisiä tuloksia.

Valitettavasti klassisen termodynamiikan lainalaisuuksia on hankala ymmärtää ja perustella itselleen. Tämä johtuu 
siitä, että pohjimmiltaan termodynamiikka käsittelee molekyylien energioiden tilastollisia ominaisuuksia.

\textbf{Tilastollinen termodynamiikka} redusoi klassisen termodynamiikan dynamiikkaan (ja kvanttimekaniikkaan).
\footnote{Vastaavasti kuin kemia pystyttiin aikoinaan redusoimaan kvanttimekaniikkaan.} 

Klassista termodynamiikkaa syvällisempänä ja tilastollisena tieteenä sitä on vaikeampi soveltaa käytäntöön. 
Toisaalta tilastollisen termodynamiikan käsitteillä termodynaamiset lainalaisuudet on mahdollista selittää ja 
perustella tyydyttävästi.

Tässä kirjassa opetellaan ennen kaikkea soveltamaan klassista termodynamiikkaa lämpöteknisiin ongelmiin. 
Kun se on ymmärryksen kannalta tarpeellista, käytän tilastollista termodynamiikkaa selittämään asioita.

\section{Mitä on energia?}

Ennen kuin sukellamme varsinaiseen termodynamiikkaan, on tärkeää selvittää, mitä energia on. Mielenkiintoista 
kyllä, energian määritelmä ei varsinaisesti kuulu termodynamiikkaan vaan mekaniikkaan. Energian määritelmä on 
seuraavanlainen:

\begin{quote}
 Energia on kykyä tehdä työtä.
\end{quote}

\subsection{Työ W}

Mitä työ sitten on? Mekaniikassa \textbf{työn W} yleinen määritelmä on

\begin{equation}
\label{eq:W}
 W = \int_S \vec{F} \cdot \vec{ds}
\end{equation}

Mitä tämä sitten tarkoittaa? 

Arkisestikin voimme todeta, että jonkin kappaleen siirtämisen ``työläys'' on suoraan verrannollinen

\begin{enumerate}
 \item Voimaan $F$, joka tarvitaan kappaleen liikuttamiseksi
 \item Matkaan $s$, joka kappaletta siirretään
\end{enumerate}

Kun voima on vakio ja reitti koko ajan voiman suuntainen, nämä verrannollisuudet voidaan yhdistää tuloksi ja 
(valitsemalla määritelmässä verrannollisuuskertoimeksi 1) määritellä työ

\begin{equation}
 W = |\vec{F}|s
\end{equation}

Yleisessä tapauksessa ei päästä näin helpolla, sillä kappaleeseen vaikuttavan voiman suuruus ja suunta voivat 
riippua esimerkiksi ajasta ja kappaleen paikasta eikä reittikään ole välttämättä lähelläkään suoraa. 

Onneksi 
mikä tahansa infinitesimaalisen lyhyt reitin pätkä $\vec{ds}$ voidaan katsoa hyvällä tarkkuudella suoraksi ja 
voiman reitin suuntainen komponentti saadaan pistetulolla eli

\begin{equation}
 dW = \vec{F} \cdot \vec{ds}
\end{equation}

Kun nämä infinitesimaalisen lyhyet reitin pätkät sitten summataan eli integroidaan saadaan työn yleinen määritelmä 
\ref{eq:W}.

Työn määritelmä voitaisiin avata sanallisesti vaikka seuraavasti:

\begin{quotation}
``Kun kappale, johon voima $\vec{F}$ vaikuttaa, kulkee 
reitin $S$ tekee voima kaavasta \ref{eq:W} laskettavissa olevan määrän työtä.''
\end{quotation}

 Huomioi, että tämä ei vaadi, että 
juuri voima $\vec{F}$ aiheuttaisi kappaleen liikkeen.

\subsection{Energia E}

Nyt kun olemme kerranneet työn määritelmän, käsitelkäämme vielä \textbf{energian E} määritelmä. 

Energia on siis kykyä tehdä työtä. Systeemin yhteydessä ja työn ymmärtämisen kautta se tarkoittaa itse 
asiassa seuraavaa:

\begin{quotation}
 Systeemillä on energiaa, kun on mahdollista löytää toinen systeemi, johon systeemi voi kohdistaa 
 voiman $\vec{F}$ kun toinen systeemi liikkuu reitin $S$.
\end{quotation}

Systeemin ei tarvitse kyetä tuottamaan mielivaltaista voimaa mielivaltaiselle reitille vaan riittää että 
voidaan keksiä jokin järjestely mitä analysoitaessa kaavaa \ref{eq:W} voidaan soveltaa tarkasteltavan 
systeemin tuottamaan voimaan $\vec{F}$.

Kääntäen voidaan todeta että kaikilla systeemeillä, jotka tuottavat johonkin muuhun voiman edes 
infinitesimaalisen lyhyellä matkalla, on energiaa.

Käsiä heiluttelevana loppukaneettina totean, että energia on siis ``voimantuottokykyä''\footnote{
Mietipä tätä: \textit{voimanlähteellä} on aina jokin \textit{teho} $P = \dfrac{dE}{dt}$!}.

\chapter{Systeemin energiat} % -----------------------------------------------

\section{Potentiaalienergia ja liike-energia}

Kuten muistamme, työn yleinen määritelmä on

\begin{equation}
 W = \int_S \vec{F} \cdot \vec{ds}
\end{equation}

ja systeemillä on energiaa mikäli se kykenee tuottamaan kaavassa toimivan voiman $\vec{F}$.

Reitti $S$ on vain mielivaltainen avaruuskäyrä eikä sitä voi analysoida sen enempää\footnote{
Ainakaan millään käytännössä kovin hyödyllisellä tavalla.}.

Sen sijaan kiinnostava ja vastattavissa oleva kysymys on:

\begin{quote}
 Millainen työn määritelmässä esiintyvä voima $\vec{F}$ voi olla? 
\end{quote}

Fysiikassa on onnistuttu palauttamaan kaikki mahdolliset maailmankaikkeudessa esiintyvät voimat 
\textit{neljään perusvuorovaikutukseen} kuuluviksi:

\begin{enumerate}
 \item Gravitaatio
 \item Sähkömagnetismi
 \item Heikko ydinvoima
 \item Vahva ydinvoima
\end{enumerate}

Näiden vuorovaikutusten voimien suuruus riippuu etäisyydestä voiman aiheuttajaan ja mahdollisesti myös ajasta. 
Tätä kuvataan \textit{voimakentillä} eli määrittämällä voima ajan ja paikan funktiona $\vec{F}(\vec{r}, t)$. 

Esimerkiksi sähkövarauksen $q_1$ toiseen sähkövaraukseen $q_2$ aiheuttama voimakenttä on

\begin{equation}
 \vec{F}(\vec{r}) = \frac{1}{4\pi\epsilon_0}\frac{q_1q_2}{|\vec{r}|^2}\hat{r}
\end{equation}

ja massan $m_1$ toiseen massaan $m_2$ aiheuttama voima on (Newtonin painovoimateoriassa)

\begin{equation}
 \vec{F}(\vec{r}) = G\frac{m_1m_2}{|\vec{r}|^2}\hat{r}
\end{equation}

\subsection{Potentiaalienergia $E_p$}

\begin{quotation}
Systeemillä on \textbf{potentiaalienergiaa $E_p$}, mikäli se on jossain siihen vaikuttavassa voimakentässä 
paikassa, josta se voi liikkua sellaisen reitin että voimakenttä tekee siihen positiivisen työn.
\end{quotation}

Potentiaalienergia on siis systeemin mahdollisuus saada voimakenttä tekemään siihen työtä. Eikö silloin 
energia ole itse asiassa voimakentän aiheuttavalla systeemillä? Pohjimmiltaan kyllä.

Mutta kun systeemi kulkee reittinsä voimakentässä se voi vuorostaan aiheuttaa voiman johonkin toiseen 
systeemin, joka voi tällöin kulkea jonkin reitin - eli systeemi tekee työtä! Se toki ``vain'' välittää 
voimakentän energiaa, mutta käytännössä näin voidaan esimerkiksi tehdä työtä johonkin, mihin voimakenttä 
ei kohdistu ainakaan toivotulla tavalla.

Esimerkiksi gravitaatiokentän energialla on vaikea saada elektroneja liikkumaan maanpinnan suuntaisesti, mutta 
siinä voidaan onnistua välillisesti näin:

\begin{enumerate}
 \item Vesimassalla on potentiaalienergiaa, sillä se on korkealla gravitaatiokentässä.
 \item Kun vesi päästetään putoamaan, gravitaatiokenttä tekee siihen työtä.
 \item Vesi kulkee turbiinin läpi ja tekee siihen työtä.
 \item Turbiini tekee työtä generaattorin roottoriin.
 \item Generaattorin roottori aiheuttaa käämien elektroneihin sähkömagneettisen voiman, joka tekee niihin 
 työtä.
 \item Elektronit liikkuvat johtimessa. Meillä on sähkövirtaa!
\end{enumerate}

Systeemille voidaan antaa potentiaalienergiaa liikuttamalla sitä voimakentässä niin, että voimakentän 
tekemä työ on negatiivinen. Tällöin joudutaan tuottamaan jokin toinen voima tekemään vastaavan 
suuruinen positiivinen työ. Tämän voiman tuottamiseen käytetty energia saadaan näin varastoitua 
potentiaalienergiaksi.

\subsection{Liike-energia $E_k$}

\begin{quotation}
 Systeemillä on liike-energiaa, kun se voi pienentämällä nopeuttaan tehdä työtä.
\end{quotation}

Newtonin toinen laki kertoo voiman, jonka hidastuva systeemi aiheuttaa:

\begin{equation}
 \vec{F} = m\vec{a}
\end{equation}

Sijoitetaan se työn määritelmään:

\begin{equation}
 W = \int_S \vec{F} \cdot \vec{ds} = \int_S m\vec{a} \cdot \vec{ds} = m\int_S \vec{a} \cdot \vec{ds}
\end{equation}

Käytetään kiihtyvyyden ja nopeuden määritelmiä 

\begin{align}
\vec{a} &= \frac{d\vec{v}}{dt} \\
\vec{v} &= \frac{d\vec{s}}{dt} \Leftrightarrow d\vec{s} =  \vec{v}dt\\
\end{align}

Sijoitetaan ja sievennetään:

\begin{equation}
 W = m\int_S \vec{a} \cdot \vec{ds} = m\int_S \frac{d\vec{v}}{dt} \cdot \vec{v}dt = m\int_S d\vec{v} \cdot \vec{v}
\end{equation}

Avataan pistetulo integroimalla komponenteittain:

\begin{equation}
 W = m\int_S d\vec{v} \cdot \vec{v} = m \left(\int_S v_xdv_x + \int_S v_ydv_y + \int_S v_zdv_z\right)
\end{equation}

saadaan

\begin{equation}
 W = m \left(\int_S v_xdv_x + \int_S v_ydv_y + \int_S v_zdv_z\right) 
 = m \left(\frac{1}{2}v_x^2 + \frac{1}{2}v_y^2 + \frac{1}{2}v_z^2\right)
\end{equation}

josta saadaan lopulta nopeudesta saatavaksi työksi

\begin{equation}
 W = m \left(\frac{1}{2}v_x^2 + \frac{1}{2}v_y^2 + \frac{1}{2}v_z^2\right)
 = \frac{1}{2}m \left(v_x^2 + v_y^2 + v_z^2\right)
 = \frac{1}{2}m (\vec{v} \cdot \vec{v})
 = \frac{1}{2}m|\vec{v}|^2
\end{equation}

Liike-energia on siis

\begin{equation}
 E_k = W = \frac{1}{2}m|\vec{v}|^2
\end{equation}

Systeemille voidaan antaa liike-energiaa aiheuttamalla siihen nettovoima, joka tekee työtä systeemin 
nopeuden kasvattamiseksi. Tähän kulunut energia varastoituu systeemin liike-energiaksi.

\section{Energiat lämpötieteissä}

Systeemillä voi siis olla montaa erityyppistä energiaa:

\begin{enumerate}
 \item Liike-energiat:
 \begin{enumerate}
  \item \underline{Systeemin liike-energia}
  \item \underline{Lämpöliikkeen liike-energia eli sisäenergia}
 \end{enumerate}
 \item Fysiikan perusvoimiin liittyvät potentiaalienergiat:
 \begin{enumerate}
  \item \underline{Gravitaation potentiaalienergia}
  \item Sähköiset energiat:
  \begin{enumerate}
   \item Sähkömagneettinen potentiaalienergia
   \item \textit{Kemiallinen sidosenergia}
  \end{enumerate}
  \item \textit{Heikon vuorovaikutuksen sidosenergia}
  \item Vahvan vuorovaikutuksen sidosenergia
 \end{enumerate}
\end{enumerate}

\underline{Alleviivattuja energioita} käsitellään lämpötieteissä jatkuvasti. \textit{Kursivoidut 
energiat} tulevat kyseeseen, mikäli systeemissä tapahtuu kemiallisia tai ydinreaktioita. 

Sähkömagneettisella potentiaalienergialla on väliä vain mikäli systeemin kokonaisvaraus ei ole nolla ja 
se on ulkoisessa sähkömagneettisessa kentässä. Lämpötieteissä ja -tekniikassa tällaiset tapaukset ovat 
harvinaisia.

Vahvan vuorovaikutuksen sidosenergia sitoo kvarkit protoneiksi ja neutroneiksi. Sillä ei toistaiseksi 
ole käytännön sovelluksia.

\section{Energioiden koordinaatistoriippuvuus}

Systeemin energiat voidaan jakaa myös seuraavasti:

\begin{enumerate}
 \item Koordinaattijärjestelmän ja/tai potentiaalienergian nollakohdan valinnasta riippuvat:
 \begin{enumerate}
  \item Systeemin liike-energia
  \item Gravitaation potentiaalienergia
  \item Sähkömagneettinen potentiaalienergia
 \end{enumerate}
 \item Näistä valinnoista riippumattomat:
 \begin{enumerate}
  \item Lämpöliikkeen liike-energia eli sisäenergia
  \item Kemiallinen sidosenergia
  \item Heikon vuorovaikutuksen sidosenergia
  \item Vahvan vuorovaikutuksen sidosenergia
 \end{enumerate}
\end{enumerate}

\section{Systeemin energiasuureet}

\subsection{Lämpöliikkeen liike-energia eli sisäenergia U}

Vaikka systeemi kokonaisuutena ei olisi liikkeessä, ovat siihen kuuluvat molekyylit jatkuvasti 
\textit{lämpöliikkeessä}\footnote{Mikäli systeemin lämpötila on yli 0 K.}. Tämä lämpöliike jakautuu kolmeen tyyppiin: molekyylien 
siirtymiseen eli \textit{translaatioon}, ei-pallosymmetristen molekyylien pyörimiseen omien symmetria-akseleidensa ympäri eli 
\textit{rotaatioon} ja moniatomiseen molekyyliin kuuluvien atomien värähtelyyn toistensa suhteen eli \textit{vibraatioon}. 
Molekyylit voivat siis liikkua, pyöriä ja värähdellä vaikka systeemi kokonaisuutena ei tekisi mitään näistä asioista.

Missä on liikettä, siellä on myös liike-energiaa. Termodynamiikassa lämpöliikkeen energiaa kutsutaan systeemin 
\textbf{sisäenergiaksi $U$}. \textbf{Ominaissisäenergia $u$} on vain systeemin lämpötilan funktio:

\begin{equation}
 u = u(T)
\end{equation}

Tämä ominaisuus on seurausta siitä, että myös lämpötila on pohjimmiltaan samaa ilmiötä eli lämpöliikkeen voimakkuutta mittaava suure.

\subsection{Sisäenergian johdannaisenergiat}

\subsubsection{Entalpia H} \label{sssection:entalpia}

\textbf{Entalpia $H$} on vain apusuure, joka on määritelty seuraavasti:

\begin{equation}
 h = u + pv
\end{equation}

Näistä kaavoista nähdään että myös ominaisentalpia on ominaisenergiaa ja sisältää ominaissisäenergian sekä termin $pv$ 
suuruisen lisäominaisenergian. Mikä sitten on entalpian fysikaalinen merkitys ja käytännön hyöty?

% h = h(T)?

\paragraph{Lämmönsiirto vakiopaineessa}

Olkoon meillä vakiopaineinen\footnote{``Isobaarinen''.} systeemi, johon tai josta siirtyy lämpöä. Vakiopaineinen systeemi 
ei välttämättä ole vakiotilavuuksinen\footnote{``Isokoorinen''.}, eli sen tilavuus voi muuttua jolloin systeemi tekee 
työtä ulkoisia painevoimia vastaan tai ympäristö tekee työtä systeemin painevoimia vastaan. 

Siirtyvä lämpö voi nyt  
olla positiivinen (systeemiin) tai negatiivinen (systeemistä). Työ aiheutuu lämmön aikaansaamasta tilavuuden muutoksesta. 
Mikäli lämpöä tuodaan systeemiin, sen tilavuus kasvaa lämpölaajenemisen johdosta ja systeemi tekee työtä ympäristöön. 
Tilavuudenmuutoksen merkki on siis sama kuin lämmön ja työn merkki on päinvastainen.

Termodynamiikan ensimmäisen pääsäännön mukaan differentiaaliselle lämmöntuonnille pätee

\begin{equation}
 du = dq + dw
\end{equation}

Differentiaalinen tilavuudenmuutostyö $dw$ voidaan korvata seuraavasti:

\begin{equation}
 dw = -pdv
\end{equation}

Ja sijoittaa I pääsääntöön:

\begin{equation}
\label{psIpdv}
 du = dq - pdv
\end{equation}

Tästä nähdään että sisäenergian muutos on erisuuri kuin tuotu lämpömäärä:

\begin{equation}
 du \neq dq
\end{equation}

Pidemmän päälle kaavan \ref{psIpdv} muistaminen johtaisi työläyteen (ja luultavasti myös huolimattomuusvirheisiin). Entalpian 
differentiaalinen muutos on yleisesti

\begin{equation}
 dh = du + pdv + vdp
\end{equation}

Vakiopaineessa $dp = 0$, joten

\begin{equation}
 dh = du + pdv + vdp = du + pdv 
\end{equation}

ja tästä saadaan ratkaistua sisäenergian muutos

\begin{equation}
 du = dh - pdv
\end{equation}

Sijoitetaan yhtälöön \ref{psIpdv}:

\begin{equation}
 dh - pdv = dq - pdv
\end{equation}

Tilavuudenmuutostyö supistuu ja

\begin{equation}
 dh = dq
\end{equation}

Koska yhtäsuuruus on näin yksinkertainen, se voidaan suoraan yleistää muillekin kuin differentiaalisille lämpömäärille:

\begin{equation}
 \Delta h = \Delta q
\end{equation}

Eli kun systeemistä tai systeemiin siirtyy lämpöä vakiopaineessa systeemin entalpia muuttuu tuodun lämmön verran. Tämä 
on erityisen kätevää kemiallisia reaktioita ja lämpövoimakoneita käsiteltäessä; kun seurataan sisäenergian sijaan 
entalpian kehitystä voidaan reaktiolämmöt ja lämmönsiirtimissä siirtyvä lämpö lisätä tai vähentää suoraan siitä.

\paragraph{Virtausenergia}
% Kuva!

Entalpian määritelmässä esintyvä termi $pdv$ voidaan ymmärtää myös aineen siirtymisen vaatimaksi energiaksi. 

\textbf{Huom:}

\begin{enumerate}
 \item Tässä on kysessä työ, joka tehdään (virtausaineesta koostuvaa) systeemiä siirrettäessä. Kyse ei ole siis liike-energiasta, 
  joka on oma terminsä.
 \item Tämä työ ei ole verrannollinen systeemin nopeuteen toisin kuin liike-energia. (Muistathan että $v \neq \left|\vec{v}\right|$!)
\end{enumerate}

Selvennän tätä entalpian tulkintaa esimerkillä.

Olkoon meillä putki, jossa virtaa jotain virtausainetta. Valitaan putken sisäpoikkileikkauksen (ala $A$) muotoinen ja $L$:n 
pituinen kontrollitilavuus avoimeksi systeemiksemme. Kun systeemin ajanhetkellä $t_1$ sisältämä virtausaine on 
ajanhetkellä $t_2$ siirtynyt juuri kokonaisuudessaan ulos systeemistä on sen täytynyt tehdä edessään olevia 
painevoimia vastaan työ, jonka suuruus on

\begin{equation}
 W = |\vec{F}| L = pAL
\end{equation}

Huomataan, että systeemin tilavuushan on $AL$, joten

\begin{equation}
 W = pV
\end{equation}

ja ominaissuureilla 

\begin{equation}
 w = pv
\end{equation}

joka esiintyy ominaisentalpian määritelmässä. Tässä tulkinnassa systeemin ominaisentalpia sisältää siis

\begin{enumerate}[a)]
 \item Sisäenergian eli lämpöliikkeen energian
 \item Virtausenergian eli virtauksen siirtotyön tekevän energian
\end{enumerate}

Tämä entalpian ominaisuus taas on kätevä niissä lukemattomissa lämpötekniikan sovellutuksissa, missä prosessissa on 
olennaisessa osassa putkessa virtaava virtausaine. Nimittäin entalpiaa näin käyttämällä virtauksen jatkuminen muuttuu 
analyysissa ikäänkuin sisäänrakennetuksi itsestäänselvyydeksi ja voidaan keskittyä tavoitteen kannalta kiinnostavampiin 
ilmiöihin, esim. lämmönsiirtoon.

\subsubsection{Helmholtzin vapaaenergia F} \label{sssection:helmholtz}

Myös \textbf{Helmholtzin vapaaenergia} $F$ on apusuure, ja se on määritelty näin:

\begin{equation}
 F = U - TS
\end{equation}

Helmholtzin vapaaenergia on siis sisäenergian sekä entropian ja lämpötilan tulon erotus. Mutta mikä on sen 
käytännön merkitys?

\paragraph{Lämmönsiirto vakiolämpötilassa ja -tilavuudessa}

Olkoon meillä systeemi, josta tai johon siirtyy lämpöä. Prosessi tapahtuu niin, että systeemin lämpötila ja 
tilavuus ovat vakioita\footnote{``Isoterminen ja isokoorinen prosessi''.}. 
Koska tilavuus on vakio, edes tilavuudenmuutostyötä ei tapahdu:

\begin{align}
 T(t) & = T\\
 V(t) & = V\\
 \Delta U & = Q_{siirto} + W = Q_{siirto}
\end{align}

eli 

\begin{equation}
\label{eqn:isoTV}
 \Delta U = Q_{siirto}
\end{equation}

Systeemiin lämpönä tuotu energia menee siis systeemin sisäenergian (molekyylien kineettinen energia) kasvattamiseen.

Lämpöä voi tulla systeemiin kahdesta lähteestä; systeemin ulkopuolelta siirtymällä ja sen sisältä kun jokin muu 
energia muuttuu epäjärjestyneeseen muotoon (esimerkiksi turbulenssi muuttaa molekyylien ``koordinoitunutta'' 
liike-energiaa satunnaiseksi eli lämpöliikkeen liike-energiaksi).

\begin{equation}
\label{eqn:QQQ}
 Q = Q_{siirto} + Q_{h\ddot{a}vi\ddot{o}t}
\end{equation}

Kokonaislämpö voidaan lausua entropian muutoksen kautta entropian ja lämpötilan avulla ja
koska lämpötila on vakio:

\begin{equation}
\label{eqn:DeltaS}
 \Delta S = \frac{Q}{T} \Leftrightarrow Q = T\Delta S
\end{equation}

Sijoitetaan kaava \ref{eqn:DeltaS} kaavaan \ref{eqn:QQQ}:

\begin{equation}
 T\Delta S = Q_{siirto} + Q_{h\ddot{a}vi\ddot{o}t}
\end{equation}

Häviöt synnyttävät aina positiivisen lämpömäärän, joten

\begin{align}
 Q_{h\ddot{a}vi\ddot{o}t} & \geq 0\\
 \Rightarrow T\Delta S & \geq Q_{siirto}\\
 \label{eqn:QTSsiirto}
 \Leftrightarrow Q_{siirto} & \leq T \Delta S
\end{align}

Sijoitetaan kaava \ref{eqn:QTSsiirto} kaavaan \ref{eqn:isoTV}:

\begin{equation}
 \Delta U = Q_{siirto} \leq T\Delta S
\end{equation}

eli 

\begin{align}
 & \Delta U \leq T\Delta S\\
 & \Delta U - T\Delta S \leq 0\\
 & \Delta(U - TS) \leq 0\\
 & \Delta F \leq 0
\end{align}

Siis kun

\begin{align}
 T(t) = T\\
 V(t) = V\\
 W = 0
\end{align}

On Helmholtzin vapaaenergian muutoksen oltava

\begin{equation}
 \Delta F \leq 0
\end{equation}

\paragraph{Fysikaalinen ja kemiallinen merkitys}

Kun systeemi on vakiolämpötilassa ja -tilavuudessa eikä sisäisiä häviöitä tapahdu, on 

\begin{equation}
 \Delta F = 0
\end{equation}

Ja systeemi on termodynaamisessa tasapainossa.

Mikäli

\begin{equation}
 \Delta F < 0
\end{equation}

tarkoittaa se sitä, että systeemissä muut energian muodot muuttuvat häviöiden kautta lämmöksi 
kunnes systeemi on termodynaamisessa tasapainossa.

\subsubsection{Gibbsin vapaaentalpia G}

Kuten entalpia ja Helmholtzin vapaaenergia, \textbf{Gibbsin vapaaentalpia} $G$ on apusuure, 
ja se on määritelty näin:

\begin{equation}
 G = H - TS
\end{equation}

Gibbsin vapaaentalpian määritelmä näyttää hyvin samankaltaiselta kuin Helmholtzin vapaaenergian määritelmä. 
Miksiköhän?

\paragraph{Lämmönsiirto vakiolämpötilassa ja -paineessa}

Huomaa, että seuraava päättely etenee hyvin samalla tavalla kuin Helmholtzin vapaaenergian tapauksessa.

Siirretään lämpöä systeemistä tai systeemiin vakiolämpötilassa ja -paineessa:

\begin{align}
 T(t) & = T\\
 p(t) & = p\\
 \Delta U & = Q_{siirto} + W
\end{align}

Nyt tilavuudenmuutostyötä voi esiintyä ja analyysi voisi mennä monimutkaiseksi, muttei mene jos muistamme 
kappaleessa \nameref{sssection:entalpia} saamaamme tulosta, eli että vakiopaineessa:

\begin{equation}
 \Delta H = Q_{siirto}
\end{equation}

ja kuten kappaleessa \nameref{sssection:helmholtz} päättelimme

\begin{equation}
 Q_{siirto} \leq T \Delta S
\end{equation}

Kun yhdistämme nämä kaavat, saamme

\begin{equation}
 \Delta H = Q_{siirto} \leq T\Delta S
\end{equation}

Ja edelleen: 

\begin{align}
 & \Delta H \leq T\Delta S\\
 & \Delta H - T\Delta S \leq 0\\
 & \Delta(H - TS) \leq 0\\
 & \Delta G \leq 0
\end{align}

Siis kun

\begin{align}
 T(t) = T\\
 p(t) = p\\
 W = 0
\end{align}

On Gibbsin vapaaentalpian muutoksen oltava

\begin{equation}
 \Delta G \leq 0
\end{equation}

\paragraph{Fysikaalinen ja kemiallinen merkitys}

Kun systeemi on vakiolämpötilassa ja -paineessa eikä sisäisiä häviöitä tapahdu, on 

\begin{equation}
 \Delta G = 0
\end{equation}

Ja systeemi on termodynaamisessa tasapainossa.

Mikäli

\begin{equation}
 \Delta G < 0
\end{equation}

tarkoittaa se sitä, että systeemissä muut energian muodot muuttuvat häviöiden kautta lämmöksi 
kunnes systeemi on termodynaamisessa tasapainossa.

\chapter{Siirtymäenergiat}

Systeemillä voi siis olla monenlaista energiaa. Myös systeemin ympäristöllä voi olla näitä 
energioita.

Sillä hetkellä kun energia siirtyy systeemin rajan yli voi se kuitenkin termodynamiikassa 
olla vain kahta tyyppiä: \textbf{lämpöä $Q$} tai \textbf{työtä $W$}. Vastaavasti systeemillä 
tai ympäristöllä sinänsä ei voi olla näitä energioita vaan varastoituessaan systeemiin tai 
ympäristöön ne muuttuvat aina johonkin muuhun muotoon.

\section{Merkkisopimus} \label{sec:merkkisopimus}

Systeemiin siirtyvä työ ja lämpö ovat aina positiivisia, systeemistä siirtyvät negatiivisia.

Päinvastoin voidaan todeta että mikäli työtä tai lämpöä ei alunperin tiedetä, ratkaistun 
työn tai lämmön etumerkki kertoo, siirtyikö se systeemistä vai systeemiin.

\section{Lämpö Q}

Lämpö on energiaa, joka siirtyy systeemin rajojen yli siksi, että systeemin ja ympäristön 
välillä on lämpötilaero. 

Siirtyminen systeemin rajojen yli voi tapahtua millä tahansa \textbf{lämmönsiirtotavalla}:

\begin{itemize}
 \item johtumalla
 \item kulkeutumalla\footnote{``Konvektiolla''.}
 \item säteilemällä
 \item tai jollain näiden 
yhdistelmällä\footnote{Erityisen merkittävä yhdistelmä on johtumisesta ja kulkeutumisesta 
koostuva ``\textbf{konvektiivinen lämmönsiirto}''.}.
\end{itemize}

Lämpö siirtyy aina korkeammasta lämpötilasta matalampaan. Kuten myöhemmin selviää, tämä 
on seurausta termodynamiikan II pääsäännöstä. Aina kun lämpöä siirtyy, joko systeemin tai 
ympäristön entropia ja sen myötä maailmankaikkeuden epäjärjestys kasvaa.
Tähän liittyen lämpö on ``epäjärjestynyttä energiaa'', jolla voidaan tehdä vähemmän 
erilaisia asioita kuin työllä.

Lämmönsiirto muuttaa ensisijaisesti systeemin tai ympäristön sisäenergiaa. (Mieti, 
miten tämä liittyy siihen, että sisäenergia on ``lämpöliikkeen liike-energiaa''.)

\section{Työ W}

Työ on lämpöä monimuotoisempi ja vaikeammin määriteltävissä oleva siirtymäenergia. 

Työ voi olla esimerkiksi systeemin tilavuudenmuutostyötä, turbiinin akselityötä, 
sähkövirran energiaa jne.

Työ on ``järjestynyttä energiaa'', joka voidaan muuttaa vaihtelevilla hyötysuhteilla 
moniksi muiksi energian muodoiksi. Kun pelkästään työtä siirtyy ei systeemin tai ympäristön 
entropia muutu.

Työ voidaan muuttaa myös sisäenergian kautta kokonaan lämmöksi. Lämpöä taas ei voida 
(sisäenergiankaan kautta) muuttaa kokonaan työksi, koska se rikkoisi termodynamiikan 
II pääsääntöä.

\chapter{Termodynamiikan I pääsääntö}

\textbf{Termodynamiikan I pääsääntö} on seuraava kokeellisesti havaittu luonnonlaki:

\begin{quote}
 Energia säilyy.
\end{quote}

Sen voi ilmaista myös seuraavilla tavoilla:

\begin{quote}
 Energia on tuhoutumatonta
\end{quote}

\begin{quote}
 Energia ei tuhoudu, ainoastaan muuttaa muotoaan.
\end{quote}

\begin{quote}
 Maailmankaikkeuden kokonaisenergia on vakio.
\end{quote}

Termodynamiikan I pääsäännön voi määritellä myös käytännössä hyödyllisellä tavalla 
systeemin energioiden ja siirtymäenergioiden avulla. 

Mikäli energiaa siirtyy systeemiin, 
täytyy systeemin energian kasvaa. Mikäli energiaa siirtyy systeemistä, täytyy systeemin 
energian pienentyä. \hyperref[sec:merkkisopimus]{Merkkisopimuksen} ansiosta seuraava 
lause kattaa nämä molemmat tapaukset:

\begin{quote}
 Systeemin kokonaisenergian muutos = siirtymäenergioiden summa
\end{quote}

Siistissä täysin matemaattisessa kaavamuodossa termodynamiikan I pääsääntö on siis

\begin{equation}
 \Delta E_{sys} = Q_{tot} + W_{tot}
\end{equation}

Termodynamiikan pääsäännöt ovat kaikkien fysiikan lakien tavoin \textbf{universaaleja} 
eli voimassa kaikille systeemeille ja prosesseille kaikkialla, kaikkina aikoina. Voimme 
siis aina käyttää laskelmissamme termodynamiikan I pääsääntöä tässä muodossa yhtenä 
yhtälöistämme.

% \chapter{Suureet} %---------------------------------------------------------------------
% % Tarvittavat prujuista ja kirjoista
% 
% \section{Aika ja avaruus}
% 
% \subsection{Aika}
% 
% \subsection{Pituus, pinta-ala ja tilavuus}
% 
% \section{Aine ja energia}
% 
% \subsection{Ainemäärä}
% 
% \subsection{Massa}
% 
% \subsection{Tiheys}
% 
% \subsection{Voima}
% 
% \subsection{Energia}
% 
% \section{Ekstensiivi- ja intensiivisuureet}
% 
% \section{Termodynaamiset suureet}
% 
% \subsection{Viskositeetti}
% 
% \subsection{Lämmönjohtavuus}
% 
% \subsection{Terminen diffusiviteetti}
% 
% \section{Tilasuureet l. termodynaamiset potentiaalit} 
% 
% \subsection{Paine}
% 
% \subsection{Tiheys ja ominaistilavuus}
% 
% \newpage % A temporary hack
% 

% 
% \subsection{Lämpötila}
% 
% \subsection{Ominaislämpökapasiteetit}
% 
% Lämpötila, sisäenergia ja sisäenergian sisältävä entalpia mittaavat siis enemmän tai vähemmän samaa asiaa. Ne voidaan 
% näinollen luultavasti kytkeä toisiinsa jollakin yksinkertaisella tavalla. Tästä kytkennästä on myös se olennainen hyöty, 
% että laskennassa hyödylliset mutta vaikeasti mitattavat sisäenergia ja entalpia saadaan kytkettyä harvemmin kiinnostavaan 
% mutta helposti mitattavaan lämpötilaan.

% Termodynamiikassa kytkentään käytetään \textbf{ominaislämpökapasiteettia vakiotilavuudessa $c_v$} ja 
% \textbf{ominaislämpökapasiteettia  vakiopaineessa $c_p$}. Määritellään nämä ominaislämpökapasiteetit. Sisäenergian ja 
% entalpian differentiaaliset muutokset voi kytkeä lämpötilan differentiaaliseen muutokseen osittaisderivaattojen avulla:
% 
% \begin{align}
%  du = \frac{\partial u}{\partial T}\bigg|_vdT\\
%  dh = \frac{\partial h}{\partial T}\bigg|_pdT
% \end{align}
% 
% Nämä derivoinnin tuloksena syntyvät funktiot on nimetty ominaislämpökapasiteeteiksi (lyhyesti ``ominaislämmöiksi'') 
% vakiotilavuudessa ja vakiopaineessa:
% 
% \begin{align}
%  \frac{\partial u}{\partial T}\bigg|_v = c_v(T)\\
%  \frac{\partial h}{\partial T}\bigg|_p = c_p(T)\\
%  du = c_vdT\\
%  dh = c_pdT
% \end{align}
% 
% Yleisesti ottaen ominaislämmöt eivät ole yksinkertaisia tai helposti teoreettisesti johdettavissa olevia lämpötilan 
% funktioita. 
% 
% Käytännön laskennassa ominaislämmöille käytetään taulukoituja, käyräksi piirrettyjä tai kokeellisen 
% polynomiapproksimaation muodossa olevia funktioita. Jos toimitaan kapealla lämpötila-alueella ja tulokset on 
% tärkeämpää saada nopeasti kuin tarkkoina voidaan ominaislämpö olettaa vakioksi lämpötila-alueella. Tällaisia tilanteita 
% ovat esim. alustavat tunnustelulaskelmat, pika-analyysit ja tentit.
% 
% % \frac{c_p}{c_v} = \gamma, c_p - c_v = R
% % Kokoonpuristumattomattomalle c_p = c_v
% 
% \subsection{Entropia}
% 
% \subsection{Helmhotzin vapaaenergia}
% 
% \subsection{Gibbsin vapaaenergia}
% 
% \subsection{Vapaat tilasuureet l. vapaat muuttujat l. vapausasteet}
% 
% \paragraph{Tilanyhtälö}
% 
% \section{Suureluettelo}
% % Kunkin suureen nimet, symbolit, yksiköt
% 
% \chapter{Säilymislait} % ---------------------------------------------------------------
% 
% \section{Taselaskenta}
% % reynoldsin siirtoteoreema
% 
% \section{Massan säilyminen}
% 
% \subsection{Suljettu systeemi}
% 
% \subsection{Avoin systeemi}
% 
% \section{Liikemäärän säilyminen}
% 
% \subsection{Suljettu systeemi}
% 
% \subsection{Avoin systeemi}
% 
% \section{Kulmaliikemäärän säilyminen}
% 
% \subsection{Suljettu systeemi}
% 
% \subsection{Avoin systeemi}
% 
% \section{Energian säilyminen}
% 
% \subsection{Suljettu systeemi}
% 
% \subsection{Avoin systeemi}
% 
% \chapter{Termodynamiikan pääsäännöt} % -------------------------------------------------
% 
% \section{Nollas pääsääntö: lämpötilan määritelmä}
% 
% \section{Ensimmäinen pääsääntö: energian säilyminen}
% 
% \subsection{Systeemin energia ja siirtyvä energia}
% 
% \subsection{Suljettu systeemi}
% 
% \subsection{Avoin systeemi}
% 
% \section{Toinen pääsääntö: epäjärjestyksen lisääntyminen}
% 
% \subsection{Energian laatu}
% 
% \subsection{Suljettu systeemi}
% 
% \subsection{Avoin systeemi}
% 
% \section{Kolmas pääsääntö: absoluuttinen nollapiste}
% 
% \subsection{Suljettu systeemi}
% 
% \subsection{Avoin systeemi}
% 
% \chapter{Muut lait} % ------------------------------------------------------------------
% % esim. fourierin "laki"
% 
% \part{Termodynamiikkaa}% ________________________________________________________________
% 
% \chapter{Termodynaamiset prosessit}
% % Huomaa tilasuureiden ja muiden ero (Q ja W eivät potentiaaleja)
% % Työn ja lämmöntuonti vakiopaineessa, -tilavuudessa, -lämpötilassa, entropiassa...
% % Polytrooppiprosessi, mielivaltaiset prosessit
% 
% \chapter{Aineen olomuodot ja faasimuutokset}
% % Kiinteä, neste, kaasu, ylikrittinen, plasma(...:P)
% % Faasidiagrammit
% 
% \part{Virtausoppia} % ___________________________________________________________________
% 
% \chapter{Virtausaineen käsite}
% 
% \part{Lämmönsiirtoa} % ___________________________________________________________________
% 
% \part{Matemaattisia menetelmiä} % ________________________________________________________
% 
% \chapter{Differentiaalilaskenta} % -------------------------------------------------------
% % Osittais- ja kokonaisdifferentiaalit
% 
% \section{Derivointi}
% % Osittais- ja kokonaisderivaatat
% 
% \section{Integrointi}
% 
% \section{Differentiaaliyhtälöt}
% 
% \chapter{Vektorianalyysi} % --------------------------------------------------------------
% 
% \section{Skalaari- ja vektorikentät}
% 
% \section{Aikaderivoidut suureet}
% % ``Virrat``, totaalinen derivaatta 
% 
% \chapter{Dimensioanalyysi} % -------------------------------------------------------------
% 
% \section{Dimensiottomat suureet}
% 
% \section{Lämpöteknisiä dimensiottomia suureita}
% % Suureet-osioon?
% 
% \part{Laskuharjoitusten ratkaisuja} % ____________________________________________________

\end{document}

