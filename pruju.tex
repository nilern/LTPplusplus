% Copyright 2013 Pauli Jaakkola
% 
% This program is free software: you can redistribute it and/or modify
% it under the terms of the GNU General Public License as published by
% the Free Software Foundation, either version 3 of the License, or
% (at your option) any later version.
% 
% This program is distributed in the hope that it will be useful,
% but WITHOUT ANY WARRANTY; without even the implied warranty of
% MERCHANTABILITY or FITNESS FOR A PARTICULAR PURPOSE.  See the
% GNU General Public License for more details.
% 
% You should have received a copy of the GNU General Public License
% along with this program.  If not, see <http://www.gnu.org/licenses/>.

\documentclass[12pt,a4paper,finnish]{book}

\usepackage[utf8]{inputenc}               % Tekstiasetuksia, sisältää ääkköset
\usepackage[T1]{fontenc}                    % Tekstiasetuksia, T1-koodatut fontit
\usepackage[finnish]{babel}                 % Suomenkielinen tavutus ja otsikot
\usepackage{a4wide}
\usepackage{amsmath}                        % Lisää kaavavoimaa!
\usepackage{enumerate}
\usepackage{hyperref}
\makeatother
\hypersetup{
  colorlinks=true,
  linkcolor=blue,
}

\title{LTP++\\Lämpötekniikan perusteita}
\date{\today}
\author{Pauli Jaakkola}

\begin{document}

\maketitle
% Creative Commons, LaTeX-maininnat tähän
\newpage
\pagenumbering{Roman}
\setcounter{tocdepth}{3}
\tableofcontents
\newpage
\pagenumbering{arabic}

% \chapter{Johdanto}
% 
% \part{Perustietoja} % ___________________________________________________________________

\setcounter{secnumdepth}{3}

\chapter{Systeemi} % -----------------------------------------------
% % Avoin, suljettu, adiabaattinen

Systeemin käsite on lämpötieteissä hyvin keskeinen ja hyödyllinen.

\section{Kontrollitilavuus} \label{section:kontrollitilavuus}

\textbf{Kontrollitilavuus} on mielivaltaisen avaruudessa sijaitsevan \textit{kontrollipinnan} sisältämä tilavuus. 
Pinnan sijainti ja muoto voi myös riippua ajasta. 

Kontrollitilavuus voidaan siis valita täysin vapaasti, 
mutta usein oikein valittu kontrollitilavuus helpottaa haluttujen tulosten saamista tai jopa ylipäätään mahdollistaa sen.

\section{Systeemin käsite}

\textbf{Systeemin} käsite on erityisesti termodynamiikassa keskeinen. Systeemiksi voidaan valita mikä tahansa 
\nameref{section:kontrollitilavuus}.

Vaikka automaatiolle keskeisessä systeemiteoriassa systeemin käsite on suomennettu \textit{järjestelmäksi},
lämpötieteissä puhutaan anglisistisesti \textit{systeemeistä}. Mahdollisesti sellaiset asiat kuin 
sisäenergia ja entropia merkityksineen luovat sellaista kuvaa, että lämpötieteelliset systeemit eivät 
yleensä ole pohjimmiltaan erityisen ``järjestelmällisiä'' tai ``järjestyksessä''.

\section{Ympäristön käsite}

\textbf{Ympäristö} käsittää termodynamiikassa kaiken systeemin ulkopuolella olevan. Yhdessä systeemi ja ympäristö 
muodostavat siis \textbf{maailmankaikkeuden}.

\section{Avoin ja suljettu systeemi}

\textbf{Avoimen systeemin} kontrollipinta on avoin eli sen läpi voi kulkea ainetta. Tyypillisiä esimerkkejä 
ovat \textit{lämmönsiirrin} ja \textit{turbiini}.

\textbf{Suljetun systeemin} kontrollipinta on suljettu eli sen läpi ei voi kulkea ainetta. Tyypillisiä 
esimerkkejä ovat \textit{suljettu kaasusäiliö} ja \textit{metallikuutio}.

\section{Eristämätön ja eristetty systeemi}

\textbf{Eristämättömän systeemin} kontrollipinnan läpi voi vapaasti siirtyä lämpöä. 
Tällainen on esimerkiksi \textit{ilmatilavuus keskellä muuta ilmaa}.

\textbf{Täydellisesti eristetyn eli adiabaattisen systeemin} kontrollipinnan läpi ei siirry lämpöä.

Täydellisesti eristettyjä systeemejä ei tietenkään ole todellisuudessa olemassa, vaan \textit{eristeillä} 
on jokin äärellinen lämpövastus, joka vähentää systeemistä poistuvan lämmön määrää. Hyvin eristettyjä 
systeemejä ovat esimerkiksi \textit{termospullo} ja \textit{passiivienergiatalo}.

\chapter{Systeemin energiat}

Lämpötieteissä ja lämpötekniikassa energia on keskeinen kiinnostuksen kohde. Systeemillä voi 
olla montaa erityyppistä energiaa:

\begin{enumerate}
 \item Liike-energiat:
 \begin{enumerate}
  \item \underline{Systeemin liike-energia}
  \item \underline{Lämpöliikkeen liike-energia eli sisäenergia}
 \end{enumerate}
 \item Fysiikan perusvoimiin liittyvät energiat:
 \begin{enumerate}
  \item \underline{Gravitaation potentiaalienergia}
  \item Sähköiset energiat:
  \begin{enumerate}
   \item Sähkömagneettinen potentiaalienergia
   \item \textit{Kemiallinen sidosenergia}
  \end{enumerate}
  \item \textit{Heikon vuorovaikutuksen sidosenergia}
  \item Vahvan vuorovaikutuksen sidosenergia
 \end{enumerate}
\end{enumerate}

\underline{Alleviivattuja energioita} käsitellään lämpötieteissä jatkuvasti. \textit{Kursivoidut 
energiat} tulevat kyseeseen, mikäli systeemissä tapahtuu kemiallisia tai ydinreaktioita. 

Sähkömagneettisella potentiaalienergialla on väliä vain mikäli systeemin kokonaisvaraus ei ole nolla ja 
se on ulkoisessa sähkömagneettisessa kentässä. Lämpötieteissä ja -tekniikassa tällaiset tapaukset ovat 
harvinaisia.

Vahvan vuorovaikutuksen sidosenergia sitoo kvarkit protoneiksi ja neutroneiksi. Sillä ei toistaiseksi 
ole käytännön sovelluksia.

\section{Energioiden koordinaatistoriippuvuus}

Systeemin energiat voidaan jakaa myös seuraavasti:

\begin{enumerate}
 \item Koordinaattijärjestelmän ja/tai potentiaalienergian nollakohdan valinnasta riippuvat:
 \begin{enumerate}
  \item Systeemin liike-energia
  \item Gravitaation potentiaalienergia
  \item Sähkömagneettinen potentiaalienergia
 \end{enumerate}
 \item Näistä valinnoista riippumattomat:
 \begin{enumerate}
  \item Lämpöliikkeen liike-energia eli sisäenergia
  \item Kemiallinen sidosenergia
  \item Heikon vuorovaikutuksen sidosenergia
  \item Vahvan vuorovaikutuksen sidosenergia
 \end{enumerate}
\end{enumerate}

\section{Systeemin energiasuureet}

\subsection{Lämpöliikkeen liike-energia eli sisäenergia U}

Vaikka systeemi kokonaisuutena ei olisi liikkeessä, ovat siihen kuuluvat molekyylit jatkuvasti 
\textit{lämpöliikkeessä}\footnote{Mikäli systeemin lämpötila on yli 0 K.}. Tämä lämpöliike jakautuu kolmeen tyyppiin: molekyylien 
siirtymiseen eli \textit{translaatioon}, ei-pallosymmetristen molekyylien pyörimiseen omien symmetria-akseleidensa ympäri eli 
\textit{rotaatioon} ja moniatomiseen molekyyliin kuuluvien atomien värähtelyyn toistensa suhteen eli \textit{vibraatioon}. 
Molekyylit voivat siis liikkua, pyöriä ja värähdellä vaikka systeemi kokonaisuutena ei tekisi mitään näistä asioista.

Missä on liikettä, siellä on myös liike-energiaa. Termodynamiikassa lämpöliikkeen energiaa kutsutaan systeemin 
\textbf{sisäenergiaksi $U$}. \textbf{Ominaissisäenergia $u$} on vain systeemin lämpötilan funktio:

\begin{equation}
 u = u(T)
\end{equation}

Tämä ominaisuus on seurausta siitä, että myös lämpötila on pohjimmiltaan samaa ilmiötä eli lämpöliikkeen voimakkuutta mittaava suure.

\subsection{Sisäenergian johdannaisenergiat}

\subsubsection{Entalpia H} \label{sssection:entalpia}

\textbf{Entalpia $H$} on vain apusuure, joka on määritelty seuraavasti:

\begin{equation}
 h = u + pv
\end{equation}

Näistä kaavoista nähdään että myös ominaisentalpia on ominaisenergiaa ja sisältää ominaissisäenergian sekä termin $pv$ 
suuruisen lisäominaisenergian. Mikä sitten on entalpian fysikaalinen merkitys ja käytännön hyöty?

% h = h(T)?

\paragraph{Lämmönsiirto vakiopaineessa}

Olkoon meillä vakiopaineinen\footnote{``Isobaarinen''.} systeemi, johon tai josta siirtyy lämpöä. Vakiopaineinen systeemi 
ei välttämättä ole vakiotilavuuksinen\footnote{``Isokoorinen''.}, eli sen tilavuus voi muuttua jolloin systeemi tekee 
työtä ulkoisia painevoimia vastaan tai ympäristö tekee työtä systeemin painevoimia vastaan. 

Siirtyvä lämpö voi nyt  
olla positiivinen (systeemiin) tai negatiivinen (systeemistä). Työ aiheutuu lämmön aikaansaamasta tilavuuden muutoksesta. 
Mikäli lämpöä tuodaan systeemiin, sen tilavuus kasvaa lämpölaajenemisen johdosta ja systeemi tekee työtä ympäristöön. 
Tilavuudenmuutoksen merkki on siis sama kuin lämmön ja työn merkki on päinvastainen.

Termodynamiikan ensimmäisen pääsäännön mukaan differentiaaliselle lämmöntuonnille pätee

\begin{equation}
 du = dq + dw
\end{equation}

Differentiaalinen tilavuudenmuutostyö $dw$ voidaan korvata seuraavasti:

\begin{equation}
 dw = -pdv
\end{equation}

Ja sijoittaa I pääsääntöön:

\begin{equation}
\label{psIpdv}
 du = dq - pdv
\end{equation}

Tästä nähdään että sisäenergian muutos on erisuuri kuin tuotu lämpömäärä:

\begin{equation}
 du \neq dq
\end{equation}

Pidemmän päälle kaavan \ref{psIpdv} muistaminen johtaisi työläyteen (ja luultavasti myös huolimattomuusvirheisiin). Entalpian 
differentiaalinen muutos on yleisesti

\begin{equation}
 dh = du + pdv + vdp
\end{equation}

Vakiopaineessa $dp = 0$, joten

\begin{equation}
 dh = du + pdv + vdp = du + pdv 
\end{equation}

ja tästä saadaan ratkaistua sisäenergian muutos

\begin{equation}
 du = dh - pdv
\end{equation}

Sijoitetaan yhtälöön \ref{psIpdv}:

\begin{equation}
 dh - pdv = dq - pdv
\end{equation}

Tilavuudenmuutostyö supistuu ja

\begin{equation}
 dh = dq
\end{equation}

Koska yhtäsuuruus on näin yksinkertainen, se voidaan suoraan yleistää muillekin kuin differentiaalisille lämpömäärille:

\begin{equation}
 \Delta h = \Delta q
\end{equation}

Eli kun systeemistä tai systeemiin siirtyy lämpöä vakiopaineessa systeemin entalpia muuttuu tuodun lämmön verran. Tämä 
on erityisen kätevää kemiallisia reaktioita ja lämpövoimakoneita käsiteltäessä; kun seurataan sisäenergian sijaan 
entalpian kehitystä voidaan reaktiolämmöt ja lämmönsiirtimissä siirtyvä lämpö lisätä tai vähentää suoraan siitä.

\paragraph{Virtausenergia}
% Kuva!

Entalpian määritelmässä esintyvä termi $pdv$ voidaan ymmärtää myös aineen siirtymisen vaatimaksi energiaksi. 

\textbf{Huom:}

\begin{enumerate}
 \item Tässä on kysessä työ, joka tehdään (virtausaineesta koostuvaa) systeemiä siirrettäessä. Kyse ei ole siis liike-energiasta, 
  joka on oma terminsä.
 \item Tämä työ ei ole verrannollinen systeemin nopeuteen toisin kuin liike-energia. (Muistathan että $v \neq \left|\vec{v}\right|$!)
\end{enumerate}

Selvennän tätä entalpian tulkintaa esimerkillä.

Olkoon meillä putki, jossa virtaa jotain virtausainetta. Valitaan putken sisäpoikkileikkauksen (ala $A$) muotoinen ja $L$:n 
pituinen kontrollitilavuus avoimeksi systeemiksemme. Kun systeemin ajanhetkellä $t_1$ sisältämä virtausaine on 
ajanhetkellä $t_2$ siirtynyt juuri kokonaisuudessaan ulos systeemistä on sen täytynyt tehdä edessään olevia 
painevoimia vastaan työ, jonka suuruus on

\begin{equation}
 W = |\vec{F}| L = pAL
\end{equation}

Huomataan, että systeemin tilavuushan on $AL$, joten

\begin{equation}
 W = pV
\end{equation}

ja ominaissuureilla 

\begin{equation}
 w = pv
\end{equation}

joka esiintyy ominaisentalpian määritelmässä. Tässä tulkinnassa systeemin ominaisentalpia sisältää siis

\begin{enumerate}[a)]
 \item Sisäenergian eli lämpöliikkeen energian
 \item Virtausenergian eli virtauksen siirtotyön tekevän energian
\end{enumerate}

Tämä entalpian ominaisuus taas on kätevä niissä lukemattomissa lämpötekniikan sovellutuksissa, missä prosessissa on 
olennaisessa osassa putkessa virtaava virtausaine. Nimittäin entalpiaa näin käyttämällä virtauksen jatkuminen muuttuu 
analyysissa ikäänkuin sisäänrakennetuksi itsestäänselvyydeksi ja voidaan keskittyä tavoitteen kannalta kiinnostavampiin 
ilmiöihin, esim. lämmönsiirtoon.

\subsubsection{Helmholtzin vapaaenergia F} \label{sssection:helmholtz}

Myös \textbf{Helmholtzin vapaaenergia} $F$ on apusuure, ja se on määritelty näin:

\begin{equation}
 F = U - TS
\end{equation}

Helmholtzin vapaaenergia on siis sisäenergian sekä entropian ja lämpötilan tulon erotus. Mutta mikä on sen 
käytännön merkitys?

\paragraph{Lämmönsiirto vakiolämpötilassa ja -tilavuudessa}

Olkoon meillä systeemi, josta tai johon siirtyy lämpöä. Prosessi tapahtuu niin, että systeemin lämpötila ja 
tilavuus ovat vakioita\footnote{``Isoterminen ja isokoorinen prosessi''.}. 
Koska tilavuus on vakio, edes tilavuudenmuutostyötä ei tapahdu:

\begin{align}
 T(t) & = T\\
 V(t) & = V\\
 \Delta U & = Q_{siirto} + W = Q_{siirto}
\end{align}

eli 

\begin{equation}
\label{eqn:isoTV}
 \Delta U = Q_{siirto}
\end{equation}

Systeemiin lämpönä tuotu energia menee siis systeemin sisäenergian (molekyylien kineettinen energia) kasvattamiseen.

Lämpöä voi tulla systeemiin kahdesta lähteestä; systeemin ulkopuolelta siirtymällä ja sen sisältä kun jokin muu 
energia muuttuu epäjärjestyneeseen muotoon (esimerkiksi turbulenssi muuttaa molekyylien ``koordinoitunutta'' 
liike-energiaa satunnaiseksi eli lämpöliikkeen liike-energiaksi).

\begin{equation}
\label{eqn:QQQ}
 Q = Q_{siirto} + Q_{h\ddot{a}vi\ddot{o}t}
\end{equation}

Kokonaislämpö voidaan lausua entropian muutoksen kautta entropian ja lämpötilan avulla ja
koska lämpötila on vakio:

\begin{equation}
\label{eqn:DeltaS}
 \Delta S = \frac{Q}{T} \Leftrightarrow Q = T\Delta S
\end{equation}

Sijoitetaan kaava \ref{eqn:DeltaS} kaavaan \ref{eqn:QQQ}:

\begin{equation}
 T\Delta S = Q_{siirto} + Q_{h\ddot{a}vi\ddot{o}t}
\end{equation}

Häviöt synnyttävät aina positiivisen lämpömäärän, joten

\begin{align}
 Q_{h\ddot{a}vi\ddot{o}t} & \geq 0\\
 \Rightarrow T\Delta S & \geq Q_{siirto}\\
 \label{eqn:QTSsiirto}
 \Leftrightarrow Q_{siirto} & \leq T \Delta S
\end{align}

Sijoitetaan kaava \ref{eqn:QTSsiirto} kaavaan \ref{eqn:isoTV}:

\begin{equation}
 \Delta U = Q_{siirto} \leq T\Delta S
\end{equation}

eli 

\begin{align}
 & \Delta U \leq T\Delta S\\
 & \Delta U - T\Delta S \leq 0\\
 & \Delta(U - TS) \leq 0\\
 & \Delta F \leq 0
\end{align}

Siis kun

\begin{align}
 T(t) = T\\
 V(t) = V\\
 W = 0
\end{align}

On Helmholtzin vapaaenergian muutoksen oltava

\begin{equation}
 \Delta F \leq 0
\end{equation}

\paragraph{Fysikaalinen ja kemiallinen merkitys}

Kun systeemi on vakiolämpötilassa ja -tilavuudessa eikä sisäisiä häviöitä tapahdu, on 

\begin{equation}
 \Delta F = 0
\end{equation}

Ja systeemi on termodynaamisessa tasapainossa.

Mikäli

\begin{equation}
 \Delta F < 0
\end{equation}

tarkoittaa se sitä, että systeemissä muut energian muodot muuttuvat häviöiden kautta lämmöksi 
kunnes systeemi on termodynaamisessa tasapainossa.

\subsubsection{Gibbsin vapaaentalpia G}

Kuten entalpia ja Helmholtzin vapaaenergia, \textbf{Gibbsin vapaaentalpia} $G$ on apusuure, 
ja se on määritelty näin:

\begin{equation}
 G = H - TS
\end{equation}

Gibbsin vapaaentalpian määritelmä näyttää hyvin samankaltaiselta kuin Helmholtzin vapaaenergian määritelmä. 
Miksiköhän?

\paragraph{Lämmönsiirto vakiolämpötilassa ja -paineessa}

Huomaa, että seuraava päättely etenee hyvin samalla tavalla kuin Helmholtzin vapaaenergian tapauksessa.

Siirretään lämpöä systeemistä tai systeemiin vakiolämpötilassa ja -paineessa:

\begin{align}
 T(t) & = T\\
 p(t) & = p\\
 \Delta U & = Q_{siirto} + W
\end{align}

Nyt tilavuudenmuutostyötä voi esiintyä ja analyysi voisi mennä monimutkaiseksi, muttei mene jos muistamme 
kappaleessa \nameref{sssection:entalpia} saamaamme tulosta, eli että vakiopaineessa:

\begin{equation}
 \Delta H = Q_{siirto}
\end{equation}

ja kuten kappaleessa \nameref{sssection:helmholtz} päättelimme

\begin{equation}
 Q_{siirto} \leq T \Delta S
\end{equation}

Kun yhdistämme nämä kaavat, saamme

\begin{equation}
 \Delta H = Q_{siirto} \leq T\Delta S
\end{equation}

Ja edelleen: 

\begin{align}
 & \Delta H \leq T\Delta S\\
 & \Delta H - T\Delta S \leq 0\\
 & \Delta(H - TS) \leq 0\\
 & \Delta G \leq 0
\end{align}

Siis kun

\begin{align}
 T(t) = T\\
 p(t) = p\\
 W = 0
\end{align}

On Gibbsin vapaaentalpian muutoksen oltava

\begin{equation}
 \Delta G \leq 0
\end{equation}

\paragraph{Fysikaalinen ja kemiallinen merkitys}

Kun systeemi on vakiolämpötilassa ja -paineessa eikä sisäisiä häviöitä tapahdu, on 

\begin{equation}
 \Delta G = 0
\end{equation}

Ja systeemi on termodynaamisessa tasapainossa.

Mikäli

\begin{equation}
 \Delta G < 0
\end{equation}

tarkoittaa se sitä, että systeemissä muut energian muodot muuttuvat häviöiden kautta lämmöksi 
kunnes systeemi on termodynaamisessa tasapainossa.

\chapter{Siirtymäenergiat}

Systeemillä voi siis olla monenlaista energiaa. Myös systeemin ympäristöllä voi olla näitä 
energioita.

Sillä hetkellä kun energia siirtyy systeemin rajan yli voi se kuitenkin termodynamiikassa 
olla vain kahta tyyppiä: \textbf{lämpöä $Q$} tai \textbf{työtä $W$}. Vastaavasti systeemillä 
tai ympäristöllä sinänsä ei voi olla näitä energioita vaan varastoituessaan systeemiin tai 
ympäristöön ne muuttuvat aina johonkin muuhun muotoon.

\section{Merkkisopimus}

Systeemiin siirtyvä työ ja lämpö ovat aina positiivisia, systeemistä siirtyvät negatiivisia.

Päinvastoin voidaan todeta että mikäli työtä tai lämpöä ei alunperin tiedetä, ratkaistun 
työn tai lämmön etumerkki kertoo, siirtyikö se systeemistä vai systeemiin.

\section{Lämpö Q}

Lämpö on energiaa, joka siirtyy systeemin rajojen yli siksi, että systeemin ja ympäristön 
välillä on lämpötilaero. 

Siirtyminen systeemin rajojen yli voi tapahtua millä tahansa \textbf{lämmönsiirtotavalla}:

\begin{itemize}
 \item johtumalla
 \item kulkeutumalla\footnote{``Konvektiolla''.}
 \item säteilemällä
 \item tai jollain näiden 
yhdistelmällä\footnote{Erityisen merkittävä yhdistelmä on johtumisesta ja kulkeutumisesta 
koostuva ``\textbf{konvektiivinen lämmönsiirto}''.}.
\end{itemize}

Lämpö siirtyy aina korkeammasta lämpötilasta matalampaan. Kuten myöhemmin selviää, tämä 
on seurausta termodynamiikan II pääsäännöstä. Aina kun lämpöä siirtyy, joko systeemin tai 
ympäristön entropia ja sen myötä maailmankaikkeuden epäjärjestys kasvaa.
Tähän liittyen lämpö on ``epäjärjestynyttä energiaa'', jolla voidaan tehdä vähemmän 
erilaisia asioita kuin työllä.

Lämmönsiirto muuttaa ensisijaisesti systeemin tai ympäristön sisäenergiaa. (Mieti, 
miten tämä liittyy siihen, että sisäenergia on ``lämpöliikkeen liike-energiaa''.)

\section{Työ W}

Työ on lämpöä monimuotoisempi ja vaikeammin määriteltävissä oleva siirtymäenergia. 

Työ voi olla esimerkiksi systeemin tilavuudenmuutostyötä, turbiinin akselityötä, 
sähkövirran energiaa jne.

Työ on ``järjestynyttä energiaa'', joka voidaan muuttaa vaihtelevilla hyötysuhteilla 
moniksi muiksi energian muodoiksi. Kun pelkästään työtä siirtyy ei systeemin tai ympäristön 
entropia muutu.

Työ voidaan muuttaa myös sisäenergian kautta kokonaan lämmöksi. Lämpöä taas ei voida 
(sisäenergiankaan kautta) muuttaa kokonaan työksi, koska se rikkoisi termodynamiikan 
II pääsääntöä.

% \chapter{Suureet} %---------------------------------------------------------------------
% % Tarvittavat prujuista ja kirjoista
% 
% \section{Aika ja avaruus}
% 
% \subsection{Aika}
% 
% \subsection{Pituus, pinta-ala ja tilavuus}
% 
% \section{Aine ja energia}
% 
% \subsection{Ainemäärä}
% 
% \subsection{Massa}
% 
% \subsection{Tiheys}
% 
% \subsection{Voima}
% 
% \subsection{Energia}
% 
% \section{Ekstensiivi- ja intensiivisuureet}
% 
% \section{Termodynaamiset suureet}
% 
% \subsection{Viskositeetti}
% 
% \subsection{Lämmönjohtavuus}
% 
% \subsection{Terminen diffusiviteetti}
% 
% \section{Tilasuureet l. termodynaamiset potentiaalit} 
% 
% \subsection{Paine}
% 
% \subsection{Tiheys ja ominaistilavuus}
% 
% \newpage % A temporary hack
% 

% 
% \subsection{Lämpötila}
% 
% \subsection{Ominaislämpökapasiteetit}
% 
% Lämpötila, sisäenergia ja sisäenergian sisältävä entalpia mittaavat siis enemmän tai vähemmän samaa asiaa. Ne voidaan 
% näinollen luultavasti kytkeä toisiinsa jollakin yksinkertaisella tavalla. Tästä kytkennästä on myös se olennainen hyöty, 
% että laskennassa hyödylliset mutta vaikeasti mitattavat sisäenergia ja entalpia saadaan kytkettyä harvemmin kiinnostavaan 
% mutta helposti mitattavaan lämpötilaan.

% Termodynamiikassa kytkentään käytetään \textbf{ominaislämpökapasiteettia vakiotilavuudessa $c_v$} ja 
% \textbf{ominaislämpökapasiteettia  vakiopaineessa $c_p$}. Määritellään nämä ominaislämpökapasiteetit. Sisäenergian ja 
% entalpian differentiaaliset muutokset voi kytkeä lämpötilan differentiaaliseen muutokseen osittaisderivaattojen avulla:
% 
% \begin{align}
%  du = \frac{\partial u}{\partial T}\bigg|_vdT\\
%  dh = \frac{\partial h}{\partial T}\bigg|_pdT
% \end{align}
% 
% Nämä derivoinnin tuloksena syntyvät funktiot on nimetty ominaislämpökapasiteeteiksi (lyhyesti ``ominaislämmöiksi'') 
% vakiotilavuudessa ja vakiopaineessa:
% 
% \begin{align}
%  \frac{\partial u}{\partial T}\bigg|_v = c_v(T)\\
%  \frac{\partial h}{\partial T}\bigg|_p = c_p(T)\\
%  du = c_vdT\\
%  dh = c_pdT
% \end{align}
% 
% Yleisesti ottaen ominaislämmöt eivät ole yksinkertaisia tai helposti teoreettisesti johdettavissa olevia lämpötilan 
% funktioita. 
% 
% Käytännön laskennassa ominaislämmöille käytetään taulukoituja, käyräksi piirrettyjä tai kokeellisen 
% polynomiapproksimaation muodossa olevia funktioita. Jos toimitaan kapealla lämpötila-alueella ja tulokset on 
% tärkeämpää saada nopeasti kuin tarkkoina voidaan ominaislämpö olettaa vakioksi lämpötila-alueella. Tällaisia tilanteita 
% ovat esim. alustavat tunnustelulaskelmat, pika-analyysit ja tentit.
% 
% % \frac{c_p}{c_v} = \gamma, c_p - c_v = R
% % Kokoonpuristumattomattomalle c_p = c_v
% 
% \subsection{Entropia}
% 
% \subsection{Helmhotzin vapaaenergia}
% 
% \subsection{Gibbsin vapaaenergia}
% 
% \subsection{Vapaat tilasuureet l. vapaat muuttujat l. vapausasteet}
% 
% \paragraph{Tilanyhtälö}
% 
% \section{Suureluettelo}
% % Kunkin suureen nimet, symbolit, yksiköt
% 
% \chapter{Säilymislait} % ---------------------------------------------------------------
% 
% \section{Taselaskenta}
% % reynoldsin siirtoteoreema
% 
% \section{Massan säilyminen}
% 
% \subsection{Suljettu systeemi}
% 
% \subsection{Avoin systeemi}
% 
% \section{Liikemäärän säilyminen}
% 
% \subsection{Suljettu systeemi}
% 
% \subsection{Avoin systeemi}
% 
% \section{Kulmaliikemäärän säilyminen}
% 
% \subsection{Suljettu systeemi}
% 
% \subsection{Avoin systeemi}
% 
% \section{Energian säilyminen}
% 
% \subsection{Suljettu systeemi}
% 
% \subsection{Avoin systeemi}
% 
% \chapter{Termodynamiikan pääsäännöt} % -------------------------------------------------
% 
% \section{Nollas pääsääntö: lämpötilan määritelmä}
% 
% \section{Ensimmäinen pääsääntö: energian säilyminen}
% 
% \subsection{Systeemin energia ja siirtyvä energia}
% 
% \subsection{Suljettu systeemi}
% 
% \subsection{Avoin systeemi}
% 
% \section{Toinen pääsääntö: epäjärjestyksen lisääntyminen}
% 
% \subsection{Energian laatu}
% 
% \subsection{Suljettu systeemi}
% 
% \subsection{Avoin systeemi}
% 
% \section{Kolmas pääsääntö: absoluuttinen nollapiste}
% 
% \subsection{Suljettu systeemi}
% 
% \subsection{Avoin systeemi}
% 
% \chapter{Muut lait} % ------------------------------------------------------------------
% % esim. fourierin "laki"
% 
% \part{Termodynamiikkaa}% ________________________________________________________________
% 
% \chapter{Termodynaamiset prosessit}
% % Huomaa tilasuureiden ja muiden ero (Q ja W eivät potentiaaleja)
% % Työn ja lämmöntuonti vakiopaineessa, -tilavuudessa, -lämpötilassa, entropiassa...
% % Polytrooppiprosessi, mielivaltaiset prosessit
% 
% \chapter{Aineen olomuodot ja faasimuutokset}
% % Kiinteä, neste, kaasu, ylikrittinen, plasma(...:P)
% % Faasidiagrammit
% 
% \part{Virtausoppia} % ___________________________________________________________________
% 
% \chapter{Virtausaineen käsite}
% 
% \part{Lämmönsiirtoa} % ___________________________________________________________________
% 
% \part{Matemaattisia menetelmiä} % ________________________________________________________
% 
% \chapter{Differentiaalilaskenta} % -------------------------------------------------------
% % Osittais- ja kokonaisdifferentiaalit
% 
% \section{Derivointi}
% % Osittais- ja kokonaisderivaatat
% 
% \section{Integrointi}
% 
% \section{Differentiaaliyhtälöt}
% 
% \chapter{Vektorianalyysi} % --------------------------------------------------------------
% 
% \section{Skalaari- ja vektorikentät}
% 
% \section{Aikaderivoidut suureet}
% % ``Virrat``, totaalinen derivaatta 
% 
% \chapter{Dimensioanalyysi} % -------------------------------------------------------------
% 
% \section{Dimensiottomat suureet}
% 
% \section{Lämpöteknisiä dimensiottomia suureita}
% % Suureet-osioon?
% 
% \part{Laskuharjoitusten ratkaisuja} % ____________________________________________________

\end{document}

