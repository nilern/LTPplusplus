% Copyright 2013 Pauli Jaakkola
% 
% This program is free software: you can redistribute it and/or modify
% it under the terms of the GNU General Public License as published by
% the Free Software Foundation, either version 3 of the License, or
% (at your option) any later version.
% 
% This program is distributed in the hope that it will be useful,
% but WITHOUT ANY WARRANTY; without even the implied warranty of
% MERCHANTABILITY or FITNESS FOR A PARTICULAR PURPOSE.  See the
% GNU General Public License for more details.
% 
% You should have received a copy of the GNU General Public License
% along with this program.  If not, see <http://www.gnu.org/licenses/>.

\documentclass[12pt,a4paper,finnish]{book}

\usepackage[utf8]{inputenc}                 % Tekstiasetuksia, sisältää ääkköset
\usepackage[T1]{fontenc}                    % Tekstiasetuksia, T1-koodatut fontit
\usepackage{ae,aecompl}                     % Paremman näköiset fontit
\usepackage[finnish]{babel}                 % Suomenkielinen tavutus ja otsikot
% \usepackage{a4wide}
\usepackage[intlimits]{amsmath}             % Lisää kaavavoimaa!
\usepackage{fixltx2e}                       % \textsubscript
\usepackage{enumerate}
\usepackage{framed}
\usepackage{hyperref}
\makeatother
\hypersetup{
  colorlinks=true,
  linkcolor=blue,
}

\title{LTP++\\Lämpötekniikan perusteita}
\date{\today}
\author{Pauli Jaakkola}

\begin{document}

\maketitle
% GPL, LaTeX-maininnat tähän
\newpage
\pagenumbering{Roman}
\setcounter{tocdepth}{3}
\tableofcontents
\newpage
\pagenumbering{arabic}
\setcounter{secnumdepth}{3}

\chapter*{Johdanto} %------------------------------------------------------------------------------------------
\addcontentsline{toc}{chapter}{Johdanto}
\renewcommand{\thesection}{\arabic{section}}

\section{Lämpötieteet ja lämpötekniikka} %..............................................

On helppo ajatella suoraviivaisesti, että \textbf{tieteet} lähtevät liikkeelle kiinnostuksesta johonkin 
luonnossa tapahtuvaan \textbf{ilmiöön}. 
Sitten tehdään \textbf{perustutkimusta} -- laaditaan \textbf{teorioita} ja \textbf{testataan niitä kokeellisesti}. 
Lopulta kun teoria on riittävän \textbf{yleispätevä}, joku käyttää sitä ja luovuuttaan teknisen tai muun \textbf{käytännön 
sovelluksen} luomiseen. Ollaan edetty \textbf{soveltavaan tutkimukseen}.

\textbf{Lämpötieteet}, kuten tässä kirjassa käsiteltävät

\begin{itemize}
 \item Termodynamiikka
 \item Virtausoppi
 \item Lämmönsiirto
\end{itemize}

ovat kuitenkin suurelta osin ns. \textbf{teknisiä tieteitä} eli insinöörien työkaluja. Ne ovat 
syntyneet pikemminkin tutkittaessa, miten \textbf{lämpöteknisiä} sovelluksia, kuten 

\begin{itemize}
 \item Lämpövoimakoneita (voimalaitokset)
 \item Lämpöpumppuja (jäähdytys ja lämmitys)
 \item Virtauskoneita (pumppuja, puhaltimia, kompressoreja ja turbiineja)
 \item Lämmönsiirtimiä (monien prosessien osana)
\end{itemize}

voitaisiin parantaa. Vaikka lämpötieteet ovat sittemmin monin osin kehittyneet maailmaa syleilevän yleispäteviksi, 
niiden keskeisin tai ainakin hyvödyllisin sovellusalue on edelleen juuri lämpötekniikka.

\section{Suureet, luonnonlait ja kaavat; ymmärrys ja tulokset} %...............................................

Mielikuvamme luonnontieteistä ja teknisistä tieteistä on usein sellainen, että ne koostuvat pääosin \textbf{kaavoista}. 
Itse asiassa kaavat ovat kuitenkin vain korkealle abstraktiotasolle jalostettuja yhteenvetoja siitä ymmärryksestä, 
joka on saavutettu \textbf{teoreettisten mallien} ja \textbf{kokeellisen tutkimuksen} vuorovaikutuksessa.
Kaavoja suositaan luonnontieteissä myös sen takia, että ne ovat \textbf{kvantitatiivisia} tieteitä, jotka 
pyrkivät tarkkuuteen ja yksiselitteisyyteen eli \textbf{eksaktiuteen}\footnote{Luonnontieteilijät pitävät 
joskus -- tai useinkin -- itseään jotenkin ihmistieteilijöitä parempina tällä perusteella. Tämä näkyy 
teekkarien ja humanistien välisessä vastakkainasettelussa mutta myös siinä, että englannin kielen tiedettä 
tarkoittava sana \textit{science} voi yksinään tarkoittaa nimenomaan luonnontiedettä, jopa erotuksena 
ihmistieteistä. Todellisuudessa luonnontieteiden kvantitatiivisuus ja eksaktius johtuu kuitenkin siitä, että 
tarkasteltavat ilmiöt ovat oikeastaan hyvin yksinkertaisia verrattuna vaikkapa ihmisen käyttäytymiseen.}

Useimmiten kaavat kertovat joidenkin \textbf{suureiden} välisen yhteyden matemaattisessa muodossa. Yhteys sinänsä 
saattaa olla syvällinen ajatus, jopa \textbf{luonnonlaki} -- monet kaavat kuvaavat esimerkiksi energian säilymistä:

\begin{align}
 \frac{\dot{Q} + \dot{W}}{\dot{m}} &= \Delta\left(h + \frac{1}{2}\vec{V}^2 + gz\right)\\
 \rho c_v\frac{dT}{dt} &= k\nabla^2T + \dot{Q}
\end{align}

Kuitenkin ehkä suurempi osa luonnontieteen oivalluksista on käsitteellistetty itse suureisiin. Monet kaavatkin 
ovat itse asiassa vain suureiden määritelmiä:

\begin{align}
 H &= U + pV\\
 G &= H - T\Delta S 
\end{align}

Niinpä jos tämän kirjan ``punainen lanka'' ovat luonnonlait, niin ehkä suureet ovat toinen yhtälailla tärkeä
``vihreä lanka''\footnote{Myös erään puolueen lehti. Tämän alaviitteen tarkoitus on kuitenkin huomauttaa, 
ettei tässä ole kyse tuotesijoittelusta tai poliittisesta propagandasta.}. Kaavat ovat toki tärkeitä, sillä 
niillä saadaan \textbf{tuloksia}, mutta vasta niiden taustalla vaikuttavien suureiden ja luonnonlakien 
ymmärtäminen mahdollistaa \textbf{luovuuden}. Tai edes oikeiden kaavojen käytön oikeassa tilanteessa ja siten 
tulosten \textbf{oikeellisuuden}.

\section{Miksi nämä tieteet?}

Lämpötieteellisten ilmiöiden ja -teknisten laitteiden analyysi on käytännössä useimmiten monitieteellistä. 
Mietitäänpä vaikkapa \textbf{lämmönsiirrintä}, joka siirtää \textbf{lämpötehoa $\dot{Q}$} vesivirtauksesta $a$ 
vesivirtaukseen $b$. Seuraavassa esiintyviä kaavoja ei tietenkään tarvitse tässä vaiheessa vielä ymmärtää.

Ensinnäkin meitä tietenkin kiinnostaa lämmönsiirron suunta. \textbf{Termodynamiikan toisen pääsäännön} mukaan 
lämpö siirtyy spontaanisti\footnote{``itsestään, luonnostaan''} korkeammasta lämpötilasta matalampaan. Tämä on 
kokeellinen havainto, mutta klassinen termodynamiikka selittää sen niin, että \textbf{entropian} täytyy kasvaa. 
Tilastollinen termodynamiikka selittää, miksi näin on. Matemaattisesti:

\begin{align}
 &T_a > T_b\\
 &\Rightarrow \dot{Q}_a < 0\\
 &\Rightarrow \dot{Q}_b > 0
\end{align}

Kun lämmönsiirron suunta on nyt selvillä, meitä tietenkin kiinnostaa kummankin vesivirran lämpötilan muutos. 
\textbf{Termodynamiikan ensimmäisen pääsäännön} mukaan energia säilyy eli virtausten entalpiat muuttuvat 
lämmön verran. Oletetaan että kaikki lämpö siirtyy $a$:sta $b$:hen (eikä esim. lämmönsiirtimen rakenteisiin):

\begin{align}
 &\dot{Q}_b = \dot{Q}_a\\
 &\Delta \dot{H_a} = \dot{Q_a}\\
 &\Delta \dot{H_b} = \dot{Q_b}
\end{align}

Käyttämällä entalpiavirran ja ominaisentalpian ($\dot{H} = \dot{m}h$) sekä ominaisentalpian ja lämpötilan 
($\Delta h = c_p \Delta T$) välisiä yhteyksiä saadaan virtausten lämpötilojen muutoksen lämmönsiirtimessä:

\begin{align}
 &\Delta T_a = \frac{\dot{Q_a}}{c_p\dot{m}_a}\\
 &\Delta T_b = \frac{\dot{Q_b}}{c_p\dot{m}_b}
\end{align}

Tässä vaiheessa ongelmaksi tulee tietenkin sen määrittäminen, miten suuri siirtyvä lämpöteho on. Tähän 
tarvitaan \textbf{lämmönsiirtoa}. Lämpö siirtyy virtauksissa \textbf{konvektiolla} ja \textbf{johtumalla} 
ja putkien läpi johtumalla. Lämmön johtumisen teoria on melko yksinkertainen ja tarkka. Konvektiivisesta 
lämmönsiirrosta saadaan kohtuullinen arvio dimensiottomien lukujen avulla ilmaistuilla kokeellisilla 
korrelaatioilla.

Konvektiivisen lämmönsiirron tarkempi määrittäminen vaatisi \textbf{virtausopin} tuntemusta. Sitä 
tarvitaan myös sen määrittämiseen, miten suuren \textbf{mekaanisen tehon} $\dot{W}$ virtauksen 
pumppaaminen lämmönsiirtimen läpi vaatisi.

Nämä kolme ovat siis keskeisimmät lämpötekniikassa tarvittavat tieteet. Tietenkään poikkitieteellisyys 
ei välttämättä lopu vielä tähän. Esimerkiksi lämpötilaeron lämmönsiirtimen rakentaisiin kohdistamien 
lämpörasitusten laskemiseen tarvittaisiin teknillistä mekaniikkaa. Usein voimalaitoksissa lämpö saadaan 
joko polttoprosessista tai ydinreaktiosta, joiden analysoimiseen tarvitaan fysikaalista kemiaa tai 
ydinfysiikkaa jne.

\section{Merkinnöistä} %................................................................

Lämpötieteiden kirjallinen perinne on vanha ja julkaisujen määrä valtava. Tämän seikan valossa on täysin 
ymmärrettävää, että \textbf{käytetyt merkinnätkin vaihtelevat melkoisesti}:

\begin{itemize}
 \item Esimerkiksi $q$:lla voidaan merkitä ominaislämpöä ($J/kg$), lämpövirran tiheyttä ($W/m^2$) tai jopa 
  tilavuusvirtaa ($m^3/s$). 
 \item Samaten $u$, $v$ ja $h$ voivat merkitä sisäenergiaa, ominaistilavuutta ja entalpiaa -- tai 
  sitten nopeusvektorin x- ja y-komponentteja sekä lämmönsiirtokerrointa.
\end{itemize}
  
Tämän tilanteen syntyä on edesauttanut myös se, että jo lämpötieteiden sisällä -- saati sitten fysiikassa 
yleensä -- on käytössä niin monta suuretta, että latinalaiset tai kreikkalaisetkaan aakkoset eivät tahdo riittää.

Kun teoria on hyvin hallussa, suureet menevät harvoin sekaisin sekalaisista merkinnöistä huolimatta. 
Laskentatilanteesta, kaavojen muodosta ja yksiköistä näkee, mistä suureista on kyse. Mutta tätä kirjaa 
lukevat ainakin toivottavasti ne, joilla teoria ei ole vielä juuri ollenkaan hallussa.

Niinpä \textbf{olen pyrkinyt yksiselitteiseen merkintätapaan}, jossa eri suureita ei merkitä samalla merkinnällä. 
Mikäli eri kirjaimen käyttäminen olisi täysin yleisen käytännön vastaista käytän vektorimerkkejä 
tai aikaderivaattoja  suureiden erottelemiseksi:
 
\begin{itemize}
 \item ominaistilavuus $v$, vauhti eli nopeusvektorin pituus $|\vec{v}|$
 \item tilavuus $V$, tilavuusvirta $\dot{V}$
\end{itemize}

\renewcommand{\thepart}{\arabic{part}}
\setcounter{part}{-1}
\part{Suureita} % _____________________________________________________________________________________________
\setcounter{chapter}{0}
\renewcommand{\thesection}{\arabic{chapter}.\arabic{section}}

Ennen kuin alamme varsinaisesti käsitellä termodynamiikkaa tai muitakaan lämpötieteitä on syytä palauttaa mieleen 
muutama perussuure yksiköineen ja määritelmineen. Luultavasti suureet ovat ennestään tuttuja etkä halua käyttää 
niihin juurikaan aikaa mutta perusteelliseen ymmärrykseen on hyvä pyrkiä -- pidemmällä tähtäimellä sitä kautta 
pääsee vähemmällä. Ja mistäpä muualta perusteellinen ymmärrys lähtisi kuin perusteista, perusasoista.

% \section{Puhtaista luvuista vektoreihin}
% 
% \begin{itemize}
%  \item \textbf{Puhtailla luvuilla} ei ole yksikköä. Meillä voi olla esimerkiksi 2 pakastinta, 3 turbiinia ja 1 lämmönvaihdin.
%  \item \textbf{Skalaarisuureilla} on yksikkö eli suuruus: pituudella metri, massalla kilogramma jne.
%  \item \textbf{Vektorisuureilla} on suuruus ja suunta, esimerkiksi voiman yksikkö on newton mutta on välttämätöntä 
%  tietää myös mihin suuntaan voima vetää tai työntää.
% \end{itemize}

% kunkin suureen kohdalla, onko se mikä näistä!

\chapter{Perussuureita} %--------------------------------------------------------------------------------------------

\section{Avaruus ja aika} % ................................................................

Lämpötieteissä ulottuvuuksia käsitellään klassisen fysiikan tapaan eli avaruusulottuvuuksia on kolme ja ne ovat 
toisistaan sekä ajan yhdestä ulottuvuudesta erillisiä. Syitä tähän on pohjimmiltaan kaksi:

\begin{enumerate}
 \item Lämpötieteet syntyivät ennen suhteellisuusteoriaa ja muuta modernia fysiikkaa.
 \item Käytännön sovelluksissa on harvinaista joutua käsittelemään tilanteita joissa tarvittaisiin 
 suhteellisuusteorian aika-avaruutta\footnote{Miksi tämä on englanniksi ``spacetime'' ja suomeksi ``aika-avaruus''?}
 tai ylimääräisiä avaruusulottuvuuksia.
\end{enumerate}

\subsection{Avaruus}

\subsubsection{Pituus L}

\textbf{Pituuden} (usein \textbf{L}) yksikkönä käytetään SI-perusyksikkö \textbf{metriä}:

\begin{equation}
 [L] = m
\end{equation}

(``Metri on sellaisen matkan pituus, jonka valo kulkee tyhjiössä aikavälissä 1/299 792 458 sekuntia 
(17. CGPM, 1983).'')

\subsubsection{Pinta-ala A}

\textbf{Pinta-alan} (usein \textbf{A}) yksikkönä käytetään \textbf{neliömetriä}:

\begin{equation}
 [A] = m^2
\end{equation}

(Suorakaiteen pinta-alaa voi kuvata kertomalla sen sivujen pituudet toisillaan. Koska minkä muotoisen tasokuvion 
tahansa voi ajatella muodostuvan esimerkiksi mielivaltaisen pienistä neliöistä, on neliömetri pätevä mittaamaan 
mielivaltaisen muotoisia pinta-aloja).

\subsubsection{Tilavuus V}

\textbf{Tilavuuden} (usein \textbf{V}) yksikkönä käytetään \textbf{kuutiometriä}:

\begin{equation}
 [V] = m^3
\end{equation}

(Suorakulmaisen särmiön tilavuutta voi kuvata kertomalla sen sivujen pituudet toisillaan. Koska minkä muotoisen avaruuskappaleen 
tahansa voi ajatella muodostuvan esimerkiksi mielivaltaisen pienistä kuutioista, on kuutiometri pätevä mittaamaan 
mielivaltaisen muotoisia tilavuuksia).

\subsubsection{Yksiulotteinen sijainti s}

Ensimmäisenä on syytä mainita, että jos tilannetta voidaan kuvata yksiulotteisena\footnote{ajattele vaikkapa 
raiteillaan pysyvää junaa}, käytetään joskus ainoana avaruuskoordinaattina \textbf{sijaintia s}.

\subsubsection{Kolmiulotteinen sijainti \={r}}

Avaruuden ulottuvuuksia on siis kolme ja ne muodostavan kolmiulotteisen avaruuden. 

Mikä tahansa piste tässä avaruudessa voidaan määrittää \textbf{kolmen koordinaatin} avulla. 
Ensin koordinaatit täytyy kalibroida määrittämällä niille nollakohdat sekä yksiköt. ``Kartesiolainen'' eli 
suorakulmainen $(x, y, z)$-koordinaatisto on yleisin, mutta lämpötieteissä eivät ole erityisen harvinaisia tilanteet 
joissa esimerkiksi sylinterikoordinaatisto $(z, r, \theta)$ tai pallokoordinaatisto $(r, \theta, \phi)$ on kätevämpi.

\textbf{Origo} sijaitsee koordinaattien nollakohtien leikkauspisteessä $(0, 0, 0)$. Minkä tahansa pisteen sijainti 
voidaan ilmoittaa \textbf{paikkavektorilla $\vec{r}$} origosta kyseiseen pisteeseen. Kartesiolaisessa koordinaatistossa

\begin{equation}
 \vec{r} = \begin{bmatrix}
            x\\ y\\ z
           \end{bmatrix}
\end{equation}

\subsection{Aika t}

\textbf{Ajan t} yksikkönä käytetään SI-perusyksikkö \textbf{sekuntia}:

\begin{equation}
 [t] = s
\end{equation}

(``Sekunti on 9 192 631 770 kertaa sellaisen säteilyn jaksonaika, joka vastaa cesium 133 -atomin siirtymää 
perustilan ylihienorakenteen kahden energiatason välillä (13. CGPM, 1967).'')

% Kenttäsuureet

\section{Aineen määrä} % ..............................................................

\subsection{Ainemäärä n}

\textbf{Ainemäärä n} kertoo kuinka monta kappaletta jotain hiukkasta (yleensä molekyylia) on. Se on siis itse 
asiassa puhdas luku, mutta koska yleensä käsitellään niin suuria molekyylimääriä, on sille määritetty 
SI-perusyksikkö \textbf{mooli}:

\begin{equation}
 [n] = mol
\end{equation}

``Mooli on sellaisen systeemin ainemäärä, joka sisältää yhtä monta keskenään samanlaista perusosasta kuin 0,012 
kilogrammassa hiili 12:ta on atomeja. Perusosaset voivat olla atomeja, molekyylejä, ioneja, elektroneja,
muita hiukkasia tai sellaisten hiukkasten määriteltyjä ryhmiä. (14. CGPM, 1971)''

0,012 kilogrammassa hiili-12:ta eli yhdessä moolissa olevien hiukkasten lukumäärä on \textbf{Avogadron luku $N_A$}:

\begin{equation}
 [N_A] \approx (6,02214129 \pm 0,00000027)\cdot10^{23}
\end{equation}

Ainemäärä on hyödyllinen yleensä kemiassa (koska reaktioissa väliä on molekyylien määrällä) ja kaasuja käsiteltäessä 
(koska mm. tilavuudet ja paineet riippuvat molekyylien määristä).

\subsection{Massa m}

Mekaniikassa meitä kiinnostaa kuitenkin yleensä pikemminkin se, miten ``painava'' tai ``hidas'' käsiteltävä systeemi 
on. Tätä mitataan \textbf{massalla m}. Klassisessa fysiikassa esiintyy itse asiassa kahdenlaista massaa:

\begin{itemize}
 \item \textbf{Hidas massa} on Newtonin II laissa esiintyvä massa. Se mittaa siis sitä miten suuri voima tarvitaan 
 kappaleen kiihdyttämiseen.
 \item \textbf{Painava massa} taas on Newtonin gravitaatiolaissa esiintyvä massa. Se mittaa siis sitä miten suuren 
 voiman gravitaatiokenttä aiheuttaa kappaleeseen\footnote{Vrt. varaus sähkömagneettisissa kentissä.}.
\end{itemize}

Hidas ja painava massa ovat kuitenkin saman suuruiset, mikä ei ollut klassisen fysiikan teorioiden perusteella  
mitenkään itsestään selvää. Kuitenkin jo Galileo Galilei huomasi kokeellisesti, että kaikkien kappaleiden 
kappaleen \textbf{putoamiskiihtyvyys g} on sama. Näin voi olla vain, mikäli hidas ja painava massa ovat yhtä suuret.

% Todistus

Suppea suhteellisuusteoria pätee vain vakionopeudella liikkuville koordinaatistoille (arkisemmin ``tarkkailijoille''). 
Se sai alkunsa sähkömagneettisten aaltojen teoriassa tehdystä havainnosta että valon nopeus tyhjiössä on 
koordinaatiston nopeudesta riippumaton vakio.

Yleinen suhteellisuusteoria pätee myös kiihtyvässä liikkeessä oleville koordinaatistoille. Sen perustava oivallus 
oli nimenomaan se, että hidas ja painava massa tuskin ovat sattumalta täsmälleen yhtä suuret. Putoamiskiihtyvyys 
on kiihtyvyys, joka aiheutuu aika-avaruuden kaareutumisesta massan ympärillä.

Massan SI-perusyksikkö on \textbf{kilogramma kg}:

\begin{equation}
 [m] = kg
\end{equation}

``Kilogramma on yhtä suuri kuin kansainvälisen kilogramman prototyypin massa (1. ja 3. CGPM, 1889 ja 1901).''

Kilogramma on ainoa SI-perusyksikkö, joka vielä perustuu tällaiseen prototyyppiin. Tämä on ongelmallista ensinnäkin 
siksi, että prototyyppi ei ole toistettavissa ja toisekseen siksi että - kauhistus sentään - prototyypin massa 
ei mittausten mukaan ole vakio. Alun perin kilogramma piti määritellä ``1 litra vettä on massaltaan kilogramman 
4 $^{\circ}$C:n lämpötilassa''. Vesipohjaiseen määritelmään siirtymistä on myöhemminkin ehdotettu, joskin niin että 
määritelmä vastaisi nykyistä kilogramman määritelmää paremmalla tarkkuudella.

\subsection{Moolimassa M}

Systeemin massa ja ainemäärä riippuvat toisistaan \textbf{moolimassan M} kautta:

\begin{equation}
 M = \frac{m}{n}
\end{equation}

Moolimassan SI-yksiköksi tulee

\begin{equation}
 [M] = \left[\frac{m}{n}\right] = \frac{[m]}{[n]} = \frac{kg}{mol}
\end{equation}

Tämä on kuitenkin niin suuri yksikkö että helpommin käsiteltäviä lukuja saadaan käyttämällä yksikkönä joko 
g/mol (yleisin) tai kg/kmol.

\chapter{Yksinkertaisia johdannaissuureita} % --------------------------------------------------------------------

\section{Aikaderivaattasuureet} % ................................................................

% Suureen muutos / ajan muutos - muutos tietyssä ajassa - aika mieliv. lyhyeksi - aikaderivointi

% Ajan ja tilan kytkentä - vaikka itsenäisiä, muuttuvat yhtä aikaa

\subsection{Nopeus \={v}}

Yksiulotteisen sijainnin muutoksen suhde ajan muutokseen on \textbf{vauhti $|\vec{v}|$}:

\begin{equation}
 v = \frac{\Delta s}{\Delta t}
\end{equation}

Jos vauhti halutaan hetkellisesti eli mielivaltaisen lyhyenä ajanhetkenä tämä lähestyy aikaderivaattaa:

\begin{equation}
 v = \frac{ds}{dt}
\end{equation}

Kun tämä siirretään kolmiulotteiseen avaruuteen paikkavektorin $\vec{r}$ derivaataksi saadaan \textbf{nopeus $\vec{v}$} 
joka on siis myös vektorisuure:

\begin{equation}
 \vec{v} = \frac{d\vec{r}}{dt}
\end{equation}

Aikaderivaattaa on usein tapana merkitä pisteellä:

\begin{equation}
 \vec{v} = \frac{d\vec{r}}{dt} = \dot{\vec{r}}
\end{equation}

Nopeuden yksiköksi tulee sama kuin vauhdinkin eli

\begin{equation}
 [v] = \left[\frac{ds}{dt}\right] = \left[\frac{s}{t}\right] = \frac{m}{s}
\end{equation}

\subsection{Kiihtyvyys \={a}}

Vauhdin muutos ajan suhteen on \textbf{kiihtyvyys $|\vec{a}|$}:

\begin{equation}
 a = \frac{\Delta v}{\Delta t}
\end{equation}

Jos vauhti halutaan hetkellisesti eli mielivaltaisen lyhyenä ajanhetkenä tämä lähestyy aikaderivaattaa:

\begin{equation}
 a = \frac{dv}{dt}
\end{equation}

Kolmiulotteisessa avaruudessa nopeusvektorin $\vec{v}$ derivaataksi saadaan \textbf{kiihtyvyysvektori $\vec{a}$}:

\begin{equation}
 \vec{a} = \frac{d\vec{v}}{dt} = \dot{\vec{v}} = \ddot{\vec{r}}
\end{equation}

Kiihtyvyyden yksikkö on

\begin{equation}
 [a] = \left[\frac{dv}{dt}\right] = \left[\frac{d^2s}{dt^2}\right] = \left[\frac{s}{t^2}\right] = \frac{m}{s^2}
\end{equation}

\subsection{Tilavuusvirta \.{V}}

Kun halutaan tietää, kuinka suuri tilavuus kulkee jonkin pinnan läpi aikayksikössä voidaan 
se määrittää vastaavalla menettelyllä kuin nopeus. Keskimääräinen \textbf{tilavuusvirta} $\dot{V}$ on

\begin{equation}
 \dot{V} = \frac{\Delta V}{\Delta T}
\end{equation}

ja hetkellinen

\begin{equation}
 \dot{V} = \frac{dV}{dt}
\end{equation}

Tilavuusvirran SI-yksikkö on

\begin{equation}
 [\dot{V}] = \left[\frac{dV}{dt}\right] = \left[\frac{V}{t}\right] = \frac{m^3}{s}
\end{equation}

Tämänkaltaisista aikaderivoiduista suureista, jotka eivät ole nopeutta, kiihtyvyyttä eivätkä mekaanista tai
lämpötehoa on tapana käyttää \textbf{virta}-nimitystä.

\subsection{Moolivirta \.{n}}

Pinnan läpi aikayksikössä menevä ainemäärä on \textbf{moolivirta} $\dot{n}$:

\begin{align}
 &\dot{n} = \frac{\Delta n}{\Delta t}\\ 
 &\dot{n} = \frac{dn}{dt}\\
 &[\dot{n}] = \left[\frac{dn}{dt}\right] = \left[\frac{n}{t}\right] = \frac{mol}{s}
\end{align}

\subsection{Massavirta \.{m}}

Pinnan läpi aikayksikössä menevä massa on \textbf{massavirta} $\dot{m}$:

\begin{align}
 &\dot{m} = \frac{\Delta m}{\Delta t}\\ 
 &\dot{m} = \frac{dm}{dt}\\
 &[\dot{m}] = \left[\frac{dm}{dt}\right] = \left[\frac{m}{t}\right] = \frac{kg}{s}
\end{align}

\section{Ekstensiivi- ja intensiivisuureet} %...............................................

\textbf{Ekstensiivisuureet} ovat suureita, joiden arvo riippuu systeemin koosta\footnote{``extent''} eli 
massasta tai ainemäärästä. Tyypillinen ekstensiivisuure on tilavuus $V$. \textbf{Intensiivisuureiden} arvot 
taas eivät riipu systeemin koosta. Tyypillisiä intensiivisuureita ovat paine $p$ ja lämpötila $T$.

\subsection{Ominaissuureet}

Intensiivisuureet ovat siinä mielessä toivottavampia, että niiden käyttö ei vaadi systeemin koon 
selvittämistä tai kiinnittämistä. Niistä saadaan jopa skalaarikenttiä (esim. $T(\vec{r}, t)$). 

Onneksi 
ekstensiivisuureet voidaan muuttaa intensiivisuureiksi jakamalla ne systeemin massalla. Näin 
syntyviä intensiivisuureita kutsutaan \textbf{ominaissuureiksi} ja merkitään vastaavaa ekstensiivisuureen 
suurta vastaavalla pienellä kirjaimella. Esimerkiksi ominaistilavuus on

\begin{align}
 &v = \frac{V}{m}\\
 &[v] = \left[\frac{V}{m}\right] = \frac{m^3}{kg}
\end{align}

ja ominaissisäenergia

\begin{align}
 &u = \frac{U}{m}\\
 &[u] = \left[\frac{U}{m}\right] = \frac{J}{kg}
\end{align}

\subsection{Molaariset ominaissuureet}

Toinen vaihtoehto ekstensiivisuureiden muuntamiseksi intensiivisiksi on niiden jakaminen systeemin 
ainemäärällä sen massan sijaan. Näin saadaan \textbf{molaarisia ominaissuureita}, joita merkitään 
alaindeksillä $m$. Esimerkiksi moolitilavuus on

\begin{align}
 &V_m = \frac{V}{n}\\
 &[V_m] = \left[\frac{V}{n}\right] = \frac{m^3}{mol}
\end{align}

ja molaarinen sisäenergia

\begin{align}
 &U_m = \frac{U}{n}\\
 &[U_m] = \left[\frac{U}{n}\right] = \frac{J}{mol}
\end{align}

\chapter{Monimutkaisempia johdannaissuureita} % ------------------------------------------------------------------

\section{Voima \={F}} % ...........................................................

Olemme tottuneet ajattelemaan voimaa jonkinlaisena perussuureena, mutta itse asiassa se on vain hyvin kätevä 
johdannaissuure, joka on määritelty Newtonin II lain \footnote{Newtonin I laki on II lain erikoistapaus.} perusteella:

\begin{equation}
 \vec{F} = \frac{d\vec{p}}{dt} = \frac{d(m\vec{v})}{dt}
\end{equation}

Niinpä sen yksiköksi tulee:

\begin{equation}
 [F] = [|\vec{F}|] = \left[\frac{d(m|\vec{v}|)}{dt}\right] = \left[\frac{mv}{t}\right] 
  = \left[\frac{mL}{t^2}\right] = \frac{kgm}{s^2}
\end{equation}

Tämä on edelleen nimetty\footnote{ilmeisistä syistä} Newtoniksi:

\begin{equation}
 [F] = \frac{kgm}{s^2} = N
\end{equation}

Voimia ei liene todellisuudessa olemassakaan. Ne ovat vain yksi ihmiskunnan historian hyödyllisimmistä abstraktioista. 
Tämä voiman eksakti muoto kuvaa vain sitä mitä arkikielen voima-sanakin: ``voimaa'' tarvitaan sitä enemmän mitä enemmän 
ja mitä nopeammin materiaa joudutaan kiihdyttämään (tai hidastamaan, $ a < 0$).

\section{Paine p} %....................................................................

\textbf{Paineella} $p$ tarkoitetaan yksinkertaisimmillaan voimaa jaettuna pinta-alalle, jolle se kohdistuu.
Paineen SI-yksikkö on Pascal $Pa$.

\begin{align}
 &p = \frac{|\vec{F}|}{A}\\
 &[p] = \left[\frac{|\vec{F}|}{A}\right] = \frac{N}{m^2} = Pa
\end{align}

Virtausaineissa tilanne ei kuitenkaan ole näin yksinkertainen. Paine voidaan nimittäin määrittää 
mille tahansa virtausaineen reunoilla tai sen sisällä olevalle todelliselle tai kuvitteelliselle pinnalle. 
Itse asiassa virtausopissa paine voidaan (infinitesimaalisten kontrollitilavuuksien $dV$ avulla) määrittää
virtausaineen jokaiselle pisteelle eli $p = p(\vec{r}, t)$).

\section{Työ W} \label{def:W}%......................................................................

Mitä työ on? Mekaniikassa \textbf{työn W} yleinen määritelmä on

\begin{equation}
\label{eq:W}
 W = \int_S \vec{F} \cdot \vec{ds}
\end{equation}

Mitä tämä sitten tarkoittaa? 

Arkisestikin voimme todeta, että jonkin kappaleen siirtämisen ``työläys'' on suoraan verrannollinen

\begin{enumerate}
 \item Voimaan $F$, joka tarvitaan kappaleen liikuttamiseksi
 \item Matkaan $s$, joka kappaletta siirretään
\end{enumerate}

Kun voima on vakio ja reitti koko ajan voiman suuntainen, nämä verrannollisuudet voidaan yhdistää tuloksi ja 
(valitsemalla määritelmässä verrannollisuuskertoimeksi 1) määritellä työ

\begin{equation}
 W = |\vec{F}|s
\end{equation}

Yleisessä tapauksessa ei päästä näin helpolla, sillä kappaleeseen vaikuttavan voiman suuruus ja suunta voivat 
riippua esimerkiksi ajasta ja kappaleen paikasta eikä reittikään ole välttämättä lähelläkään suoraa. 

Onneksi 
mikä tahansa infinitesimaalisen lyhyt reitin pätkä $\vec{ds}$ voidaan katsoa hyvällä tarkkuudella suoraksi ja 
voiman reitin suuntainen komponentti saadaan pistetulolla eli

\begin{equation}
 dW = \vec{F} \cdot \vec{ds}
\end{equation}

Kun nämä infinitesimaalisen lyhyet reitin pätkät sitten summataan eli integroidaan saadaan työn yleinen määritelmä 
\ref{eq:W}.

Työn määritelmä voitaisiin avata sanallisesti vaikka seuraavasti:

\begin{quotation}
``Kun kappale, johon voima $\vec{F}$ vaikuttaa, kulkee 
reitin $S$ tekee voima kaavasta \ref{eq:W} laskettavissa olevan määrän työtä.''
\end{quotation}

 Huomioi, että tämä ei vaadi, että 
juuri voima $\vec{F}$ aiheuttaisi kappaleen liikkeen.

% Teho

\part{Termodynamiikka} % _____________________________________________________________________________________
\setcounter{chapter}{0}

% "Lämmön liikkeet", "energian tiede"
~\cite{4laws}~\cite{stattherintro}~\cite{thermoeng}~\cite{ltp1}

\chapter{Klassinen ja tilastollinen termodynamiikka} %--------------------------------------------------------

\textbf{Klassinen termodynamiikka} käsittelee makroskooppisia systeemejä, ilmiöitä ja suureita. Se kehitettiin 
olennaisilta osiltaan valmiiksi ennen kuin molekyylien olemassaolo oli yleisesti hyväksyttyä tai todennettua. 

Esimerkiksi lämpötekniikassa keskeiset lämpötekniset laitteet kuten lämpövoimakoneet, 
lämpöpumput, pumput, puhaltimet, turbiinit sekä lämmönvaihtimet ovat makroskooppisia systeemejä, joiden 
tutkimuksessa klassisella termodynamiikalla saadaan suhteellisen helposti kiinnostavia ja hyödyllisiä tuloksia.

Valitettavasti klassisen termodynamiikan lainalaisuuksia on hankala ymmärtää ja perustella itselleen. Tämä johtuu 
siitä, että pohjimmiltaan termodynamiikka käsittelee molekyylien energioiden tilastollisia ominaisuuksia.

\textbf{Tilastollinen termodynamiikka} redusoi klassisen termodynamiikan dynamiikkaan (ja kvanttimekaniikkaan).
\footnote{Vastaavasti kuin kemia pystyttiin aikoinaan redusoimaan kvanttimekaniikkaan.} 

Klassista termodynamiikkaa syvällisempänä ja tilastollisena tieteenä sitä on vaikeampi soveltaa käytäntöön. 
Toisaalta tilastollisen termodynamiikan käsitteillä termodynaamiset lainalaisuudet on mahdollista selittää ja 
perustella tyydyttävästi.

Tässä kirjassa opetellaan ennen kaikkea soveltamaan klassista termodynamiikkaa lämpöteknisiin ongelmiin. 
Kun se on ymmärryksen kannalta tarpeellista, käytän tilastollista termodynamiikkaa selittämään asioita.

\chapter{Systeemi} % -----------------------------------------------------------------------------------------
% % Avoin, suljettu, adiabaattinen

Systeemin käsite on lämpötieteissä hyvin keskeinen ja hyödyllinen.

\section{Kontrollitilavuus}  %........................

\textbf{Kontrollitilavuus} on mielivaltaisen avaruudessa sijaitsevan \textit{kontrollipinnan} sisältämä tilavuus. 
Pinnan sijainti ja muoto voi myös riippua ajasta. 
\section{Systeemin käsite} %............................................................

\textbf{Systeemin} käsite on erityisesti termodynamiikassa keskeinen. Systeemiksi voidaan valita mikä tahansa 
kontrollitilavuus, kontrollitilavuuksien yhdistelmä tai molekyylijoukko. Systeemi
voidaan siis valita täysin vapaasti, mutta usein luontevasti (tai ovelasti) määritelty systeemi helpottaa 
haluttujen tulosten saamista tai jopa ylipäätään mahdollistaa sen.

Vaikka automaatiolle keskeisessä systeemiteoriassa systeemin käsite on suomennettu \textit{järjestelmäksi},
lämpötieteissä puhutaan anglisistisesti \textit{systeemeistä}\footnote{Mahdollisesti sellaiset asiat kuin 
sisäenergia ja entropia merkityksineen ovat omian luomaan sellaista kuvaa, että lämpötieteelliset systeemit eivät 
yleensä ole pohjimmiltaan erityisen ``järjestelmällisiä'' tai ``järjestyksessä''.}.

\section{Ympäristön käsite} %...........................................................

\textbf{Ympäristö} käsittää termodynamiikassa kaiken systeemin ulkopuolella olevan. Yhdessä systeemi ja ympäristö 
muodostavat siis \textbf{maailmankaikkeuden}.

\section{Avoin ja suljettu systeemi} %..................................................

\textbf{Avoimen systeemin} kontrollipinta on avoin eli sen läpi voi kulkea ainetta. Tyypillisiä esimerkkejä 
ovat \textit{lämmönsiirrin} ja \textit{turbiini}.

\textbf{Suljetun systeemin} kontrollipinta on suljettu eli sen läpi ei voi kulkea ainetta. Tyypillisiä 
esimerkkejä ovat \textit{suljettu kaasusäiliö} ja \textit{mäntämoottorin sylinteri} (venttiilien ollessa 
puristus- ja työtahtien aikana kiinni).

\section{Eristämätön ja eristetty systeemi}%............................................

\textbf{Eristämättömän systeemin} kontrollipinnan läpi voi vapaasti siirtyä lämpöä. 
Tällainen on esimerkiksi \textit{ilmatilavuus keskellä muuta ilmaa}.

\textbf{Täydellisesti eristetyn eli adiabaattisen systeemin} kontrollipinnan läpi ei siirry lämpöä.

Täydellisesti eristettyjä systeemejä ei tietenkään ole todellisuudessa olemassa, vaan \textit{eristeillä} 
on jokin äärellinen lämpövastus, joka vähentää systeemistä poistuvan lämmön määrää. Hyvin eristettyjä 
systeemejä ovat esimerkiksi \textit{termospullo} ja \textit{passiivienergiatalo}.

\section{Taselaskenta} %................................................................

Tyypillisin systeemin käsitteen hyödyntämiskohde ovat erilaiset taselaskelmat. Nämä liittyvät 
yleensä säilymislakeihin, joista lämpötekniikassa yleisimmät ovat massan säilyminen, energian 
säilyminen (\nameref{def:td1}) ja ``liikemäärävirran säilyminen'' eli Newtonin II laki.

\chapter{Energian tiede} % ------------------------------------------------------------------------------

Termodynamiikan ydin on energia. Termodynamiikan pääsäännötkin käsittelevät energiaa; sen määrää, 
muotoa, laatua, jakautumista ja niin edelleen.

\section{Energia E} %....................................................................

Ennen kuin sukellamme varsinaiseen termodynamiikkaan, on tärkeää selvittää, mitä energia on. Mielenkiintoista 
kyllä, energian määritelmä ei varsinaisesti kuulu termodynamiikkaan vaan mekaniikkaan. Energian määritelmä on 
seuraavanlainen:

\begin{quote}
 Energia on kykyä tehdä työtä.
\end{quote}

Systeemin yhteydessä ja työn ymmärtämisen kautta se tarkoittaa itse asiassa seuraavaa:

\begin{quotation}
 Systeemillä on energiaa, kun on mahdollista löytää toinen systeemi, johon systeemi voi kohdistaa 
 voiman $\vec{F}$ kun toinen systeemi liikkuu reitin $S$.
\end{quotation}

Systeemin ei tarvitse kyetä tuottamaan mielivaltaista voimaa mielivaltaiselle reitille vaan riittää että 
voidaan keksiä jokin järjestely mitä analysoitaessa kaavaa \ref{eq:W} voidaan soveltaa tarkasteltavan 
systeemin tuottamaan voimaan $\vec{F}$.

Kääntäen voidaan todeta että kaikilla systeemeillä, jotka tuottavat johonkin muuhun voiman edes 
infinitesimaalisen lyhyellä matkalla, on energiaa.

Käsiä heiluttelevana loppukaneettina totean, että energia on siis ``voimantuottokykyä''\footnote{
Mietipä tätä: \textit{voimanlähteellä} on aina jokin \textit{teho} $P = \dfrac{dE}{dt}$!}.

\section{Potentiaalienergia ja liike-energia} %........................................

Kuten muistamme, työn yleinen määritelmä on

\begin{equation}
 W = \int_S \vec{F} \cdot \vec{ds}
\end{equation}

ja systeemillä on energiaa mikäli se kykenee tuottamaan kaavassa toimivan voiman $\vec{F}$.

Reitti $S$ on vain mielivaltainen avaruuskäyrä eikä sitä voi analysoida sen enempää\footnote{
Ainakaan millään käytännössä kovin hyödyllisellä tavalla.}.

Sen sijaan kiinnostava ja vastattavissa oleva kysymys on:

\begin{quote}
 Millainen työn määritelmässä esiintyvä voima $\vec{F}$ voi olla? 
\end{quote}

Fysiikassa on onnistuttu palauttamaan kaikki mahdolliset maailmankaikkeudessa esiintyvät voimat 
\textit{neljään perusvuorovaikutukseen} kuuluviksi:

\begin{enumerate}
 \item Gravitaatio
 \item Sähkömagnetismi
 \item Heikko ydinvoima
 \item Vahva ydinvoima
\end{enumerate}

Näiden vuorovaikutusten voimien suuruus riippuu etäisyydestä voiman aiheuttajaan ja mahdollisesti myös ajasta. 
Tätä kuvataan \textit{voimakentillä} eli määrittämällä voima ajan ja paikan funktiona $\vec{F}(\vec{r}, t)$. 

Esimerkiksi sähkövarauksen $q_1$ toiseen sähkövaraukseen $q_2$ aiheuttama voimakenttä on

\begin{equation}
 \vec{F}(\vec{r}) = \frac{1}{4\pi\epsilon_0}\frac{q_1q_2}{|\vec{r}|^2}\hat{r}
\end{equation}

ja massan $m_1$ toiseen massaan $m_2$ aiheuttama voima on (Newtonin painovoimateoriassa)

\begin{equation}
 \vec{F}(\vec{r}) = G\frac{m_1m_2}{|\vec{r}|^2}\hat{r}
\end{equation}

\subsection{Potentiaalienergia E\textsubscript{p}}

\begin{quotation}
Systeemillä on \textbf{potentiaalienergiaa $E_p$}, mikäli se on jossain siihen vaikuttavassa voimakentässä 
paikassa, josta se voi liikkua sellaisen reitin että voimakenttä tekee siihen positiivisen työn.
\end{quotation}

Potentiaalienergia on siis systeemin mahdollisuus saada voimakenttä tekemään siihen työtä. Eikö silloin 
energia ole itse asiassa voimakentän aiheuttavalla systeemillä? Pohjimmiltaan kyllä.

Mutta kun systeemi kulkee reittinsä voimakentässä se voi vuorostaan aiheuttaa voiman johonkin toiseen 
systeemin, joka voi tällöin kulkea jonkin reitin - eli systeemi tekee työtä! Se toki ``vain'' välittää 
voimakentän energiaa, mutta käytännössä näin voidaan esimerkiksi tehdä työtä johonkin, mihin voimakenttä 
ei kohdistu ainakaan toivotulla tavalla.

Esimerkiksi gravitaatiokentän energialla on vaikea saada elektroneja liikkumaan maanpinnan suuntaisesti, mutta 
siinä voidaan onnistua välillisesti näin:

\begin{enumerate}
 \item Vesimassalla on potentiaalienergiaa, sillä se on korkealla gravitaatiokentässä.
 \item Kun vesi päästetään putoamaan, gravitaatiokenttä tekee siihen työtä.
 \item Vesi kulkee turbiinin läpi ja tekee siihen työtä.
 \item Turbiini tekee työtä generaattorin roottoriin.
 \item Generaattorin roottori aiheuttaa käämien elektroneihin sähkömagneettisen voiman, joka tekee niihin 
 työtä.
 \item Elektronit liikkuvat johtimessa. Meillä on sähkövirtaa!
\end{enumerate}

Systeemille voidaan antaa potentiaalienergiaa liikuttamalla sitä voimakentässä niin, että voimakentän 
tekemä työ on negatiivinen. Tällöin joudutaan tuottamaan jokin toinen voima tekemään vastaavan 
suuruinen positiivinen työ. Tämän voiman tuottamiseen käytetty energia saadaan näin varastoitua 
potentiaalienergiaksi.

\subsection{Liike-energia E\textsubscript{k}}

\begin{quotation}
 Systeemillä on liike-energiaa, kun se voi pienentämällä nopeuttaan tehdä työtä.
\end{quotation}

Newtonin toinen laki kertoo voiman, jonka hidastuva systeemi aiheuttaa:

\begin{equation}
 \vec{F} = m\vec{a}
\end{equation}

Sijoitetaan se työn määritelmään:

\begin{equation}
 W = \int_S \vec{F} \cdot \vec{ds} = \int_S m\vec{a} \cdot \vec{ds} = m\int_S \vec{a} \cdot \vec{ds}
\end{equation}

Käytetään kiihtyvyyden ja nopeuden määritelmiä 

\begin{align}
\vec{a} &= \frac{d\vec{v}}{dt} \\
\vec{v} &= \frac{d\vec{s}}{dt} \Leftrightarrow d\vec{s} =  \vec{v}dt\\
\end{align}

Sijoitetaan ja sievennetään:

\begin{equation}
 W = m\int_S \vec{a} \cdot \vec{ds} = m\int_S \frac{d\vec{v}}{dt} \cdot \vec{v}dt = m\int_S d\vec{v} \cdot \vec{v}
\end{equation}

Avataan pistetulo integroimalla komponenteittain:

\begin{equation}
 W = m\int_S d\vec{v} \cdot \vec{v} = m \left(\int_S v_xdv_x + \int_S v_ydv_y + \int_S v_zdv_z\right)
\end{equation}

saadaan

\begin{equation}
 W = m \left(\int_S v_xdv_x + \int_S v_ydv_y + \int_S v_zdv_z\right) 
 = m \left(\frac{1}{2}v_x^2 + \frac{1}{2}v_y^2 + \frac{1}{2}v_z^2\right)
\end{equation}

josta saadaan lopulta nopeudesta saatavaksi työksi

\begin{equation}
 W = m \left(\frac{1}{2}v_x^2 + \frac{1}{2}v_y^2 + \frac{1}{2}v_z^2\right)
 = \frac{1}{2}m \left(v_x^2 + v_y^2 + v_z^2\right)
 = \frac{1}{2}m (\vec{v} \cdot \vec{v})
 = \frac{1}{2}m|\vec{v}|^2
\end{equation}

Liike-energia on siis

\begin{equation}
 E_k = W = \frac{1}{2}m|\vec{v}|^2
\end{equation}

Systeemille voidaan antaa liike-energiaa aiheuttamalla siihen nettovoima, joka tekee työtä systeemin 
nopeuden kasvattamiseksi. Tähän kulunut energia varastoituu systeemin liike-energiaksi.

% Liike-energian jako translaatioon, rotaatioon ja vibraatioon kaavoineen
% Nollatason mielivaltaisuus

\chapter{Systeemin energiat} % --------------------------------------------------------------------------

\section{Ulkoiset energiat} %............................................................

\section{Sisäenergia} %.................................................................

Systeemillä kokonaisuutena voi siis olla erinäisiä potentiaali- ja liike-energioita. Se voi esimerkiksi olla verrattain korkealla 
tai siirtymä-, pyörimis- tai värähtelyliikkeessä. Tämä ``makromekaniikka'' on tuttua ja helposti ymmärrettävää.

Termodynamiikassa keskeisimmässä osassa ovat kuitenkin systeemin muodostavien molekyylien, atomien, elektronien ja atomiydinten 
energiat.

Vaikka systeemi kokonaisuutena ei olisi liikkeessä, ovat siihen kuuluvat molekyylit jatkuvasti 
\textit{lämpöliikkeessä}\footnote{Mikäli systeemin lämpötila on yli 0 K.}. Tämä lämpöliike jakautuu kolmeen tyyppiin: molekyylien 
siirtymiseen eli \textit{translaatioon}, ei-pallosymmetristen molekyylien pyörimiseen omien symmetria-akseleidensa ympäri eli 
\textit{rotaatioon} ja moniatomiseen molekyyliin kuuluvien atomien värähtelyyn toistensa suhteen eli \textit{vibraatioon}. 
Molekyylit voivat siis liikkua, pyöriä ja värähdellä vaikka systeemi kokonaisuutena ei tekisi mitään näistä asioista.

Systeemin mikroskooppisten osasten liike-energioita tutkii erityisesti \textbf{tilastollinen termodynamiikka} mutta itse asiassa 
myös ne klassisen termodynamiikan käsitteet jotka tuntuvat vaikeasti ymmärrettäviltä tai ``käsiä heilutellen hatusta vedetyiltä'' 
kuten sisäenergia, lämpötila ja entropia kuvaavat itse asiassa näiden ilmiöiden makroskooppisesti mitattavia seurauksia.

\subsection{Sisäenergia U}

Missä on liikettä, siellä on myös liike-energiaa. Termodynamiikassa systeemin kaikkien molekyylien yhteenlaskettua lämpöliikkeen 
energiaa kutsutaan systeemin \textbf{sisäenergiaksi U}. Lämpöliikkeen energioihin kuuluvat (ainakin) 
molekyylien translaation, rotaation ja värähtelyn energiat.

\subsection{Entalpia H} \label{def:H} %.......................................

\textbf{Entalpia $H$} on vain apusuure, joka on määritelty seuraavasti:

\begin{equation}
 h = u + pv
\end{equation}

Näistä kaavoista nähdään että myös ominaisentalpia on ominaisenergiaa ja sisältää ominaissisäenergian sekä termin $pv$ 
suuruisen lisäominaisenergian. Mikä sitten on entalpian fysikaalinen merkitys ja käytännön hyöty?

% h = h(T)?

\paragraph{Lämmönsiirto vakiopaineessa}

Olkoon meillä vakiopaineinen\footnote{``Isobaarinen''.} systeemi, johon tai josta siirtyy lämpöä. Vakiopaineinen systeemi 
ei välttämättä ole vakiotilavuuksinen\footnote{``Isokoorinen''.}, eli sen tilavuus voi muuttua jolloin systeemi tekee 
työtä ulkoisia painevoimia vastaan tai ympäristö tekee työtä systeemin painevoimia vastaan. 

Siirtyvä lämpö voi nyt  
olla positiivinen (systeemiin) tai negatiivinen (systeemistä). Työ aiheutuu lämmön aikaansaamasta tilavuuden muutoksesta. 
Mikäli lämpöä tuodaan systeemiin, sen tilavuus kasvaa lämpölaajenemisen johdosta ja systeemi tekee työtä ympäristöön. 
Tilavuudenmuutoksen merkki on siis sama kuin lämmön ja työn merkki on päinvastainen.

Termodynamiikan ensimmäisen pääsäännön mukaan differentiaaliselle lämmöntuonnille pätee

\begin{equation}
 du = dq + dw
\end{equation}

Differentiaalinen tilavuudenmuutostyö $dw$ voidaan korvata seuraavasti:

\begin{equation}
 dw = -pdv
\end{equation}

Ja sijoittaa I pääsääntöön:

\begin{equation}
\label{psIpdv}
 du = dq - pdv
\end{equation}

Tästä nähdään että sisäenergian muutos on erisuuri kuin tuotu lämpömäärä:

\begin{equation}
 du \neq dq
\end{equation}

Pidemmän päälle kaavan \ref{psIpdv} muistaminen johtaisi työläyteen (ja luultavasti myös huolimattomuusvirheisiin). Entalpian 
differentiaalinen muutos on yleisesti

\begin{equation}
 dh = du + pdv + vdp
\end{equation}

Vakiopaineessa $dp = 0$, joten

\begin{equation}
 dh = du + pdv + vdp = du + pdv 
\end{equation}

ja tästä saadaan ratkaistua sisäenergian muutos

\begin{equation}
 du = dh - pdv
\end{equation}

Sijoitetaan yhtälöön \ref{psIpdv}:

\begin{equation}
 dh - pdv = dq - pdv
\end{equation}

Tilavuudenmuutostyö supistuu ja

\begin{equation}
 dh = dq
\end{equation}

Koska yhtäsuuruus on näin yksinkertainen, se voidaan suoraan yleistää muillekin kuin differentiaalisille lämpömäärille:

\begin{equation}
 \Delta h = \Delta q
\end{equation}

Eli kun systeemistä tai systeemiin siirtyy lämpöä vakiopaineessa systeemin entalpia muuttuu tuodun lämmön verran. Tämä 
on erityisen kätevää kemiallisia reaktioita ja lämpövoimakoneita käsiteltäessä; kun seurataan sisäenergian sijaan 
entalpian kehitystä voidaan reaktiolämmöt ja lämmönsiirtimissä siirtyvä lämpö lisätä tai vähentää suoraan siitä.

\paragraph{Virtausenergia}
% Kuva!

Entalpian määritelmässä esintyvä termi $pdv$ voidaan ymmärtää myös aineen siirtymisen vaatimaksi energiaksi. 

\textbf{Huom:}

\begin{enumerate}
 \item Tässä on kysessä työ, joka tehdään (virtausaineesta koostuvaa) systeemiä siirrettäessä. Kyse ei ole siis liike-energiasta, 
  joka on oma terminsä.
 \item Tämä työ ei ole verrannollinen systeemin nopeuteen toisin kuin liike-energia. (Muistathan että $v \neq \left|\vec{v}\right|$!)
\end{enumerate}

Selvennän tätä entalpian tulkintaa esimerkillä.

Olkoon meillä putki, jossa virtaa jotain virtausainetta. Valitaan putken sisäpoikkileikkauksen (ala $A$) muotoinen ja $L$:n 
pituinen kontrollitilavuus avoimeksi systeemiksemme. Kun systeemin ajanhetkellä $t_1$ sisältämä virtausaine on 
ajanhetkellä $t_2$ siirtynyt juuri kokonaisuudessaan ulos systeemistä on sen täytynyt tehdä edessään olevia 
painevoimia vastaan työ, jonka suuruus on

\begin{equation}
 W = |\vec{F}| L = pAL
\end{equation}

Huomataan, että systeemin tilavuushan on $AL$, joten

\begin{equation}
 W = pV
\end{equation}

ja ominaissuureilla 

\begin{equation}
 w = pv
\end{equation}

joka esiintyy ominaisentalpian määritelmässä. Tässä tulkinnassa systeemin ominaisentalpia sisältää siis

\begin{enumerate}[a)]
 \item Sisäenergian eli lämpöliikkeen energian
 \item Virtausenergian eli virtauksen siirtotyön tekevän energian
\end{enumerate}

Tämä entalpian ominaisuus taas on kätevä niissä lukemattomissa lämpötekniikan sovellutuksissa, missä prosessissa on 
olennaisessa osassa putkessa virtaava virtausaine. Nimittäin entalpiaa näin käyttämällä virtauksen jatkuminen muuttuu 
analyysissa ikäänkuin sisäänrakennetuksi itsestäänselvyydeksi ja voidaan keskittyä tavoitteen kannalta kiinnostavampiin 
ilmiöihin, esim. lämmönsiirtoon.

\subsection{Lämpötila T} \label{sssection:T} %..............................................

Sisäenergia on siis systeemin molekyylien liike-energioiden summa. Mitkä sitten ovat yksittäisten molekyylien energiat? 
Missä tahansa käytännön systeemissä on niin valtava määrä molekyyleja\footnote{Muistathan että yhdessä moolissa on 
noin $6,022\cdot10^{23}$ molekyylia.}, ettei ole mielekästä määrittää kunkin energiaa erikseen.

On kuitenkin mahdollista määrittää molekyylien energioiden jakauma. Molekyylien energiat ovat tietenkin kvantittuneet, 
jolloin energiajakauma kertoo mikä osuus molekyyleista on milläkin energiatasolla. Tämä voidaan tehdä kullekin molekyylien 
energian tyypille (translaatio, rotaatio, vibraatio).

Tilastollisen termodynamiikan mukaan $N$ molekyylin jakauma $m$ energiatasojoukolle voi toteutua

\begin{equation}
% \prod:in rajat näkyvät väärin vaikka intlimits käytössä!
\label{eq:ejakW}
 W = \frac{N!}{\prod_{i=1}^m(n_i!)}
\end{equation}

tavalla kun $n_i$ on molekyylien määrä energiatasolla $i$.

Mikä $n_i$ eli molekyylien energiajakauma sitten on? Yllä oleva kaava ei ota tähän kantaa. 
Nimittäin luonto toimii satunnaisesti eli kaikki energiajakaumat ovat mahdollisia. Tilastollisen termodynamiikan 
teoria kertoo kuitenkin meille, mikä jakauma on todennäköisin eli voi toteutua kaavan \ref{eq:ejakW} mukaisesti 
suurimmalla määrällä eri tapoja. Yleensä se on \textit{Boltzmann-jakauma}:

\begin{equation}
\label{eq:boltzmann}
 n_i = n_0p_ie^{-\dfrac{\epsilon_i}{k_BT}}
\end{equation}

jossa $\epsilon_i$ on energiatasojoukon $i$ keskimääräinen energia, $p_i$ on energiatasojoukkoon kuuluvien energioiden 
määrä\footnote{Tällä otetaan huomioon se, että molekyylien vaikuttaessa toisiinsa energiatasot leviävät energiavöiksi ja 
se, että useaa erilaista liiketilaa voi vastata sama energia.}, $n_0$ on molekyylien määrä alimmalla 
energiatasolla, $k_B$ Boltzmannin vakio ja \textbf{T on systeemin lämpötila kelvineinä}.

% Kelvinasteen ja Boltzmannin vakion määritelmät!

Systeemin lämpötila kertoo siis sen, millä tavalla molekyylien liike-energiat ovat jakautuneet.

\subsection{Sisäenergia lämpötilan funktiona} %.......................................

Systeemin sisäenergia on kaikkien sen molekyylien energioiden summa:

\begin{equation}
 U = \sum_{i=1}^m\epsilon_in_i
\end{equation}

Sijoitetaan tähän kaava \ref{eq:boltzmann}:

\begin{equation}
 U = \sum_{i=1}^m\epsilon_in_0p_ie^{-\dfrac{\epsilon_i}{kT}}
\end{equation}

Tästä näemme että systeemin sisäenergia riippuu

\begin{enumerate}
 \item Systeemin lämpötilasta $T$
 \item Systeemin ainemäärästä $n$ ($n_i$:n kautta)
 \item Systeemin koostumuksesta (sen muodostavan aineen energiatasojen suuruuksista $\epsilon_i$:n kautta)
\end{enumerate}

Mikäli systeemin koostumus ei muutu (esim. reaktioiden tai virtausten johdosta) pätee

\begin{equation}
 U = U(T, n)
\end{equation}

Ainemäärän vaikutus saadaan eliminoitua käyttämällä molaarista sisäenergiaa

\begin{equation}
 U_m = \frac{U}{n} = U_m(T)
\end{equation}

Koska systeemin massa riippuu sen ainemäärästä moolimassan kautta pätee myös ominaissisäenergialle vastaavasti:

\begin{equation}
 u = u(T)
\end{equation}

Eli ominaissisäenergia on vain lämpötilan funktio. Tämä on olennainen tulos, joka klassisessa termodynamiikassa 
otetaan 'annettuna' eli puhtaasti empiirisenä havaintona. Opittuamme tilastollisen termodynamiikan avulla lämpötilan 
todellisen merkityksen tulos oli kuitenkin helppo johtaa ja ymmärtää.

% \subsection{Mikroskooppiset potentiaalienergiat eli sidosenergiat}
% 
% Mikroskooppisia potentiaalienergioita kutsutaan yleensä \textbf{sidosenergioiksi} sillä ne sitovat aineen eri osasia 
% yhteen niihin muotoihin, joihin olemme tottuneet. Niiden tutkimus kuuluu enemmän kemian, kvanttimekaniikan ja ydinfysiikan 
% kuin (edes tilastollisen) termodynamiikan alueelle. 
% 
% Niistä on kuitenkin syytä puhua jonkin verran myös lämpötekniikan oppikirjassa, sillä niillä on keskeinen rooli
% etenkin monissa energiantuotantoprosesseissa: polttoprosesseissa (kemiallinen sidosenergia), ydinvoimassa (ytimen 
% sidosenergia) sekä myös erilaisissa aurinkopaneeleissa (elektronien sidosenergia yleisesti, joskus myös kemiallinen 
% sidosenergia).
% 
% \section{Energiat lämpötieteissä}
% 
% Systeemillä voi siis olla montaa erityyppistä energiaa:
% 
% \begin{enumerate}
%  \item Liike-energiat:
%  \begin{enumerate}
%   \item \underline{Systeemin liike-energia}
%   \item \underline{Lämpöliikkeen liike-energia eli sisäenergia}
%  \end{enumerate}
%  \item Fysiikan perusvoimiin liittyvät potentiaalienergiat:
%  \begin{enumerate}
%   \item \underline{Gravitaation potentiaalienergia}
%   \item Sähköiset energiat:
%   \begin{enumerate}
%    \item Sähkömagneettinen potentiaalienergia
%    \item \textit{Kemiallinen sidosenergia}
%   \end{enumerate}
%   \item \textit{Heikon vuorovaikutuksen sidosenergia}
%   \item Vahvan vuorovaikutuksen sidosenergia
%  \end{enumerate}
% \end{enumerate}
% 
% \underline{Alleviivattuja energioita} käsitellään lämpötieteissä jatkuvasti. \textit{Kursivoidut 
% energiat} tulevat kyseeseen, mikäli systeemissä tapahtuu kemiallisia tai ydinreaktioita. 
% 
% Sähkömagneettisella potentiaalienergialla on väliä vain mikäli systeemin kokonaisvaraus ei ole nolla ja 
% se on ulkoisessa sähkömagneettisessa kentässä. Lämpötieteissä ja -tekniikassa tällaiset tapaukset ovat 
% harvinaisia.
% 
% Vahvan vuorovaikutuksen sidosenergia sitoo kvarkit protoneiksi ja neutroneiksi. Sillä ei toistaiseksi 
% ole käytännön sovelluksia.
% 
% \section{Energioiden koordinaatistoriippuvuus}
% 
% Systeemin energiat voidaan jakaa myös seuraavasti:
% 
% \begin{enumerate}
%  \item Koordinaattijärjestelmän ja/tai potentiaalienergian nollakohdan valinnasta riippuvat:
%  \begin{enumerate}
%   \item Systeemin liike-energia
%   \item Gravitaation potentiaalienergia
%   \item Sähkömagneettinen potentiaalienergia
%  \end{enumerate}
%  \item Näistä valinnoista riippumattomat:
%  \begin{enumerate}
%   \item Lämpöliikkeen liike-energia eli sisäenergia
%   \item Kemiallinen sidosenergia
%   \item Heikon vuorovaikutuksen sidosenergia
%   \item Vahvan vuorovaikutuksen sidosenergia
%  \end{enumerate}
% \end{enumerate}

\subsection{Lämpökapasiteetit} %.........................................................

Lämpötila, sisäenergia ja sisäenergian sisältävä entalpia mittaavat siis enemmän tai vähemmän samaa asiaa. Ne voidaan 
näinollen luultavasti kytkeä toisiinsa jollakin yksinkertaisella tavalla. Tästä kytkennästä on myös se olennainen hyöty, 
että laskennassa hyödylliset mutta vaikeasti mitattavat sisäenergia ja entalpia saadaan kytkettyä harvemmin kiinnostavaan 
mutta helposti mitattavaan lämpötilaan.

Termodynamiikassa kytkentään käytetään \textbf{ominaislämpökapasiteettia vakiotilavuudessa $c_v$} ja 
\textbf{ominaislämpökapasiteettia  vakiopaineessa $c_p$}. Määritellään nämä ominaislämpökapasiteetit. Sisäenergian ja 
entalpian differentiaaliset muutokset voi kytkeä lämpötilan differentiaaliseen muutokseen osittaisderivaattojen avulla:

\begin{align}
 du = \frac{\partial u}{\partial T}\bigg|_vdT\\
 dh = \frac{\partial h}{\partial T}\bigg|_pdT
\end{align}

Nämä derivoinnin tuloksena syntyvät funktiot on nimetty ominaislämpökapasiteeteiksi (lyhyesti ``ominaislämmöiksi'') 
vakiotilavuudessa ja vakiopaineessa:

\begin{align}
 \frac{\partial u}{\partial T}\bigg|_v = c_v(T)\\
 \frac{\partial h}{\partial T}\bigg|_p = c_p(T)\\
 du = c_vdT\\
 dh = c_pdT
\end{align}

Yleisesti ottaen ominaislämmöt eivät ole yksinkertaisia tai helposti teoreettisesti johdettavissa olevia lämpötilan 
funktioita. 

Käytännön laskennassa ominaislämmöille käytetään taulukoituja, käyräksi piirrettyjä tai kokeellisen 
polynomiapproksimaation muodossa olevia funktioita. Jos toimitaan kapealla lämpötila-alueella ja tulokset on 
tärkeämpää saada nopeasti kuin tarkkoina voidaan ominaislämpö olettaa vakioksi lämpötila-alueella. Tällaisia tilanteita 
ovat esim. alustavat tunnustelulaskelmat, pika-analyysit ja tentit.

% \frac{c_p}{c_v} = \gamma, c_p - c_v = R
% Kokoonpuristumattomattomalle c_p = c_v

\chapter{Energian siirtymistavat} % ----------------------------------------------------------------------------------

Systeemillä voi siis olla monenlaista energiaa. Myös systeemin ympäristöllä voi olla näitä 
energioita.Systeemin rajojen yli energia voi kuitenkin termodynamiikassa siirtyä vain kahdella tavalla: \textbf{lämpönä $Q$} 
tai \textbf{työnä $W$}. 

Nämä eivät sinänsä ole energiamuotoja, vaan \textbf{energian siirtymistapoja}, vaikka niillä onkin energian
yksiköt. Vastaavasti systeemillä tai ympäristöllä sinänsä ei myöskään voi olla työtä tai lämpöä vaan 
varastoituessaan systeemiin tai ympäristöön ne muuttuvat aina johonkin muuhun muotoon\footnote{Vaikka 
sisäenergiasta puhutaankin usein ``lämpöenergiana'' ja ``entalpia'' tulee kreikan sanasta \textit{enthalpos}, 
``lämpö sisällä''.}.

Helpoin tapa ymmärtää tämä ero on ehkä energian määritelmän mieleen palauttaminen:

\begin{quote}
 Energia on kykyä tehdä työtä.
\end{quote}

Tämän määritelmän valossa vaikuttaa siltä, että työ on jotain muuta kuin energiaa\footnote{Itse asiassahan 
se on määritelty mekaniikassa paljon energiaa eksaktimmin (katso \nameref{def:W}).}.

\section{Merkkisopimus} \label{sec:merkkisopimus} %........................................

Systeemiin siirtyvä työ ja lämpö ovat aina positiivisia, systeemistä siirtyvät negatiivisia.

Päinvastoin voidaan todeta että mikäli työtä tai lämpöä ei alunperin tiedetä, ratkaistun 
työn tai lämmön etumerkki kertoo, siirtyikö se systeemistä vai systeemiin.

\section{Lämpö Q} %........................................................................

Energia siirtyy lämpönä systeemin rajojen yli siksi, että systeemin ja ympäristön 
välillä on lämpötilaero. 

Siirtyminen systeemin rajojen yli voi tapahtua millä tahansa \textbf{lämmönsiirtotavalla}:

\begin{itemize}
 \item johtumalla
 \item kulkeutumalla\footnote{``Konvektiolla''.}
 \item säteilemällä
 \item tai jollain näiden 
yhdistelmällä\footnote{Erityisen merkittävä yhdistelmä on johtumisesta ja kulkeutumisesta 
koostuva ``\textbf{konvektiivinen lämmönsiirto}''.}.
\end{itemize}

Lämpö siirtyy aina korkeammasta lämpötilasta matalampaan. Kuten myöhemmin selviää, tämä 
on seurausta termodynamiikan II pääsäännöstä. Aina kun lämpöä siirtyy, joko systeemin tai 
ympäristön entropia ja sen myötä maailmankaikkeuden epäjärjestys kasvaa.
Tähän liittyen lämpö on ``epäjärjestynyttä energiaa'', jolla voidaan tehdä vähemmän 
erilaisia asioita kuin työllä.

Lämmönsiirto muuttaa ensisijaisesti systeemin tai ympäristön sisäenergiaa. (Mieti, 
miten tämä liittyy siihen, että sisäenergia on ``lämpöliikkeen liike-energiaa''.)

\section{Työ W} %.................................................................

Energia siirtyy työnä systeemin rajojen yli siksi, että systeemin ja ympäristön 
välillä on nettovoima. 

Työ voi olla esimerkiksi systeemin tilavuudenmuutostyötä, turbiinin akselityötä, 
sähkövirran energiaa jne.

Työ on ``järjestynyttä energiaa'', joka voidaan muuttaa vaihtelevilla hyötysuhteilla 
moniksi muiksi energian muodoiksi. Kun pelkästään työtä siirtyy ei systeemin tai ympäristön 
entropia muutu.

Työ voidaan muuttaa myös sisäenergian kautta kokonaan lämmöksi. Lämpöä taas ei voida 
(sisäenergiankaan kautta) muuttaa kokonaan työksi, koska se rikkoisi termodynamiikan 
II pääsääntöä.

\chapter{Tasapaino ja epätasapaino} %---------------------------------------------------------

\section{Mekaaninen tasapaino} %...............................................

Systeemit ovat mekaanisessa tasapainossa, kun niiden välillä ei esiinny nettovoimaa eikä 
siis työtä:

\begin{align}
 &\sum \vec{F_{1-2}} = \vec{0} \\
 &W = \int_S \vec{F{1-2}} \cdot \vec{ds} = \int_S \vec{0} \cdot \vec{ds} = 0
\end{align}

\section{Terminen tasapaino} %.................................................

Systeemit ovat termisessä tasapainossa, kun niiden välillä ei ole lämpötilaeroa eikä 
siis siirry lämpöä:

\begin{align}
 &T_{1} = T_{2}\\
  &\Rightarrow Q_{1-2} = 0
\end{align}

(Koska eri lämmönsiirtotapoja ei voi työn tavoin palauttaa yhteen kaavaan, lämmön 
siirtymättömyyttä ei tässä perustella matemaattisesti.)

%\section{Termodynaaminen tasapaino} % (?)

\section{Dynaaminen tasapaino} %...............................................

% Myös jatkuvuustila

\chapter{Tilasuureet} \label{chapter:tilasuureet} % ------------------------------------------

Niitä termodynaamisia suureita, jotka riippuvat vain toistensa arvoista tarkasteluhetkellä 
eivätkä siitä millaisella prosessilla näihin arvoihin on tultu kutsutaan tilasuureiksi.

\begin{itemize}
 \item Perustavimmat tilasuureet ovat \textbf{paine p}, \textbf{lämpötila T}, 
    \textbf{tilavuus V}, \textbf{sisäenergia U} sekä \textbf{entropia S}.
 \item Keskeisiä ovat myös näiden johdannaissuureet \textbf{entalpia H}, 
  \textbf{Helmhotzin vapaaenergia F} sekä \textbf{Gibbsin vapaaenergia G}.
\end{itemize}

Mikä tahansa funktio, joka sisältää vain tilasuureita, massaa ja ainemäärää on myös tilasuure. 
Niinpä myös ominaistilavuus v, tiheys $\rho = \frac{1}{v}$, ominaissisäenergia u, molaarinen 
ominaissisäenergia $U_m$ jne. ovat tilasuureita.

% Esimerkkejä tilasuureen merkityksestä.

\textbf{Lämpö Q} ja \textbf{työ W} \underline{eivät} ole tilasuureita!

% Esimerkkejä tästä.

\section{Termodynaamiset potentiaalit} %................................................

% ``Termodynaamiset potentiaalit'', selitys eksaktista differentiaalista, pyörteettömyydestä, 
% vertaus potentiaalienergiaan ja sähköiseen potentiaaliin.

\section{Aineen olomuodot} %............................................................

% Kanneton astia: vrt. kiinteä - neste (asettuu muotoon) - kaasu ja plasma (karkaavat ilman selkeää vaakasuoraa pintaa)

\section{Vapausasteet} %................................................................

\section{Vakioprosessit}%...............................................................

\section{Tilanyhtälöt}%.................................................................

% Tilanyhtälö, ideaalikaasun tilanyhtälö

\subsection{Ideaalikaasun tilanyhtälö}

Ideaalikaasu on yksinkertainen kaasun malli. Ideaalikaasumallissa oletetaan että kaasun muodostavilla 
hiukkasilla ei ole tilavuutta eivätkä ne vaikuta toisiinsa muuten kuin törmäämällä kimmoisasti. 
Mikäli kaasun tiheys ei ole kovin korkea tai lämpötila matala näissä oletuksissa 
ei tehdä suurta virhettä, sillä kaasuhiukkaset ovat hyvin pieniä verrattuna niiden väliseen tilaan 
ja hiukkasten korkea keskimääräinen energia peittää niiden epäideaaliset vuorovaikutukset 
(hiukkasten välillä esiintyy nimittäin tietenkin esimerkiksi sähkömagneettisia, kaukovaikutteisia 
voimia).

En mene tässä ideaalikaasumalliin sen syvällisemmin, sillä se kuuluu varsinaisesti tilastolliseen 
termodynamiikkaan liittyvän \textit{kineettisen kaasuteorian}\footnote{Kineettinen kaasuteoria 
palauttaa kaasujen makroskooppiset ominaisuudet ne muodostavien hiukkasten liikkeisiin.} piiriin. 
Kineettisestä kaasuteoriasta saadaan kuitenkin ideaalikaasun kontrollitilavuuden $V$, kontrollitilavuuden 
``seinämiin'' kohdistuvan paineen $p$, hiukkasten lukumäärän $N$ ja kaasun lämpötilan $T$ välille 
seuraava yhteys:

\begin{equation}
 pV = Nk_BT
\end{equation}

Tätä sanotaan \textbf{ideaalikaasun tilanyhtälöksi}.

Yhtälön molempien puolten yksiköksi tulee itse asiassa joule. Tämä johtuu siitä, että ne ovat 
kumpikin verrannollisia kaasun hiukkasten translaatioliike-energiaan. Jo luvussa \nameref{def:H} 
näimme, että termi $pV$ liittyy nimenomaan systeemin rajoihinsa kohdistamaan paineeseen. Kaasuhiukkasten 
keskimääräinen translaation liike-energia taas on vapausasteiden määrä kertaa $\frac{1}{2}k_bT$:

\begin{equation}
 \overline{E}_{k,tr} = 3\cdot \frac{1}{2}k_bT = \frac{3}{2}k_bT
\end{equation}

Emme tietenkään yleensä tiedä hiukkasten lukumäärää kovin tarkasti eikä se kiinnostakaan. Onneksi 
ideaalikaasun tilanyhtälö saadaan helposti kätevämpään muotoon:

\begin{equation}
 pV = Nk_BT = nN_Ak_BT
\end{equation}

Nyt voimme määritellä Boltzmannin ja Avogadron vakioiden avulla \textit{yleisen kaasuvakion} $R_u$:

\begin{equation}
 R_u = N_Ak_B \approx 8,3145 \frac{J}{molK}
\end{equation}

Jolloin ideaalikaasun tilanyhtälö tulee ainakin kemiassa yleisimpään muotoonsa:

\begin{framed}
\begin{equation}
 pV = nR_uT
\end{equation}
\end{framed}

Moolimäärä usein tiedetään ja se on reaktioiden, liuosten jne. kannalta muutenkin olennainen.
Lämpötekniikassa se ei kuitenkaan aina ole olennainen ja massa on helpompi mitata, joten 
meidän tarkoituksiimme usein vielä kätevämpi muoto saadaan moolimassan määritelmän avulla:

\begin{align}
 &pV = nR_uT \quad \bigg|\bigg| \quad M = \frac{m}{n} \Leftrightarrow n = \frac{m}{M}\\
 &pV = \frac{m}{M}R_uT
\end{align}

Voimme edelleen määritellä \textit{kaasukohtaisen kaasuvakion} $R$\footnote{Kemiassa 
kaasukohtaista kaasuvakiota ei juurikaan esiinny, joten \underline{yleistä kaasuvakiota} merkitään 
vain $R$:llä.}:

\begin{framed}
\begin{equation}
\label{eq:pVmRT}
 pV = mRT
\end{equation}
\end{framed}

Aina ei massaakaan tiedetä tai muusta syystä on kätevintä toimia ominaissuureilla. Kun yhtälö 
\ref{eq:pVmRT} jaetaan puolittain massalla, saadaan ominaissuureille ideaalikaasun tilanyhtälöksi

\begin{framed}
\begin{equation}
\label{eq:pVmRT}
 pv = RT
\end{equation}
\end{framed}

Joskus saatetaan vielä korvata ominaistilavuus tiheydellä. Nehän ovat käänteislukuja ($v = \frac{1}{\rho}$):

\begin{equation}
\label{eq:pVmRT}
 \frac{p}{\rho} = RT
\end{equation}

\subsection{Reaalikaasujen tilanyhtälöitä}

\chapter{Termodynamiikan 0. pääsääntö} %----------------------------------------------------------------

Termodynamiikassa on neljä pääsääntöä. Ne ovat tämän tieteen keskeisimmät luonnonlait. Ensimmäisenä 
keksittiin tai oikeammin määriteltiin havaintojen pohjalta ensimmäinen pääsääntö, seuraavaksi tietenkin 
toinen. Nollas ja kolmas pääsääntö keksittiin tai pikemminkin nähtiin tarpeellisiksi määritellä vasta 
myöhemmin. 

Tästä johtuu se, että pääsääntöjen numerointi alkaa nollasta. Nollas 
pääsääntö haluttiin yksinkertaisempana ja perustavampana sijoittaa ennen ensimmäistä ja toista 
pääsääntöä, mutta näiden numerointi oli jo vakiintunut.\footnote{Nollasta alkava numerointi ei siis 
ole ohjelmoinnin tietorakenteiden indeksoinnin vaikutusta...}

\section{Teoria} %....................................................................

Nollas pääsääntö on seuraava, ehkä pedantin tuntuinen lausunto:

\begin{quote}
 Mikäli systeemit $A$ ja $B$ ovat termisessä tasapainossa keskenään ja systeemit $B$ ja $C$ ovat 
 termisessä tasapainossa keskenään, myös systeemit $A$ ja $C$ ovat termisessä tasapainossa keskenään.
\end{quote}

Mitä tällä lausunnolla sitten saavutetaan? Terminen tasapainohan tarkoitti sitä, että systeemien 
välillä ei siirry lämpöä. Termodynamiikan toisen pääsäännön seurauksena\footnote{Kuten Navier-Stokesin 
yhtälöt, myös termodynamiikan pääsäännöt ovat ``elliptisiä'' eli viittaavat kaikki toisiinsa eivätkä 
rakennu ainoastaan edellisten pääsääntöjen pohjalle.} lämpö siirtyy aina korkeammasta 
lämpötilasta matalampaan, minkä kääntöpuolena lämpöä ei siirry silloin, kun systeemien lämpötila 
on sama.

Kun terminen tasapaino ilmaistaan nyt lämpötilojen yhtäsuuruutena, saadaan termodynamiikan nollanneksi 
pääsäännöksi

\begin{quote}
 Mikäli systeemit $A$ ja $B$ ovat keskenään samassa lämpötilassa ja systeemit $B$ ja $C$ ovat 
 keskenään samassa lämpötilassa, myös systeemit $A$ ja $C$ ovat keskenään samassa lämpötilassa.
\end{quote}

Mikäli matemaattis-looginen ilmaisu tuntuu sinusta selkeämmältä, yllä oleva voidaan kirjoittaa 
sillä tavalla kompaktisti 

\begin{equation}
 T_A = T_B \wedge T_B = T_C \Rightarrow T_A = T_C
\end{equation}

\section{Käytäntö} %..................................................................

Varsinkin tuo viimeisin, matemaattis-looginen nollannen pääsäännön muotoilu tuntui itsestään 
selvältä. Mitä nollas pääsääntö oikeastaan määrittelee?

\paragraph{Se määrittelee lämpötilan mitattavana suureena.}

Mittaamme lämpömittareilla muiden systeemien lämpötiloja, mutta itse asiassa lämpömittari kertoo aina 
\textit{oman} lämpötilansa. Esimerkiksi kun perinteisen elohopea- tai alkoholilämpömittarin nestepatsas 
nousee, se johtuu nesteen lämpölaajenemisesta, mikä taas johtuu nesteen lämpötilan noususta.

Kuitenkin kun lämpömittarin annetaan vaihtaa lämpöä lämpötilamittauksen kohteena olevan systeemin (ja 
vain sen) kanssa riittävän pitkään, päädytään lopulta tilanteeseen, jossa ne ovat termisessä tasapainossa. 
Eli ylläolevan järkeilyn mukaan lämpömittarin lämpötila on sama kuin mittauksen kohteena olevan systeemin.

Lämpömittarit olisivat hyödyttömiä laitteita, mikäli ne antaisivat kahdelle samassa lämpötilassa olevalle 
systeemille eri lukeman tai käänteisesti kahdelle eri lämpötilassa olevalle systeemille saman lukeman. 

Termodynamiikan nollannen pääsäännön mukaan asian laita ei kuitenkaan ole näin onnettomasti, vaan mikäli 
systeemi $B$ on lämpömittari, se antaa (teoriassa\footnote{Käytännössähän lämpömittari ei välttämättä 
saavuta termistä tasapainoa mitattavan systeemin kanssa riittävän nopeasti tai kerro omaa lämpötilaansa 
luotettavasti.}) saman lukeman kummallekin samassa lämpötilassa olevalle systeemille $A$ ja $C$. 
Termodynamiikan nollas pääsääntö siis vakuuttaa ja varmistaa, että lämpömittarit ovat ainakin teoreettisesti 
luotettavia. 

\paragraph{Se määrittelee lämpötilan tilasuureena.}

Edellisen kokeellista tutkijaa helpottavan seikan lisäksi termodynamiikan nollas pääsääntö vahvistaa lämpötilan 
ylipäätään tilasuureena, myös laskelmissa käytettäväksi. Eli lämpötila kertoo jotain systeemin tilasta, 
nimellisesti sen, minkä systeemien kanssa systeemi voi vaihtaa lämpöä ja mihin suuntaan.

Nollas pääsääntö, kuten muutkin pääsäännöt, toimii ensisijaisesti klassisen termodynamiikan viitekehyksessä. 
Aiemmin lämpötilaa määritellessämmehän näimme, että lämpötilan syvällinen merkitys on se, että se kertoo 
systeemin energiajakaumien todennäköisimmät muodot.

\chapter{Termodynamiikan 1. pääsääntö} \label{def:td1} % -------------------------------------------

\section{Teoria} %....................................................................

\textbf{Termodynamiikan I pääsääntö} on seuraava kokeellisesti havaittu luonnonlaki:

\begin{quote}
 Energia säilyy.
\end{quote}

Sen voi ilmaista myös seuraavilla tavoilla:

\begin{quote}
 Energia on tuhoutumatonta
\end{quote}

\begin{quote}
 Energia ei tuhoudu, ainoastaan muuttaa muotoaan.
\end{quote}

\begin{quote}
 Maailmankaikkeuden kokonaisenergia on vakio.
\end{quote}

\begin{quote}
 On mahdotonta rakentaa laite, joka synnyttää maailmankaikkeuteen uutta energiaa. (Ns. tyypin I ikiliikkuja.)
\end{quote}

Termodynamiikan I pääsäännön voi määritellä myös käytännössä hyödyllisellä tavalla 
systeemin energioiden ja siirtymäenergioiden avulla. 

Mikäli energiaa siirtyy systeemiin, 
täytyy systeemin energian kasvaa. Mikäli energiaa siirtyy systeemistä, täytyy systeemin 
energian pienentyä. \hyperref[sec:merkkisopimus]{Merkkisopimuksen} ansiosta seuraava 
lause kattaa nämä molemmat tapaukset:

\begin{quote}
 Systeemin kokonaisenergian muutos = siirtymäenergioiden summa
\end{quote}

Siistissä täysin matemaattisessa kaavamuodossa termodynamiikan I pääsääntö on siis

\begin{equation}
 \Delta E_{sys} = Q_{tot} + W_{tot}
\end{equation}

Termodynamiikan pääsäännöt ovat kaikkien fysiikan lakien tavoin \textbf{universaaleja} 
eli voimassa kaikille systeemeille ja prosesseille kaikkialla, kaikkina aikoina. Voimme 
siis aina käyttää laskelmissamme termodynamiikan I pääsääntöä tässä muodossa yhtenä 
yhtälöistämme.

\subsection{Noetherin teoreema}

\section{Käytäntö} %....................................................................

\subsection{Suljettu systeemi}

\subsection{Avoin systeemi}

\subsection{Bernoullin yhtälö}

% Energiat, paineet, nostokorkeudet

\chapter{Termodynamiikan 2. pääsääntö} %--------------------------------------------------------------

\section{Teoria} %....................................................................

Ensimmäisen pääsäännön kannalta on samantekevää, onko energia liike- vai potentiaalienergiana tai 
siirtyykö se systeemistä toiseen työnä vai lämpönä, kunhan energian kokonaismäärä säilyy jokaisella 
ajanhetkellä.

Jo lämpötieteiden ja -tekniikan pioneerit kuitenkin huomasivat, että lämpöön ja työhön päti muitakin
lainalaisuuksia.

\paragraph{Lämmönsiirron suunta}

Havaittiin että lämpö siirtyi luonnollisesti vain tiettyyn suuntaan:

\begin{quote}
 Spontaanissa prosessissa lämpö siirtyy aina korkeammasta lämpötilasta matalampaan.
\end{quote}

Tämän lämpötekninen seuraus on:

\begin{quote}
 On mahdotonta rakentaa laitetta, joka siirtää lämpöä matalammasta lämpötilasta korkeampaan tekemättä 
 lainkaan työtä. (Eli jäähdytystekniikka kuluttaa väistämättä energiaa.)
\end{quote}

\paragraph{Lämmön ja työn muuttaminen toisikseen}

Havaittiin että työllä ja lämmöllä oli perustavanlaatuinen ero:

\begin{quote}
 Työ voidaan muuttaa kokonaan lämmöksi, mutta lämpöä ei kokonaan työksi.
\end{quote}

Tämän lämpötekninen seuraus on:

\begin{quote}
 On mahdotonta rakentaa laite, joka muuttaa lämmön kokonaan työksi. (Ns. tyypin II ikiliikkuja).
\end{quote}

\subsection{Entropia S}

Myöhemmin onnistuttiin kehittämään uusi tilasuure, \textbf{entropia S}, jonka avulla kaikki
edellämainitut lainalaisuudet voitiin lausua yhdellä lauseella:

\begin{quote}
 Eristetyn systeemin entropia kasvaa tai pysyy vakiona jokaisessa prosessissa.
\end{quote}

Tällä lauseella on erikoistapaus, joka tuntuu itse asiassa paljon kaikenkattavammalta:

\begin{quote}
 Maailmankaikkeuden entropia on ajan aidosti kasvava funktio\footnote{Eli dramaattisemmin ``entropia 
 kasvaa maailmanlopun edellä''. 
 
 Isälläni oli tapana sanoa ``pyy pienenee maailmanlopun 
 edellä''. Kun kysyin että miksi, niin vastaus oli ``entropian ja pyyn summa on vakio''. Arvelin, että 
 pyy on jokin kreikkalainen aakkonen koska tuntui siltä että entropiaa, mitä se sitten onkaan, tuskin 
 mitataan samoissa yksiköissä kuin pieniä lintuja. 
 
 Saattaa olla myös jokin satu, jossa säälimätön jumala kiroaa pyyt aina vain pienenemään sukupolvi 
 sukupolvelta. Oli miten oli, minusta kuitenkin tuli teekkari eikä ornitologia tai teologia.}.
\end{quote}

Tämä on seurausta siitä, että maailmankaikkeus on eristetty systeemi\footnote{Mikäli oletamme että sen 
sen ulkopuolella ei ole mitään. Toisaalta yhtä lailla voidaan olettaa, että siellä on toisia maailmankaikkeuksia 
eli multiversumi.} ja ajan myötä tapahtuu vain prosesseja, joissa sen entropia kasvaa tai pysyy vakiona.

Tuo ensimmäinen lause tunnetaan \textbf{termodynamiikan toisena pääsääntönä} ja käytännönläheisesti 
lausumme sen systeemiterminologialla ja matemaattisessa differentiaalimuodossa seuraavasti:

\begin{equation}
\label{eqn:psII}
 dS_{tot} = dS_{sys} + dS_{surr} \geq 0
\end{equation}

Eli maailmankaikkeuden kokonaisentropian, joka on systeemin ja ympäristön entropioiden summa, muutos on $\geq 0$. 
Tässä kokonaisentropian kasvu on ilmaistu entropian muutoksen avulla koska muutoksia on helpompi mitata 
ja myös käsitellä matemaattisesti kuin absoluuttisia arvoja.

Termodynamiikan II pääsääntö pätee aina ja kaikille prosesseille. Kaava \ref{eqn:psII} sisältää kaiken, 
mitä tässä luvussa on sitä ennen käsitelty.

% Todistus siitä, että entropia on tilasuure.
% Todistukset, että sisältää.

\paragraph{Mitä entropia on}

Entropia kehitettiin klassisen termodynamiikan työkaluilla. Todettiin että tämä uusi tilasuure on kätevä 
työkalu, koska se tiivistää havaitut lainalaisuudet lämmönsiirron suunnasta sekä työn ja lämmön suhteesta.

% Edellä todistettiin klassisen termodynamiikan teorialla, että entropia todella on tilasuure, ja että 
% termodynamiikan II pääsääntö todella sisältää nuo molemmat lainalaisuudet. 
Klassinen termodynamiikka 
ei kuitenkaan kerro mitään siitä, mitä entropia itse asiassa \textit{on}.

Entropiaa vaivaa sama ongelma kuin lämpötilaa, eli suuri osa sen käyttäjistä ei tiedä mitä se itse asiassa 
kuvaa. On vain jokin epämääräinen kuva siitä, että ``entropia on epäjärjestyksen mitta''. Tätä voidaan 
havainnollistaa vaikka sellaisella analogialla että ``keittiökin menee ajan mittaan epäjärjestykseen 
(mikäli järjestyksen pitämisessä ei nähdä suurta vaivaa)''. 

\textbf{Ei pidä paikkaansa, että entropia olisi suoranaisesti epäjärjestyksen mitta.} Koetapa kokata jonkun 
toisen keittiössä. Sinusta ehkä tuntuu, että kaikki löytyy mistä sattuu eli on epäjärjestyksessä, mutta keittiön 
omistajan mielestä kaikki voi olla juuri siellä missä pitääkin eli järjestyksessä. Järjestys on siis ihmisten 
keksintö ja määrittelykysymys. Moisten kanssa painiminen on \textit{``humanistien'' hommaa}.

\subsection{Kohti todennäköisintä tilaa}

\paragraph{Mitä entropia sitten todella mittaa?}

Tilastollisesta termodynamiikasta entropialle saadaan kuitenkin brutaalin yksinkertainen määritelmä ja kaava:

\begin{equation}
\label{eqn:S}
 S = k_B \ln W
\end{equation}

missä $W$ ei ole työ vaan sama kuin kappaleessa \nameref{sssection:T} eli niiden tapojen lukumäärä, jolla 
systeemin energiajakauma voi toteutua. $k_B$ on samaisessa kappaleessa ensi kertaa kohtaamamme Boltzmannin vakio.

Luonnollinen logaritmi entropian kaavassa on selitettävissä seuraavasti:

\begin{itemize}
 \item Olkoon meillä systeemi, joka koostuu osasysteemeistä 1 ja 2.
 \item Entropia on ekstensiivisuure jolloin $S_{sys} = S_1 + S_2$
 \item Toisaalta koko systeemin energiajakauma voi toteutua $W_{sys} = W_1W_2$ tavalla\footnote{Tämä ajattelutapa 
 on mahdollisesti tuttu todennäköisyyslaskennasta.}.
 \item Oletetaan että $S$ ja $\ln W$ ovat suoraan verrannollisia: $S = k \ln W$
 \item Systeemin entropian on tällöin oltava\\ $S_{sys} = k\ln W_{sys} = k\ln W_1W_2 = k \ln W_1 + k \ln W_2 = S_1 + S_2$
 \item Eli oletus toimii. Verrannollisuuskertoimen $k$ mitattua arvoa $k_B$ kutsutaan Boltzmannin 
	vakioksi\footnote{$k_B = 1,3806488\cdot10^{-23}\frac{J}{K}$}.
\end{itemize}

W on puhdas luku, joten entropian yksiköksi tulee

\begin{equation}
 [S] = [k_B\ln W] = [k_B] = \frac{J}{K} % Tää ei ehkää oo ihan OK tapa selvittää yksikkö!
\end{equation}

\paragraph{Mutta mitä entropia oikeastaan on?}

Verrannollisuuskerroin $k_B$ ja luonnollinen logaritmi esiintyvät entropian tilastollisessa määritelmässä vain 
jotta se saadaan täsmäämään klassisen termodynamiikan entropiaan. \textbf{Entropia on siis pohjimmiltaan vain 
W:n funktio:}

\begin{equation}
\label{eqn:SfW}
 S = f(W)
\end{equation}

Ja W tosiaan on niiden tapojen lukumäärä, jolla systeemin osasten energiajakauma voi toteutua. 

\paragraph{Ja miksi se aina kasvaa?}

Kaavoista \ref{eqn:S} ja \ref{eqn:SfW} nähdään että S on aidosti kasvava W:n funktio ja voi siis 
kasvaa vain kun W kasvaa. Miksi W sitten aina kasvaa? Koska energiajakauman todennäköisyys 
on suoraan verrannollinen siihen, kuinka monella eri tavalla kyseinen energiajakauma voi toteutua:

\begin{equation}
 P \sim W
\end{equation}

Ajan kuluessa sitä suurempi osa systeemeistä asettuu tiettyyn tilaan mitä todennäköisempi tuo tila on.
Eli systeemit asettuvat tiloihin, joiden P on mahdollisimman suuri. Ja P on suoraan verrannollinen W:hen 
jonka funktio entropia S on.

Termodynamiikan II pääsääntö tarkoittaa siis pohjimmiltaan seuraavaa:

\begin{quote}
 Maailmankaikkeus pyrkii kohti todennäköisintä tilaansa.
\end{quote}

Vaikuttaa aika itsestäänselvältä latteudelta, varsinkin kun vertaa siihen epämääräiseen kuvaan, joka 
minullakin ennen oli. Kyse ei olekaan siitä, että ``ajan kuluessa epäjärjestys kasvaa'' eli 
``maailmanlopun edellä kaikki menee päin helvettiä''\footnote{Itse asiassa kun entropia on saavuttanut 
maksiminsa ja lämmönsiirto loppunut on kaikkialla maailmankaikkeudessa sama muutaman kelvinin lämpötila. 
Kyllähän joku voisi sanoa että se on ``helvetin kylmä''...}.

% Srev ja [S]

\section{Käytäntö} %....................................................................

\subsection{Helmholtzin vapaaenergia F} \label{sssection:helmholtz}

Myös \textbf{Helmholtzin vapaaenergia} $F$ on apusuure, ja se on määritelty näin:

\begin{equation}
 F = U - TS
\end{equation}

Helmholtzin vapaaenergia on siis sisäenergian sekä entropian ja lämpötilan tulon erotus. Mutta mikä on sen 
käytännön merkitys?

\paragraph{Lämmönsiirto vakiolämpötilassa ja -tilavuudessa}

Olkoon meillä systeemi, josta tai johon siirtyy lämpöä. Prosessi tapahtuu niin, että systeemin lämpötila ja 
tilavuus ovat vakioita\footnote{``Isoterminen ja isokoorinen prosessi''.}. 
Koska tilavuus on vakio, edes tilavuudenmuutostyötä ei tapahdu:

\begin{align}
 T(t) & = T\\
 V(t) & = V\\
 \Delta U & = Q_{siirto} + W = Q_{siirto}
\end{align}

eli 

\begin{equation}
\label{eqn:isoTV}
 \Delta U = Q_{siirto}
\end{equation}

Systeemiin lämpönä tuotu energia menee siis systeemin sisäenergian (molekyylien kineettinen energia) kasvattamiseen.

Lämpöä voi tulla systeemiin kahdesta lähteestä; systeemin ulkopuolelta siirtymällä ja sen sisältä kun jokin muu 
energia muuttuu epäjärjestyneeseen muotoon (esimerkiksi turbulenssi muuttaa molekyylien ``koordinoitunutta'' 
liike-energiaa satunnaiseksi eli lämpöliikkeen liike-energiaksi).

\begin{equation}
\label{eqn:QQQ}
 Q = Q_{siirto} + Q_{h\ddot{a}vi\ddot{o}t}
\end{equation}

Kokonaislämpö voidaan lausua entropian muutoksen kautta entropian ja lämpötilan avulla ja
koska lämpötila on vakio:

\begin{equation}
\label{eqn:DeltaS}
 \Delta S = \frac{Q}{T} \Leftrightarrow Q = T\Delta S
\end{equation}

Sijoitetaan kaava \ref{eqn:DeltaS} kaavaan \ref{eqn:QQQ}:

\begin{equation}
 T\Delta S = Q_{siirto} + Q_{h\ddot{a}vi\ddot{o}t}
\end{equation}

Häviöt synnyttävät aina positiivisen lämpömäärän, joten

\begin{align}
 Q_{h\ddot{a}vi\ddot{o}t} & \geq 0\\
 \Rightarrow T\Delta S & \geq Q_{siirto}\\
 \label{eqn:QTSsiirto}
 \Leftrightarrow Q_{siirto} & \leq T \Delta S
\end{align}

Sijoitetaan kaava \ref{eqn:QTSsiirto} kaavaan \ref{eqn:isoTV}:

\begin{equation}
 \Delta U = Q_{siirto} \leq T\Delta S
\end{equation}

eli 

\begin{align}
 & \Delta U \leq T\Delta S\\
 & \Delta U - T\Delta S \leq 0\\
 & \Delta(U - TS) \leq 0\\
 & \Delta F \leq 0
\end{align}

Siis kun

\begin{align}
 T(t) = T\\
 V(t) = V\\
 W = 0
\end{align}

On Helmholtzin vapaaenergian muutoksen oltava

\begin{equation}
 \Delta F \leq 0
\end{equation}

\paragraph{Fysikaalinen ja kemiallinen merkitys}

Kun systeemi on vakiolämpötilassa ja -tilavuudessa eikä sisäisiä häviöitä tapahdu, on 

\begin{equation}
 \Delta F = 0
\end{equation}

Ja systeemi on termodynaamisessa tasapainossa.

Mikäli

\begin{equation}
 \Delta F < 0
\end{equation}

tarkoittaa se sitä, että systeemissä muut energian muodot muuttuvat häviöiden kautta lämmöksi 
kunnes systeemi on termodynaamisessa tasapainossa.

\subsection{Gibbsin vapaaentalpia G}

Kuten entalpia ja Helmholtzin vapaaenergia, \textbf{Gibbsin vapaaentalpia} $G$ on apusuure, 
ja se on määritelty näin:

\begin{equation}
 G = H - TS
\end{equation}

Gibbsin vapaaentalpian määritelmä näyttää hyvin samankaltaiselta kuin Helmholtzin vapaaenergian määritelmä. 
Miksiköhän?

\paragraph{Lämmönsiirto vakiolämpötilassa ja -paineessa}

Huomaa, että seuraava päättely etenee hyvin samalla tavalla kuin Helmholtzin vapaaenergian tapauksessa.

Siirretään lämpöä systeemistä tai systeemiin vakiolämpötilassa ja -paineessa:

\begin{align}
 T(t) & = T\\
 p(t) & = p\\
 \Delta U & = Q_{siirto} + W
\end{align}

Nyt tilavuudenmuutostyötä voi esiintyä ja analyysi voisi mennä monimutkaiseksi, muttei mene jos muistamme 
kappaleessa \nameref{def:H} saamaamme tulosta, eli että vakiopaineessa:

\begin{equation}
 \Delta H = Q_{siirto}
\end{equation}

ja kuten kappaleessa \nameref{sssection:helmholtz} päättelimme

\begin{equation}
 Q_{siirto} \leq T \Delta S
\end{equation}

Kun yhdistämme nämä kaavat, saamme

\begin{equation}
 \Delta H = Q_{siirto} \leq T\Delta S
\end{equation}

Ja edelleen: 

\begin{align}
 & \Delta H \leq T\Delta S\\
 & \Delta H - T\Delta S \leq 0\\
 & \Delta(H - TS) \leq 0\\
 & \Delta G \leq 0
\end{align}

Siis kun

\begin{align}
 T(t) = T\\
 p(t) = p\\
 W = 0
\end{align}

On Gibbsin vapaaentalpian muutoksen oltava

\begin{equation}
 \Delta G \leq 0
\end{equation}

\paragraph{Fysikaalinen ja kemiallinen merkitys}

Kun systeemi on vakiolämpötilassa ja -paineessa eikä sisäisiä häviöitä tapahdu, on 

\begin{equation}
 \Delta G = 0
\end{equation}

Ja systeemi on termodynaamisessa tasapainossa.

Mikäli

\begin{equation}
 \Delta G < 0
\end{equation}

tarkoittaa se sitä, että systeemissä muut energian muodot muuttuvat häviöiden kautta lämmöksi 
kunnes systeemi on termodynaamisessa tasapainossa.

\chapter{Termodynamiikan 3. pääsääntö} %--------------------------------------------------------------

Ensimmäisen pääsäännön käyttöönotto vaati sisäenergian määrittelemisen suureena, mutta se käsitteli vain 
sisäenergian muutoksia eikä vaatinut sille absoluuttista arvoa vaan referenssitaso voitiin valita 
vapaasti. Sisäenergialle saadaan kuitenkin absoluuttinen arvo, kun asetetaan se nollaksi 
absoluuttisessa nollapisteessä. Aineella on toki tällöin epätarkkuusperiaatteen mukaan 
jäännösenergiaa\footnote{Ns. nollapiste-energia, joka on ikiliikkujien rakentajien modernien vastineiden 
suuressa suosiossa.}, mutta koska sitä on kovin vaikea saada aineesta poistettua -- ainakaan lämpöä 
poistamalla se ei onnistu -- joten tässä sisäenergian nollaamisessa ei ole normaalisti juurikaan riskiä.

Vastaavasti toisen pääsäännön määritteleminen hyvin vaati entropian käyttöönottoa suureena, mutta pääsääntö 
käsitteli vain sen muutoksia absoluuttisten arvojen sijaan. Kolmas pääsääntö määrittelee entropialle 
absoluuttisen arvon. Toisaalta siitä seuraa, että absoluuttista nollapistettä ei voida saavuttaa ainakaan 
millään ilmeisellä tavalla.

\section{Teoria} %....................................................................

Termodynamiikan kolmas pääsääntö kuuluu seuraavasti:

\begin{quote}
 Jokaisen puhtaan, kiderakenteeltaan virheettömän kristallimaisen aineen entropia on 
 nolla absoluuttisessa nollapisteessä.
\end{quote}

Eli matemaattisessa muodossa

\begin{equation}
 S_{crystal}(0 K) = 0
\end{equation}

Syynä tähän on se, että virheetön kiderakenne voidaan järjestää vain yhdellä tavalla - muut tavathan 
ovat virheellisiä. Niinpä

\begin{equation}
 S_{kiderakenne}(0K) = k_B \ln W_{kiderakenne}(0K) = k_b \ln 1 = 0
\end{equation}

Lisäksi absoluuttisessa nollapisteessä kiderakenteella voi olla vain mitätön 
määrä\footnote{Nollapiste-energian suuruinen.} liike-energiaa. Jos liike-energian määrä pyöristetään 
nollaan, myös tällöin syntyvä askelfunktiomainen energiajakauma (Fermi-Dirac-jakauma) voi toteutua 
vain yhdellä tavalla eli niin, että jokaisella kristallin yksikkökopilla on sama, potentiaalienergiasta 
koostuva kvantittunut energia. Niinpä

\begin{equation}
 S_{energiajakauma}(0K) = k_B \ln W_{energiajakauma}(0K) = k_b \ln 1 = 0
\end{equation}

ja edelleen

\begin{equation}
 S_{crystal}(0K) = S_{kiderakenne}(0K) + S_{energiajakauma}(0K) = 0 + 0 = 0
\end{equation}

% \paragraph{Absoluuttisen nollapisteen saavuttamattomuus}
% 
% seuraa termodynamiikan kolmannesta pääsäännöstä. Nimittäin jos meillä on systeemi hyvin pienessä 
% lämpötilassa $T_1$ ja ympäristö lämpötilassa $T_2$ ja
% 
% \begin{equation}
%  T_1 < T_2
% \end{equation}
% 
% Koska systeemin on annettu mennä termiseen tasapainoon ympäristön\footnote{Joka voi olla 
% esim. nestemäistä heliumia.} kanssa ($T_1 = T_2$) ja sen 
% jälkeen sitä on alettu jäähdyttää lämpöpumpulla tai vastaavalla järjestelyllä\footnote{Siis 
% käytännössä jollakin kiertoprosessilla, tyypillisesti erittäin matalien lämpötilojen tavoittelussa 
% käytetään \textit{adiabaattista demagnetaatiota}, jonka selittäminen tässä ei ole olennaista.}.
% 
% Jotta 

\section{Käytäntö} %..................................................................

Käytännössä termodynamiikan kolmas pääsääntö ei ole kovin merkittävä kuin matalan lämpötilan 
maailmanennätyksen tavoittelijoille\footnote{Tämänhetkinen (\today) ennätys on ilmeiseti luokkaa $10^{-8}K$}. 

Teorian suhteen kolmannella päännöllä on myös laskelmille merkittäviä seurauksia, se nimittäin 
tekee ideaalikaasun todellisen olemassaolon mahdottomaksi. Toisaalta jo ideaalikaasun nimestä käy ilmi, 
ettei sen arvella vastaavan todellisuutta vaan täydellisyyttä.

Virheettömiä kristallejakaan ei tietenkään ole olemassa, niinkuin ei mitään täydellistä. Kiderakenteeseen 
tulee nimittäin aina virheitä, kun aineita jäähdytetään nopeammin kuin äärettömän hitaasti. Lasin 
ja kvartsin ero on nimenomaan se, että lasi on jäähtynyt niin nopeasti, ettei silä ole lainkaan säännöllistä 
kiderakennetta. Mutta kvartsikiteetkään eivät ikinä ole täysin virheettömiä.

\part*{Liitteet} %___________________________________________________________________________________
\addcontentsline{toc}{part}{Liitteet}
\setcounter{chapter}{0}

\chapter{Lähdeluettelo} %----------------------------------------------------------------------------

\bibliography{pruju}{}
\bibliographystyle{plain}

\end{document}

