% Copyright 2013 Pauli Jaakkola
% 
% This program is free software: you can redistribute it and/or modify
% it under the terms of the GNU General Public License as published by
% the Free Software Foundation, either version 3 of the License, or
% (at your option) any later version.
% 
% This program is distributed in the hope that it will be useful,
% but WITHOUT ANY WARRANTY; without even the implied warranty of
% MERCHANTABILITY or FITNESS FOR A PARTICULAR PURPOSE.  See the
% GNU General Public License for more details.
% 
% You should have received a copy of the GNU General Public License
% along with this program.  If not, see <http://www.gnu.org/licenses/>.

\documentclass[12pt,a4paper,finnish]{article}

\usepackage[utf8]{inputenc}                 % Tekstiasetuksia, sisältää ääkköset
\usepackage[T1]{fontenc}                    % Tekstiasetuksia, T1-koodatut fontit
\usepackage{ae,aecompl}                     % Paremman näköiset fontit
\usepackage[finnish]{babel}                 % Suomenkielinen tavutus ja otsikot
% \usepackage{a4wide}
\usepackage[intlimits]{amsmath}             % Lisää kaavavoimaa!
\usepackage{amssymb} 
\usepackage{fixltx2e}                       % \textsubscript
\usepackage{enumerate}
\usepackage{framed}
\usepackage{hyperref}
\makeatother
\hypersetup{
  colorlinks=true,
  linkcolor=blue,
}

\renewcommand{\thesection}{}
\renewcommand{\thesubsection}{}
\makeatletter
\def\@seccntformat#1{\csname #1ignore\expandafter\endcsname\csname the#1\endcsname\quad}
\let\sectionignore\@gobbletwo
\let\latex@numberline\numberline
\def\numberline#1{\if\relax#1\relax\else\latex@numberline{#1}\fi}
\makeatother

\title{LTP++: Lämmönsiirto\\Laskuharjoituksen 11 ratkaisut}
\date{\today}
\author{Pauli Jaakkola}

\begin{document}

\maketitle
\tableofcontents
\newpage

\section{Tehtävä 18}

\begin{framed}
 \begin{equation}
   \rho c\frac{\partial T}{\partial t} = k\nabla^2T + \dot{Q}_v'''
 \end{equation}
\end{framed}

Lämmön varastoitumista sisäenergiaksi tai lämmön tuotantoa ei tapahdu:

\begin{align}
 &\rho c\frac{\partial T}{\partial t} = 0\\
 &\dot{Q}_v''' = 0
\end{align}

Joten

\begin{equation}
  k\nabla^2T = 0
\end{equation}

\begin{framed}
 \begin{equation}
   \nabla^2T = 0
 \end{equation}
\end{framed}

Ns. \textit{Laplacen yhtälö}. Sylinterikoordinaateissa $\nabla^2$ eli \textit{Laplacen operaattori}:

\begin{equation}
 \nabla^2 = \left(\frac{1}{r}\frac{\partial}{\partial r}\left(r\frac{\partial}{\partial r}\right)
   + \frac{1}{r^2}\frac{\partial^2}{\partial \varphi^2} + \frac{\partial^2}{\partial z^2}\right)
\end{equation}

\begin{align}
 &\left(\frac{1}{r}\frac{\partial}{\partial r}\left(r\frac{\partial}{\partial r}\right)
   + \frac{1}{r^2}\frac{\partial^2}{\partial \varphi^2} + \frac{\partial^2}{\partial z^2}\right)T = 0\\
 &\frac{1}{r}\frac{\partial}{\partial r}\left(r\frac{\partial T}{\partial r}\right)
   + \frac{1}{r^2}\frac{\partial^2T}{\partial \varphi^2} + \frac{\partial^2T}{\partial z^2} = 0
\end{align}

Nyt ratkaistaan säteen suhteen eli oletetaan ettei jakauma muutu $z$:n tai kulman $\varphi$ funktiona. 
Tällöin derivaatat niiden suhteen $= 0$ ja derivaatta $r$:n suhteen on kokonaisderivaatta:

\begin{align}
 &\frac{1}{r}\frac{\partial}{\partial r}\left(r\frac{\partial T}{\partial r}\right) = 0\\
 &\frac{d}{d r}\left(r\frac{d T}{d r}\right) = 0
\end{align}

Separoidaan ja integroidaan kahdesti:

\begin{align}
 &\int d\left(r\frac{d T}{d r}\right) = \int 0 dr\\
 &r\frac{d T}{d r} = C_1\\
 &\frac{d T}{d r} = \frac{C_1}{r} \label{eq:int18}\\
 &\int dT = \int\frac{C_1}{r}dr\\
 &T(r) = C_1\ln r + C_2
\end{align}

Kaksi vakiota $\Rightarrow$ kaksi reunaehtoa. Nyt ne ovat tunnetut sisä- ja ulkolämpötilat:

\begin{align}
 &\left\{
 \begin{aligned}
  &T(r_1) = T_1\\
  &T(r_2) = T_2
 \end{aligned}\right.\\
 &\left\{
 \begin{aligned}
  &T(r_1) = C_1\ln r_1 + C_2 = T_1\\
  &T(r_2) = C_1\ln r_2 + C_2 = T_2
 \end{aligned}\right.\\
 &\left\{
 \begin{aligned}
  &C_2 = T_1 - C_1\ln r_1\\
  &C_2 = T_2 - C_1\ln r_2
 \end{aligned}\right.\\
 &T_1 - C_1\ln r_1 = T_2 - C_1\ln r_2\\
 &C_1(\ln r_1 - \ln r_2) = C_1\ln \frac{r_1}{r_2} = T_2 - T_1\\
 &C_1 = \frac{T_2 - T_1}{\ln \frac{r_1}{r_2}} \label{eq:C_1}\\
 &C_2 = T_1 - \frac{T_2 - T_1}{\ln \frac{r_1}{r_2}}\ln r_1
\end{align}

Sijoitetaan takaisin lämpötilajakaumaan:

\begin{equation}
 T(r) = \frac{T_2 - T_1}{\ln \frac{r_1}{r_2}}\ln r + T_1 - \frac{T_2 - T_1}{\ln \frac{r_1}{r_2}}\ln r_1
\end{equation}

\begin{framed}
  \begin{equation}
  T(r) = T_1 + \frac{T_2 - T_1}{\ln \frac{r_1}{r_2}}\ln \frac{r}{r_1}
  \end{equation}
\end{framed}

\begin{equation}
 \label{eq:dotQ}
 \dot{Q} = A_v\dot{Q}'' = A_vq = PLq
\end{equation}

Kokeellinen \textit{Fourierin lämmönjohtumisyhtälö}:

\begin{equation}
 \label{eq:Fourier}
 q = q_r = -k\frac{\partial T}{\partial r} = -k\frac{dT}{dr}
\end{equation}

Sijoitetaan yhtälö \ref{eq:Fourier} yhtälöön \ref{eq:dotQ}:

\begin{align}
 &\dot{Q} = PLq = -PLk\frac{dT}{dr}\\
 &\dot{Q}' = -Pk\frac{dT}{dr}
\end{align}

\begin{align}
 &P = 2\pi r\\
 &\frac{d T}{d r} = \frac{C_1}{r} = \frac{T_2 - T_1}{\ln \frac{r_1}{r_2}r}
  \quad \bigg|\bigg| \quad \text{Yhtälöt \ref{eq:int18} ja \ref{eq:C_1}}
\end{align}

\begin{framed}
 \begin{equation}
  \dot{Q}' = -2\pi k\frac{T_2 - T_1}{\ln \frac{r_1}{r_2}}
 \end{equation}
\end{framed}

\section{Tehtävä 20}

\begin{math}
 \begin{array}{l l l l}
  d_1 = 0.010 m & d_2 = 0.020 m & r_1 = \frac{d_1}{2} = 0.005m & r_2 = \frac{d_2}{2} = 0.010m\\
  T_{\infty} = 293,15K& \dot{Q}' = 200 \frac{W}{m} & k = 1.6 \frac{W}{mK}
 \end{array}
\end{math}\\

Edellisestä tehtävästä saamme johtumisen lämpövirraksi pituutta kohden

\begin{equation}
  \dot{Q}_j' = -2\pi k\frac{T_2 - T_1}{\ln \frac{r_1}{r_2}}
\end{equation}

Lämmönsiirto eristeestä ilmaan on konvektiivista eli ilmaistaan \textit{konvektiivisen lämmönsiirtokertoimen} 
$h_c$ avulla:

\begin{align}
 &\dot{Q}_c = A_v\dot{Q}'' = A_vq = PLq \quad \bigg|\bigg| \quad q = -h_c(T_{\infty} - T_2)\\
 &\dot{Q}_c = -PLh_c(T_{\infty} - T_2)\\
 &\dot{Q}'_c = \frac{\dot{Q}_c}{L} = -2\pi r_2 h_c(T_{\infty} - T_2)
\end{align}

\subsection{a}

\begin{align}
 &\left\{
 \begin{aligned}
  &\dot{Q}_j' = -2\pi k\frac{T_2 - T_1}{\ln \frac{r_1}{r_2}}\\
  &\dot{Q}_c' = -2\pi r_2 h_c(T_{\infty} - T_2)\\
  &\dot{Q}_j = \dot{Q}_c
 \end{aligned}\right.\\
 &\left\{
 \begin{aligned}
  &T_1 = T_2 + \frac{\dot{Q}_j'}{2\pi k}\ln \frac{r_1}{r_2}\\
  &T_2 = T_{\infty} + \frac{\dot{Q}_c'}{2\pi r_2 h_c}\\
  &\dot{Q}_j' = \dot{Q}_c' = \dot{Q}'
 \end{aligned}\right.\\
 &\left\{
 \begin{aligned}
  &T_1 = T_2 + \frac{\dot{Q}'}{2\pi k}\ln \frac{r_1}{r_2}\\
  &T_2 = T_{\infty} + \frac{\dot{Q}_c'}{2\pi r_2 h_c} 
   \approx \lim_{h_c \to \infty}\left( T_{\infty} + \frac{\dot{Q}'}{2\pi r_2 h_c}\right) = T_{\infty}
 \end{aligned}\right.\\
  &T_1 \approx T_{\infty} + \frac{\dot{Q}'}{2\pi k}\ln \frac{r_1}{r_2} \approx \underline{33,8\,^{\circ}C}
\end{align}

\subsection{b}

\begin{center}
  \begin{math}
  h_c = 20\frac{W}{m^2}
  \end{math}
\end{center}

\begin{align}
  &T_2 = T_{\infty} + \frac{\dot{Q}_c'}{2\pi r_2 h_c} \approx \underline{179,2\,^{\circ}C}\\
  &\left(T_1 = T_2 + \frac{\dot{Q}'}{2\pi k}\ln \frac{r_1}{r_2} \approx 193\,^{\circ}C\right)
\end{align}

\subsection{a}

\begin{align}
 &\left\{
 \begin{aligned}
  &q = \dot{Q}'' = -h_c(T_{\infty} - T_2) - \varepsilon\sigma(T_{\infty}^4 - T_2^4)\\
  &\dot{Q}_{cr}' = Pq = \dot{Q}_j = \dot{Q}
 \end{aligned}\right.\\
  &\dot{Q}' = P(-h_c(T_{\infty} - T_2) - \varepsilon\sigma(T_{\infty}^4 - T_2^4))
\end{align}

\begin{framed}
 \begin{equation}2\pi r_2(h_c(T_2 - T_{\infty}) + \varepsilon\sigma(T_2^4 - T_{\infty}^4)) - \dot{Q}' = 0
 \end{equation}
\end{framed}

\section{Tehtävä 47}

Olkoon systeemimme ajanjaksossa $dt$ jäätyvä vesi $dm_j$. Kyseessä on siis lämmönsiirto vakiopaineessa, jolle
termodynamiikan 1. pääsääntö on

\begin{align}
 &dH_j = dQ\\
 &dH_{sf, j} = dQ\\
 &h_{sf, j}dm_j = dQ
\end{align}

Tämä tapahtuu ajanjaksossa $dt$, joten lämpövirta (= lämpöteho) saadaan jakamalla $dt$:llä:

\begin{align}
 &h_{sf, j}\frac{dm_j}{dt} = \frac{dQ}{dt}\\
 &\dot{Q} = h_{sf, j}\frac{dm_j}{dt} = \rho Ah_{sf, j}\frac{dx}{dt}\\
 &\frac{\dot{Q}}{A} = \dot{Q}'' = q = \rho h_{sf, j}\frac{dx}{dt}
\end{align}

Jäässä lämpö siirtyy johtumalla, joten lämpövirran tiheys saadaan Fourierin lämmönjohtumisyhtälöstä:

\begin{equation}
 q = -k\frac{dT}{dx}
\end{equation}

$x(t)$ on jään paksuus jollakin hetkellä $t$. Oletetaan että lämpöä ei varastoidu jäähän (ns. 
\textit{jatkuvuustila}), jolloin sen kaiken pitää johtua jään läpi. Tällöin voidaan myös 
olettaa, että lämpötilajakauma on joka hetki lineaarinen:

\begin{align}
 &q = \rho h_{sf, j}\frac{dx}{dt} = -k\frac{dT}{dx} \approx -k\frac{T_{sf} - T_s}{x(t)}\\
 &x(t)\frac{dx}{dt} = \frac{k}{\rho h_{sf, j}}(T_s - T_{sf})\\
 &\int_0^t x(t)dx = \int_{x_0}^x(t) \left(\frac{k}{\rho h_{sf, j}}(T_s - T_{sf}) \right) dt
\end{align}

Matemaattisesti $x(t)$ sekä funktiona että ylärajana on kyseenalainen, mutta merkinnät 
muuttuvat vain sekavammiksi jos ne yrittää erottaa. Integroidaan siis alkuhetkestä $0$, jolloin 
jään paksuus on $x_0$ mielivaltaiseen ajanhetkeen $t$, jolloin jään paksuus on $x(t)$.

\begin{equation}
 \frac{1}{2}x^2(t) - \frac{1}{2}x_0^2 = \frac{k}{\rho h_{sf, j}}(T_s - T_{sf}) t
\end{equation}

\begin{framed}
 \begin{equation}
  x(t) = \sqrt{\frac{2k}{\rho h_{sf, j}}(T_s - T_{sf}) t + x_0^2}
 \end{equation}
\end{framed}

Parametrit tehtävänannon tilanteelle:\\

\begin{math}
 \begin{array}{l l l l}
  k = 2,2\frac{W}{mK} & \rho = 913\frac{kg}{m^3} & h_{sf} = 330 \frac{kJ}{kg} &\\
  T_s = -10\,^{\circ}C & T_{sf} = 0\,^{\circ}C & x_0 = 0,30 m & t_1 = 1d = 86400 s
 \end{array}
\end{math}

\begin{align}
 \Delta x = x(t_1) - x_0 = \sqrt{\frac{2k}{\rho h_{sf, j}}(T_s - T_{sf}) t_1 + x_0^2} - x_0 
   \approx \underline{20 mm}
\end{align}


\end{document}