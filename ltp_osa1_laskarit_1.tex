% Copyright 2013 Pauli Jaakkola
% 
% This program is free software: you can redistribute it and/or modify
% it under the terms of the GNU General Public License as published by
% the Free Software Foundation, either version 3 of the License, or
% (at your option) any later version.
% 
% This program is distributed in the hope that it will be useful,
% but WITHOUT ANY WARRANTY; without even the implied warranty of
% MERCHANTABILITY or FITNESS FOR A PARTICULAR PURPOSE.  See the
% GNU General Public License for more details.
% 
% You should have received a copy of the GNU General Public License
% along with this program.  If not, see <http://www.gnu.org/licenses/>.

\documentclass[12pt,a4paper,finnish]{article}

\usepackage[utf8]{inputenc}                 % Tekstiasetuksia, sisältää ääkköset
\usepackage[T1]{fontenc}                    % Tekstiasetuksia, T1-koodatut fontit
\usepackage{ae,aecompl}                     % Paremman näköiset fontit
\usepackage[finnish]{babel}                 % Suomenkielinen tavutus ja otsikot
% \usepackage{a4wide}
\usepackage[intlimits]{amsmath}             % Lisää kaavavoimaa!
\usepackage{fixltx2e}                       % \textsubscript
\usepackage{enumerate}
% \usepackage{framed}
\usepackage{hyperref}
\makeatother
\hypersetup{
  colorlinks=true,
  linkcolor=blue,
}

\title{LTP++: Termodynamiikka\\Laskuharjoituksen 1 ratkaisut}
\date{\today}
\author{Pauli Jaakkola}

\begin{document}

\maketitle

\section{Tehtävä 2}

\begin{math}
\begin{aligned}
 &p = 1bar = 10^5 Pa\\
 &T = 20\,^{\circ}C = 293(,15)K\\
 &M = 28,965 g/mol\\
 &R_u = 8,314 J/molK
\end{aligned}
\end{math}

\begin{equation}
 pV = nR_uT
\end{equation}

\begin{equation}
 M = \frac{m}{n} \Leftrightarrow n = \frac{m}{M}
\end{equation}

\begin{equation}
 pV = \frac{m}{M}R_uT
\end{equation}
 
\begin{equation}
R = \frac{R_u}{M}
\end{equation}

\begin{equation}
 pV = mRT
\end{equation}

\begin{equation}
 p\frac{V}{m} = RT
\end{equation}

\begin{equation}
v = \frac{V}{m}
\end{equation}

\begin{equation}
\label{eq:pvRT}
 pv = RT
\end{equation}

\begin{equation}
 v = \frac{V}{m} = \frac{1}{\frac{m}{V}} = \frac{1}{\rho}
\end{equation}

\begin{equation}
 \frac{p}{\rho} = RT
\end{equation}

\begin{equation}
 \rho = \frac{p}{RT}
\end{equation}

\begin{equation}
 R = \frac{R_u}{M} \approx 287,036 \frac{J}{kgK}
\end{equation}

\begin{equation}
 \rho = \frac{p}{RT} \approx \underline{1,189 \frac{kg}{m^3}}
\end{equation}

\section{Tehtävä 3}

\subsection{a}

Työ saadaan muutettua kokonaan lämmöksi, mutta vain osa lämmöstä saadaan muutettua työksi.

\subsection{b}

\begin{math}
 \begin{aligned}
  &m = 1 kg\\
  &c_p = 4200 \frac{J}{kgK}\\
  &\Delta T = 1\,^{\circ}C = 1 K
 \end{aligned}
\end{math}

\begin{equation}
 \Delta E = Q + W = Q
\end{equation}

\begin{equation}
 Q = \Delta E = \Delta U = m\Delta u = mc_v\Delta T
\end{equation}

\begin{equation}
 c_v = c_p
\end{equation}

\begin{equation}
 Q = mc_p\Delta T \approx 4200 J = \underline{4,2 kJ}
\end{equation}

\subsection{c}

\begin{math}
 \begin{aligned}
  &m = 1 kg\\
  &W = Q_{b-kohta} = 4200 J
 \end{aligned}
\end{math}

\begin{equation}
 \Delta E = Q + W = W
\end{equation}

\begin{equation}
 \Delta E = \Delta E_k = W
\end{equation}

\begin{equation}
 \Delta E_k = \Delta \left(\frac{1}{2}m|\vec{v}|^2\right) = \frac{1}{2}m\Delta\left(|\vec{v}|^2\right) = 
  \frac{1}{2}m\left(|\vec{v_2}|^2 -|\vec{v_1}|^2\right) = W
\end{equation}

\begin{equation}
 \frac{1}{2}m\left(|\vec{v_2}|^2 -|\vec{v_1}|^2\right) = \frac{1}{2}m\left(|\vec{v_2}|^2 -|\vec{0}|^2\right)
  = W
\end{equation}

\begin{equation}
 \frac{1}{2}m|\vec{v_2}|^2= W
\end{equation}

\begin{equation}
 |\vec{v_2}| = \sqrt{\frac{2W}{m}} \approx \underline{91,65 \frac{m}{s}}
\end{equation}

\subsection{d}

\begin{math}
 \begin{aligned}
  &m = 1 kg\\
  &W = W_{c-kohta} = 4200 J\\
  &g = 9,81 \frac{m}{s^2}
 \end{aligned}
\end{math}

\begin{equation}
 \Delta E = Q + W = W
\end{equation}

\begin{equation}
 \Delta E = \Delta E_p = W
\end{equation}

\begin{equation}
 \Delta E_p = mg\Delta h = W
\end{equation}

\begin{equation}
 \Delta h = \frac{W}{mg} \approx \underline{428,3 m}
\end{equation}


\end{document}